% arara: pdflatex
% arara: pdflatex
% arara: pdflatex

% options:
% thesis=B bachelor's thesis 
% thesis=M master's thesis
% czech thesis in Czech language 
% slovak thesis in Slovak language
% english thesis in English language
% hidelinks remove colour boxes around hyperlinks 
 
\documentclass[thesis=B,czech]{FITthesis}[2019/12/23]
  
\usepackage[utf8]{inputenc} % LaTeX source encoded as UTF-8
\usepackage{adjustbox}   

% \usepackage{amsmath} %advanced maths
% \usepackage{amssymb} %additional math symbols
 
\usepackage{dirtree} %directory tree visualisation
\usepackage{hyperref} 
   
% % list of acronyms
% \usepackage[acronym,nonumberlist,toc,numberedsection=autolabel]{glossaries}
% \iflanguage{czech}{\renewcommand*{\acronymname}{Seznam pou{\v z}it{\' y}ch zkratek}}{}
% \makeglossaries

\newcommand{\tg}{\mathop{\mathrm{tg}}} %cesky tangens
\newcommand{\cotg}{\mathop{\mathrm{cotg}}} %cesky cotangens
\newcommand\tstrut{\rule{0pt}{2.4ex}}
\newcommand\bstrut{\rule[-1.0ex]{0pt}{0pt}}

% % % % % % % % % % % % % % % % % % % % % % % % % % % % % % 
% ODTUD DAL VSE ZMENTE
% % % % % % % % % % % % % % % % % % % % % % % % % % % % % % 

\department{Katedra softwarového inženýrství}
\title{IoT platforma s webovým rozhraním}
\authorGN{Martin} %(křestní) jméno (jména) autora
\authorFN{Skalický} %příjmení autora
\authorWithDegrees{Martin Skalický} %jméno autora včetně současných akademických titulů
\author{Martin Skalický} %jméno autora bez akademických titulů
\supervisor{Ing. Jiří Mlejnek}
%\acknowledgements{Doplňte, máte-li komu a za co děkovat. V~opačném případě úplně odstraňte tento příkaz.}
\abstractCS{Ze slova IOT se v posledních letech stal buzzword, pod kterým si lze představit téměř cokoliv od chytré žárovky až po automatizovanou linku. IOT platforma označuje  platformu, ke které lze připojit chytré věci jako např. chytrou žárovku či teplotní senzor a následně data ze zařízení zobrazovat, analyzovat a vzdáleně ovládat mimo jiné ze svého telefonu. Tato práce se zabývá porovnáním aktuálních platforem na trhu pro domácnosti a následným návrhem vlastního řešení včetně implementace. Výsledná platforma bude primárně určena pro domácí kutily, kteří si vyrobý DIY zařízení. Uživatel bude s platformou interagovat pomocí Progresivní webové aplikace. Součástí práce je dokumentace pro připojení zařízení k platformě včetně několika celých řešení založených na chipu ESP8266.}
\abstractEN{Sem doplňte ekvivalent abstraktu Vaší práce v~angličtině.}
\placeForDeclarationOfAuthenticity{V~Praze}
\declarationOfAuthenticityOption{4} %volba Prohlášení (číslo 1-6)
\keywordsCS{ESP8266, PWA, MQTT}
\keywordsEN{Nahraďte seznamem klíčových slov v angličtině oddělených čárkou.}
% \website{http://site.example/thesis} %volitelná URL práce, objeví se v tiráži - úplně odstraňte, nemáte-li URL práce

\begin{document}

% \newacronym{CVUT}{{\v C}VUT}{{\v C}esk{\' e} vysok{\' e} u{\v c}en{\' i} technick{\' e} v Praze}
% \newacronym{FIT}{FIT}{Fakulta informa{\v c}n{\' i}ch technologi{\' i}}

\begin{introduction}
    %sdělit že téma je nové, aktuální, je ho potřeba řešit, komu to bude prospěšné, proč jsme si ho zvolili - vypadá dobře oborová motivace -> vyřešení daného problému někomu pomůže (ideálně nějaké komunitě), sdělit povrchově čím se zabývá práce ->rozvedení v kapitole Cíl, pak na konci úvodu stručné představení práce, aby si čtenář udělal představu jak budu postupovat

    Internet věcí je horkým tématem posledních několika let, ale jeho vývoji předcházela spousta trnitých cest a slepých uliček. Pod kouzelnou zkratkou IoT se pro mnohé skrývá příslib pokroku od chytré domácnosti až po revoluci v průmyslu. Internet věcí je označení pro síť fyzických zařízení, které dokáží spolu komunikovat ať už napřímo nebo pomocí prostředníka. Pro efektivní správu se zařízení připojují k centrální Platformě, která sbírá data z jednolivých zařízení a dle definovaných pravidel zařízením posílá příkazy. Například v domácnosti čidlo zaregistruje příchod majitele domů a Platforma v reakce zapne vytápějí. Současně poskytuje uživatelské rozhraní, pomocí kterého lze jednotlivá zařízení ovládat přímo a definovat scénáře, na základě kterých se vykonává automatizace jako uvedené automatické zapnutí vytápějí při příchodu. Platforma je tedy nedílnou součástí světa IoT a od jejích funkcí se odvíjí možnost využití plného potenciálu.

    Na trhu již dnes existují hotová řešení, ale je velmi problematické se mezi nimi zorientovat a často bývají velmi drahá. Důvod vzniku této práce pochází z osobní negativní zkušenosti s komerční platformou s cílem vytvoření dostupné otevřené Platformy pro technické entuziasty a bastlíře, kteří si chtějí jako já vytvářet levná zařízení a jednoduše je spravovat/ovládat.

    Teoretická část práce se věnuje analýze aktulních řešení na trhu se závěrečným porovnáním. Praktická část se zaměřuje na návrh, implementaci a nasazení vlastního řešení.

    %představit koncepci koncovích zařízení
    %využití - sledování vláhy, zalévání, udírna, ovládání hlasem

    % Takovýchto řešení již dnes existuje celá řada a podíváme se na výhody a nevýhody těch nejznámějších řešení jak z komerčního, tak i OpenSource světa.
    % Následně se budeme věnovat návrhu vlastního řešení s důrazem na bezpečnost, kterou stále spousta výrobců opomíjí. Moje řešení bude mít webové rozhraní, které umožní ovládání ze všech běžných zařízení bez nutnosti vytvářet nativní aplikace.


\end{introduction}

\chapter{Cíl práce}
% návrh vlastní platformy, cíleno pro kutily, bezpečnost pochybná u komerčních řešení -> já chci mít velkou, moje řešení bude více decentralizované
Cílem této práce je porovnat výhody a nevýhody těch nejznámějších řešení jak z komerčního, tak i OpenSource světa a následně navrhnout a realizovat vlastní řešení. Tato Platforma bude koncipována primárně pro cloudové nasazení. Bude umožňovat registraci uživatelům, kteří si potom budou moci spravovat svá vlastní zařízení včetně oprávnění. Na rozdíl od komerčních řešení, která jsou založena čistě na centrální správně, tak tady bude umožněn jak centralizovaný přístup (přes platformu), tak i možnost decentralizovaného - zařízení se budou moci přihlásit k odběru události z ostatních zařízení a na základně toho reagovat. Otevře se tak možnost využít pouze nejzákladnější funkce z platformy, ale veškerou automatizaci a následné reakce implementovat pouze ve firmwaru jednotlivých zařízení. Tento přístup je pro implementaci složitější, ale zvyšuje následnou odolnost v případě výpadku Platformy.

Řešení bude cíleno primárně pro domáci kutily, kteří si budou moci připojit libovolné DIY zařízení, nebo pro specializované nasazení, které vyžaduje vysokou spolehlivost.

Vzhledem k rozsahu Bakalářské práce není cílem nahradit existující řešení, která již obsahují velké množství funkcí, ale vytvořit možnou alternativu a demonstrovat složitost vytvoření celého řešení za použití moderních technologií.



\chapter{Analýza}
Tato kapitola se zabývá definici IOT Platformy a analýzy již existujících řešení. V závěru kapitoly je matice obsahující porovnání známých řešení vůči jejich funkcím.

\section{Definice IOT Platformy}
"IOT Platforma je více vrstvá technologie, která umožňuje přímočaré zajištění, ovládání a automatizaci připojených zařízení ve světě internetu věci. Zjednodušeně propojuje Váš hardware, jakkoli rozdílný, do cloudu s možností různorodé konektivity, obsahuje bezpečnostní mechanizmy a široké možnosti pro zpracování dat. Pro vývojáře, IOT Platforma nabízí soubor předpřipravených funkcí, které vysoce zvyšují rychlost vývoje aplikací pro připojená zařízení a řeší škálování a kompatibilitu napříč zařízeními" [*, překlad autora] %https://www.kaaproject.org/blog/what-is-iot-platform 

\subsection{Definice pojmů}
V této sekci jsou vysvětleny pojmy, které budou použity v následujících kapitolách.

\begin{itemize}
    \item \textbf{Platforma} -
    \item \textbf{Koncové zařízení} -
    \item \textbf{GateWay} -
    \item \textbf{DIY} -
\end{itemize}

\section{Vlastnosti??}

\subsection{Komunikace}    % rozdíly v komunikačním mediu wifi, bluethoot, mesh řešení Z-wave a zigbee, LoRa, spotřeba, bateriový provoz, složitost instalace
Pro komunikaci mezi zařízeními se nejčastěmi používá bezdrátová komunikace, kvůli jednoduchosti instalace bez nutnosti většího zásahu do stávající infrastruktury. Tento typ lze rozdělit do dvou kategorií:
\begin{itemize}
    \item \textbf{Centralizované} - Každé zařízení komunikuje pouze s jedním centrálním prvkem, přes který jde veškerá komunikace. Mezi nejznámější technologii tohoto typu patří Wifi.
    \item \textbf{Decentrlizovanou} - V této síti komunikují zařízení přímo s ostatními bez jakéholiv prostředníka. Pokud zařízení nemohou komunikovat napřímo, tak využívají ostatní pro předání zprávy. Síť je díky tomu mnohem odolnější vůči výpadků, protože zde není tzv. "Single point of failure". Zpravidla mívá nižší datovou propustnost a je složitější pro nasazení a následnou správu. Velkou výhodou je snadnější rozšiřitelnost pokrytí, protože každé přidané zařízení rozšiřuje signál a tímto způsobem lze zařízení řetězit.
\end{itemize}
Vzhledem k rozšířenosti Wifi, kterou dnes najdeme v každé domácnosti, se přirozeně nabízí využít tuto možnost i pro internet věcí. A k tomu v posledních letech opravdu došlo. Díky extrémně levnému chipu ESP8266, který se dnes v ČR dá koupit za 40 Kč, došlo k masivní penetraci trhu s chytrými zařízeními využívající právě Wifi. Bohužel tato technologie má i svá negativa a největší je spotřeba elektrické energie a limit maximálního početu připojených zařízení na jeden centrální prvek (řádově desítky). Vysoká spotřeba je dána nutností časté komunikace jen kvůli udržení aktivního spojení a proto je možné provozovat zařízení na baterie pouze v jednotkách dní až týdnů.

Pro bateriový provoz vznikly speciální sítě, které sice na rozdíl od Wifi umožní přenos v desítka kp za sekundu (tisícina rychlosti běžné Wifi), ale jsou energeticky mnohem úspornější (umožnují provoz až desítky let na malou baterii), mají mnohonásobně větší dosah a umožňují propojení mnohem většího počtu zařízení (stovky).

% TODO - odkud je toto tvrzení převzato
Nejvíce rozšířenými z centrálně orientovaných sítí jsou SigFox a LoRa. SigFox je komerční řešení, kde se platí za každé připojené zařízení. Oproti tomu síť LoRa používá otevřený standard pro komunikaci LoRaWAN. Protože se jedná o otevřený standard, tak kdokoliv může vytvořit a provozovat kompatibilní zařízení. Samozřejmně také lze využití komerční infrastrukturu, kam lze připojit svá zařízení za poplatek, ale díky otevřenosti má každý možnost si za pár tisíc postavit vlastní GateWay a provozovat libovolná zařízení bez jakýhkoliv poplatků a prostředníků.

%https://thesmartcave.com/z-wave-vs-zigbee-home-automation/
Z decentralizovaných sítí se nejvíce ujaly Zigbee a Z-Wave. Zigbee je otevřený standard, který dokáže pracovat, jak v pásmu 2.4GHz, tak i 900 MHz. Nemá omezení na maximální počet zařízení zřetězených za sebou a dokáže vytvořit síť skládající se až ze 65 tisíc zařízení. Z-Wawe je naopak uzavřený standard, který funguje pouze v pásmu 800-900 MHz. Limituje maximální počet přeposlání zprávy na 4 a podporuje síť o velikosti až 256 zařízení. Obě sítě jsou energeticky velmi úsporné a umožňují běh zařízení na obyčejnou knoflíkovou baterii po dobu až několika let.

\subsection{Automatizace}
%https://www.iot-now.com/2020/06/10/98753-iot-home-automation-future-holds/
Automatizace je ve světě IoT pravděpodobně nejdůležitějším tématem a každá IOT Platforma by ji měla umožňovat, protože dává možnost využít zařízení úplně novým způsobem. Principiálně se jedná o možnost definování reakcí na jednotlivé události. Událost můžemý být změna teploty, otevření okna nebo detekce pohybu a reakce změna stavu zařízení - zhasnutí světla nebo zapnutí televize. V podstatě jediným limitem je zde lidská představivost. Modelový scénář:

Představme si moderní dům, ve kterém jsou všechny věci, které nás napadnou, chytré, což s dnešními technologickými možnostmi není sci-fi, ale naopak možná realita. Majitel domu, říkejme mu Joe, přichází večer unavený domů a odemyká dveře. Vejde do vnitř a světlo na chodbě a v kuchyni již svítí. Jde přímo do kuchyně, protože po dlouhém dni v práci má hlad a usedá s jídlem ke stolu. Nemá rád ticho, tak řekne: ,,Alexo, zapni hudbu" a ze Sterea se spustí Beethoven, protože Alexa ví, že je to Joeův oblíbený skladatel klasické hudby. Joe cítí jak se po místnosti rozprostřívá příjemné teplo ze zapnuté klimatizace. Po příjemné večeři odchází do druhého patra do koupelny, samozřejmně se nemusí starat o zapnuté Stereo ani světla, protože se vše samo vypne, jakmile odejde. Ve sprše pustí vodu, která má automaticky teplotu nastavenou specificky dle Joeovi preference 36 °C i přes to, že 20 min předním se sprchovala jeho přítelkyně, která si libuje v teplejší vodě. Po sprše jde do ložnice a ulehá do postele zatímco se kontroluje, jestli jsou všechny dveře zamčené, okna zavřená a zapíná se alarm pro případný pohyb ve spodním patře. A jak mohlo být vše uzpůsobené Joeovím preferencím a vše zapnuté ještě před jeho vstupem do domu? Protože zvonek u dveří má kameru s rozpoznáváním obličeje - Joea tedy poznal a vše nastavil.

Takto tedy může vypadat automatizace v domácnosti, která zpříjemní život a odprostí Vás od spousty všedních věcí. Vše nastavené dle osobních preferencí a to nejen určité rodiny ale na úrovni jednotlivců v domácnosti.


\subsection{Bezpečnost a soukromí}
Při výběru Platformy by důležitým kritériem měla být bezpečnost. Na první pohled se to nemusí zdát být důležité, co se může stát když bude s platformou komunikovat čidlo pohybu a někdo se dokáže dostat k těmto údajům? Například pro zloděje mohou být taková data zlatý důl, protože bude přesně vědět kdy je dům prázdný.

Bezpečnost je potřeba zde sledovat hned na několika faktorech. Prvním je komunikační médium. Pokud zařízení komunikují bezdrátově, tak by komunikace měla být šifrovaná, aby se nedala jednoduše odposlechnout. Druhým faktorem je bezpečnost samotné platformy. Pokud bude platforma dostupná pouze na interní síťi v domácnosti, tak bezpečnost na první pohled ohrožená není, pokud se ale zamyslíme nad tím, kolik dnes doma máme chytrých zařízení, tedy takových, která dokáží komunikovat přes internet, tak zjistíme že jich je velké množství, protože dnes už takovou chytrou televizi má doma téměř každý a je otázka na kolik věříme výrobcům těchto zařízení, že kladou důraz na jejich bezpečnost. Stačí aby nějaký vir napadl naší televizi či jiné zařízení a případný útočník má plný přístup k platformě pouze získáním přístupu do interní sítě. Proto by platforma měla využívat alespoň systém pro identifikaci, ideálně i autentifikaci a to nejen v případě, že je přístupná z internetu ale i z vnitřní sítě.

\subsection{Cílová skupina}
Internet věcí lze využít napříč všemi sférami. Od jednoduché meteostanice, která bude měřit venku teplotu, přes tvz. chytrou domácnost, kde Vám lednička pošle nákupní seznam na email podle chybějících potravin, přes využití v průmyslu pro sběr různorodých dat a jejich následnou analýzu ať pro zvýšení kvality nebo detekci poruchy, ještě před tím než k ní dojde. Tato práce se zaměřuje na využití IoT v běžné domácnost a implementací Platformy určené pro kutily a technické entusiasty, kteří chtějí mít svá data pod kontrolou a vytvářet různorodá zařízení, která si k Platformě připojí.


% \section{Procesy} % základní features z pohledu uživatele
% Tato sekce obsahuje popis 3 základních procesů, se kterými uživatel přijde do styku.

% \subsection{Přidání zařízení}%jak probíhá přidání zařízení
% Proces přidání nového zařízení byl měl být pro uživatele co možná nejjednoduší, protože první interakce uživatele se zařízením je právě počáteční zprovoznění, při kterém si uživatele vytváří názor, který se zpětně velmi těžko mění. První dojem by měl tedy být co možná nejvíce přívětivý.

% \subsection{Interakce}% jak zobrazit data a měnit stav
% Následná interakce s jednotlivými zařízení přes Platformu musí být intuitivní a nabídnout uživateli prostředí, které nebude zbytečně komplikované a umožní mu snadné zobrazení dat formou vizualizací.

% \subsection{Automatizace}% jak definovat reakce/schémata
% Platforma by měla nabídnout uživateli možnost definování reakcí na různé akce. Od jednoduchý až po složitější skládání toků pro vizuální programování.

\section{Existující řešení}
\subsection{Komerční řešení} % hotové řešení, cloud, ale drahé
%easy to use, but paid
Na trhu dnes existuje velké množství komerčních řešení od známých výrobců. Někteří jsou známí spíše výrobou harwaru jako Philips a Xiaomi, jiní se zaměřují spíše na nabídku služeb a integraci zařízení ostatních výrobců pod svojí Platformu jako Amazon nebo Google. Pro koncového zákazníka mají Komerční řešení obrovskou výhodu v jednoduchosti nasazení a následné obsluhy. Stačí zakoupit centrální jednotku, libovolná zařízení od stejného výrobce a vše krásně funguje. Avšak problém nastává ve chvíli, kdy potřebují řešení škálovat či customizovat dle svých potřeb, protože si dodavatel za úpravy na \uv{míru} začne účtovat obrovské částky a zákazníkovi nezbývá nic jiného než platit. Sám si potřebné změny udělat nemůže, protože nemá zdrojové kódy a migrace k jinému produktu by znamenal obrovské náklady a problémy se stávájícími integracemi, protože různá řešení mívají různá rozhraní.

%security
Aspekt bezpečnosti u uzavřených řešení bývá diskutabilní. Pravidelné bezpečnostní audity kvůli vysokým nákladům provádí málo kdo. Výrobci samozřejmně vždy tvrdí, že bezpečnost je u nich na prvním místě, ale bohužel tento aspekt je v přímém kontrastu s jenododuchostí použití, což je pro výrobce mnohem důležitější protože pokud se řešení dobře a jednoduše ovládá, tak mnohem spíše si ho zákazníci oblíbí, než pokud bude maximálně zabezpečeno, ale uživatel bude muset provádět úkony návíc čistě kvůli bezpečnosti, která mu na první pohled nepřínáší přidanou hodnotu.

%Cloud dependent
Od Platformy očekáváme možnost vzdáleného ovládání, tedy přístup odkudkoli z internetu. Málokdo má však doma veřejnou IP adresu, aby si mohl celé řešení provozovat doma \uv{self-hosted}. V Praxi si tedy uživatel pořidí domů GateWay, která komunikoje s chytrými zařízeními v domácnosti a současně s cloudem výrobce, přes který lze přistupovat na Platformu a ovládat všechny zařízení. Takové řešení se velmi osvědčilo díky jednoduchosti, protože neklade žádné nároky na uživatele jako např. veřejnou IP adresu. Problém však může nastat ve chvíli, kdy výrobce daného řešení po několika letech ukončí činnost a s tím přestane provozovat svojí cloudovou infrastrukturu, na které je závislá GateWay a vzdálený přístup z internetu. V lepším případě bude zachována funkčnost v lokální sití, v horším přestane řešení fungovat úplně. Najednou uživatelovi zbyde doma spousta funkčního (po fyzické stránce) harwaru, který nemůže využívat.

Výše jsem nastínil nejhorší možný scénář, který naštěstí v poslední době již přestává platit, protože výrobci společně vytvářejí otevřené standardy pro komunikaci, které by měli zaručit kompatibilitu zařízení napříč jednotlivými výrobci. Bohužel standardů vzniká současně více a ne všichni je plně dodržují, takže nekompatibilita ještě bude delší dobu přetrvávat i když ne v takovém měřítku jako před pár lety. Kromně rozdílných protokolů je také nekompatibilita v různých technologiích přenosu mezi nejznámější patří WiFi, Bluethooth, LoRa, Zigbee a Sigfox.
%podpora jiných výrobců? integrace? -> závislé na tom co výrobce se rozhodne implementovat

\subsection{OpenSource řešení}
% nepopulární/špatná reputace mezi lidmy, často potřeba znalosti problematiky, Free, flexibilní, customizovatelné
OpenSource řešení mají mezi širší veřejností špatnou reputaci, protože na rozdíl od komerčních \uv{Plug\&Play} produktů většinou vyžadují určité povědomí o dané problematice. Je to způsobeno tím, že se snaží pokrýt celou doménu stejně jako komerční řešení, ale oproti nim se zlomkem vývojářů a financí. Následkem toho není prvotní nastavení pro laika zcela přímočaré a může se střetnout s problémy. Avšak překonání prvotních nesnázích přináší následně spoustu pozitiv.

Jedním z nejatraktivnějších lákadel je samozřejmně cena. OpenSource řešení jsou zpravidla zcela zdarma, případně nabízejí placenou podporu. Mě osobně na OpenSource nejvíce zaujala komunita. Pokud se projekt dostane do určité známosti, tak kolem něho začně vznikat komunita lidí, primárně technologických nadšenců ale i lidí z IT praxe, kteří mezi sebou komunikují a spolupracují na vylepšení daného řešení, ať už přímo (napsání části funkcionality) nebo nepřímo (komunikace s vývojáři). Potom i obyčejný uživatel, který chce řešení využít, tak při objevení potíží, může požádat komunitu o pomoc a protože to jsou nadšení lidé, jsou velmi ochotní.

Pokud máme dostatečné technické znalosti, tak si můžeme prohlédlou přímo zdrojové kódy a sami si zhodnotit kvalitu i bezpečnost. U větších projektů to však již není tak úplně možné při desítkách tisíc rádků kódu, ale existují lidé, kteří tomu opravdu věnují čas a mohou tak objevit zranitelnosti. Dále OpenSource projekty bývají mnohem více sdílné ohledně architektury kterou využívají a je možno se v dokumentaci dočíst, jak vlastně řešení funguje interně, na rozdíl od komerčních, kde je to tzv. \uv{BlackBox}.

OpenSource platformy bývají postavené na systému Pluginů, tedy obsahují určitou základní sadu funkcí a dále lze funkčnost rozšiřovat pomocí instalace Pluginů. Ty mohou vytvářet přímo autoři nebo kdokoli jiný dle potřeb. Díky tomu jsou velmi robustní a podporují širokou škálu zařízení od různých výrobců napříč technologiemi a pokud ne, tak s trochou znalostí v programování si může každý dopsat plugin dle potřeb pro podporu daného zařízení.


\subsection{Známé Platformy}  % TODO uvést zdroj, ze kterého plyne že jsou nejoblínější nebo přeformulovat
Tato sekce s zabývá analýzou 4 nejznámějších Platforem a v závěru jejich vzájemným porovnáním.

\paragraph{Blynk}
Blynk se označuje jako harware-agnostic IOT Platforma s white-label mobilními aplikacemi. Umožnuje navrhnovat vlastní aplikace formou DragAndDrop pro ovládání zařízení, analýzu telemetrických dat a správu nasazených produktů ve velkém měřítku. Své řešení nabízejí jak pro domácí nasazení, tak i jako enterprise řešení pro větší firmy. Mají \textbf{3 cenové tarify}:
\begin{itemize}
    \item \textbf{Free} je omezené pouze pro osobní užití, obsahuje cloudový hosting, umožňuje připojit maximálně 5 zařízení zdarma a součástí je mobilní aplikace pro Android a iOS.
    \item \textbf{StartUp} je určeny pro komerční využití a cenou začínají na \$415/měsíc. Součástí je deployment vlastních aplikací na AppStore/Google Play, neomezený počet zařízení a uživatelů, garantované podpora
    \item \textbf{Business} začíná na \$1000/měsíc a nabízí navíc OTA aktualizace koncových zařízení (vzdáleně), webové rozhraní, datovou analýzou a dalších funkce.
\end{itemize}

Hardware-agnostic znamená, že nejsou omezeni pouze na určitý hardware a umožňují připojit v podstatě libovolné zařízení. Pro připojení maji definované rozhraní nad jednotlivými protokoly. Podporují custom TCP/IP, WebSocket, HTTP a nově i MQTT (zatím k němu nemají ale dokumentaci). Dávají k dispozici knihovný pro různé harwarové platformy, takže připojení k platformě je potom otázka dvou řádků kódu. K dispozici je velmi přehledná a detailní dokumentace.

Nativní aplikace pro iOS a Android umožnuje vytvářet vlastní dashboardy pomocí již předpřipravených Widgetů, kterých je opravdu velké množství, ale jsou placené za tzv. Energii, což je měna která lze dobíjet za peníze, dále definovat vlastní widgety a upravit chování celé aplikace. Následně lze takto upravenou aplikaci vyexportovat a přímo nahrát na Google Play a AppStore pod vlastním názvem. Tento přístup nabízí elegantní možnost pro tvorbu vlastního řešení, které následně je možné nabízet jako vlastní produkt.

Výhodou cloudové řešení je přístup k platformě odkudkoliv z internetu. Zároveň je potom ale funčknost odkázána na dostupnost internetového připojení a představa dat někde v cloudu se nemusí líbit. Pro tento připad je možnost hostovat si vlastní Blynk server, který je dostupný jako OpenSource server napsaný v Javě. %https://github.com/blynkkk/blynk-server

\paragraph{Thingspeaks}
ThingSpeak™ je služba analytické IoT platformy od MathWorks®, tvůrců MATLAB® a Simulink®. Jedná se o hardware-agnostic platformu s webovým rozhraním, která se plně zaměřuje na analýzu dat. Je ideální pro lidi se zkušeností s Matlab, protože je postavena právě na této platformě. Umožňuje v cloudu sběr dat, jejich analýzu přímo pomocí Matlab kódu, vizualizaci dat a definování reakcí. Pro různé harwarové platformy mají připravené knihovny a nativě podporují komunikace pomocí protokolů HTTP a MQTT. Řešení nabízejí podle následujících tarifů, kde omezení jsou primárně ohledně maximálního počtu zpráv, počtu kanálů do kterého posílají zařízení zprávy a minimálního časového odstupu mezi zprávami v rámci jednoho kanálu.
\begin{itemize}
    \item \textbf{Free} ~8 200 messages/day, počet kanálů 4, interval mezi zprávami 15s, Matlab maximální doba běhu 20s
    \item \textbf{STANDARD} ~90 000 messages/day, počet kanálů 250, interval mezi zprávami 60s, Matlab maximální doba běhu 20s
    \item TODO matice pro tarify jednotlivé?
\end{itemize}

\paragraph{Home Assistant}
OpenSource domací automatizace, která dává lokální kontrolu a soukromí na první místo - takto se prezentuje Home Assistant. Tato platforma není tolik zaměřena na koncová zařízení jako předchozí, ale funguje jako integrátor komerčních/OpenSource řešení pod jednotné rozhraní. Obsahuje systém pro tvorbu automatizace, tedy vytváření reakcí na jednotlivé akce. Dokáže se napojit buď přímo na jednotlivá zařízení nebo na jejich GateWay a umožnit ovládání všech zařízení od různorodých výrobců, kteří často vynucují použití vlastní aplikace, pod jednotné rozhraní jak webové, tak ve formě nativní aplikace. Integrace je řešena pomocí pluginární systému, kde zpravidla jeden plugin obsahuje integraci pro jednoho výrobce/jednu GateWay. Většina pluginů vzniká přímo od komunity této platformy. V době psání této práce obsahuje 1743 pluginů.

Celá platforma je zdarma a pro její zprovoznění stačí Raspberry Pi, na SD kartu nahrát předpřipravený image a zapnout. Prvnotním nastavením Vás následně provede webové rozhraní nebo nativní aplikace, záleží na Vaší volbě.

Platforma podporuje velké množství komerčních produkté mezi nejznámější patří Ikea TRÅDFRI, Philips Hue či Google Assistant. Samozřejmně podporují i OpenSource řešení mezi nejznámější patří ESPHome, což je framework pro konfiguraci ESP chipů (ESP8266/ESP32), který řeší vrstvu komunikace a zapojení do platformy - stačí pouze dodefinovat chování na určité události a chytré zařízení je připravené.

\paragraph{OpenHAb}
%projekt s dlouho historií (v1 in 2010), lepší dokumentace, trochu složitější nastavení, mohutnější ale více možnost oproti HA, systém Addonů 324 aktuálně
OpenHab je OpenSource projekt s dlouhou historií, který vznikl již v roce 2010. Cílí na stejný segment jako Home Assistant, tedy  propojení existujících řešení pod jednotné rozhraní a jejich automatizaci. Jedná se o hardware agnostic platformu, která komunikuje přímo s koncovými zařízeními nebo příslušnou GateWay. V základu obsahuje více funkcionalit, zatímco Homa Assistant je více minimalistický. Pro rozšířování funkcnionality používá systém doplňků (aktuálně 324), které vyvájejí autoři a především komunita. Od prvopočátku projektu je zde kladen velký důraz na nativní aplikace na rozdíl od Home assistantu, který dlouhou dobu neměl oficiální aplikaci pro Android. Webové rozhraní je samozřejmostí. Velkou výhodou je možnost využití cloud instance zcela zdarma, buď jako plnohodnotnou platformu nebo pouze pro přístup z internetu k vlastní instanci.

Prvotní instalace je stejně jednoduchá jako u Home Assistentu. Rozdíl přichází při přidávání jednotlivých zařízení, kde je proces trochu komplikovanější. OpenHab se snaží nabídnout pokročilejší funkcionalitu, která bohužel částečně zesložiťuje jednotlivé procesy. Na druhou stranu umožňuje větší flexibilitu.

Dokumentace projektu je na velmi vysoké úrovni s velmi detailním popisem. Pravděpodobně díky tomu, že projekt existuje již 10 let a má silnou základnu v komunitě i přes to, že je přibližně poloviční oproti té, kterou má Home Assistant.

\subsection{Porovnání}
% TODO závěrečné porovnání - Blynk více orientované na koncová zařízení, ThingSpeaks primárně pro analýzu dat s Matlabem, HA a openHab jsou HUBy pro domácí automatizaci 
Jednotlivé Platforma se některými funkcemi překrývají a v jiných jsou zase jedinečné. Při výběru je důležité si stanovit na co Platformu chceme využívat a jaké funkce vyžadujeme.

Blynk primárně cílí na podnikatelský segment, a nejvíce se hodí firmám, kteří chtějí na této Platforma vysvtavět své řešení, které následně budou přeprodávat pod svojí vlastní značkou. To díky příme možnosti exportu aplikace na AppStore a Google Play, hromadné zprávě zařízení a ACL (seznam oprávnění vázaný k zařízení, který specifikuje kdo k němu může přistupovat a jaké operace provádět). Kvůli chybějící podpoře komerčních zařízení lze využít pro domácnost pouze s DYI zařízeními.

ThingSpeaks míří primárně na zpracování dat díky svému ekosystému postaveném kolem MATLAB®. Pro veškeré zpracování, analýzi a vizualizace stačí znalost prostředí MATLAB®, který je světově známý a velmi oblíbený mezi akademiky.

Home Assistant je progresivní OpenSource Platforma, která umožní integraci komerčních řešení pod jednotné rozhraní a domácí automatizaci s příjemným uživatelským rozhraním.

OpenHab projekt s dlouho historií. Funkčně se velmi podobá Home Assistantu, ale snaží se uživatelům nabídnout více funkčnosti. Uživatelské rozhraní je občas trochu složitější.


\begin{center} % pro addony přidat poznámku 324 obsahuje 2585 věcí
    \begin{tabular}{ |c| m{5em}| m{5em}|m{5em}|m{4em}| m{5em}| m{4em}| m{4em}| }
        \hline
        Platforma      & Podpora komerčních produktů & Vlastní zařízení   & Hosting            & ACL              & Nativní aplikace         & Správa zařízení & Cena              \\
        \hline
        Blynk          & Ne                          & Ano                & self-hosted, cloud & Pouze Enterprise & Ano                      & Ano             & Omezený Free plan \\
        \hline
        ThingSpeaks    & 6 vendorů (primárně LoRa)   & Ano                & cloud              & Ano              & Pouze pro náhled na data & Ne              & Omezený Free plan \\
        \hline
        Home Assistant & pomocí Pluginů (1743)       & 3rd party knihovny & self-hosted        & Ano              & Ano                      & Ne              & Ano               \\
        \hline
        openHab        & pomocí Doplňků (324)        & 3rd party knihovny & self-hosted, cloud & Ne               & Ano                      & Ne              & Ano               \\
        \hline
    \end{tabular}
\end{center}

\subsection{Závěrečný verdikt??}
Blynk je první platforma, se kterou jsem se střetl ve světě IoT asi před dvěma roky a bohužel prvním dojem pro mě byl poměrně negativní. Mnohé se od té doby změnilo, ale nepřímá podpora MQTT protokolu a především nutnost platit řešení mě od této Platformi odrazuje. Thingspeaks je hezké řešení, které splňuje většinu mých představ, ale úzká integrace s MatLab a nutnost jeho znalosti pro zpracování dat, je pro mne překážkou ať už z hlediska, že MatLab nepoužívám, tak více z pohledu ceny MatLab prostředí a celého ekosystému. Sám se považuji za OpenSource zastánce a proto mě to táhne k těmto řešením. HomeAssistant je velmi progresivní a zajímáva platforma, která je ale primárně určena pro nasazení v lokální síti (nepočítá s nutností autentizace zařízení), zatímco já bych chtěl primárně Platformu provozovat jako řešení, kde se stačí zaregistrovat a každý kutil může přidávat vlastní zařízení a veškerý tok dat bude oddělen mezi uživateli. OpenHAB řešení mě velmi zaujalo, především možnost hostingu cloudového řešení, zcela zdarma. Bohužel chybějící ACL je pro mne nepřekonatelnou překážkou, protože chci platformu využívat pro více uživatelů a tedy definovat jednotlivá oprávnění. Proto jsem se rozhodl vytvořit si vlastní řešení, které mi dá prostor realizovat vše dle svých představ s důrazem na bezpečnost.

\chapter{Analýza a návrh}

\section{Analýza procesů}

\subsection{Přidání zařízení}

\subsection{Sběr dat}

\subsection{Ovládání}

\subsection{Automatizace}



\section{Analýza požadavků}

\subsection{Funkční požadavky}

\paragraph{F1 Registrace uživatelů}
- systém bude umožnovat libovolnému uživateli registraci. Každému registrovanému uživateli bude zřízen na server účet, který bude uchovávat jeho registrační údaje - jméno, přijmení, uživatelské jméno, heslo, email. Součástí registračního formuláře bude zaškrtávátko \uv{Automaticky přihlásit}, které bude ve výchozím stavu aktivní. Pokud ho uživatel nezruší, tak po úspěšné registraci bude uživatel automaticky přihlášen do systému.

\paragraph{F2 Obnovení zapomenutého hesla}
- součástí přihlašovacího formuláře bude odkaz na stránku pro obnovení zapomenutého hesla, kde bude uživatel dotázán na emailovou adresu, kterou použil při registraci. Po zadání, pokud daná emailová adresa je součástí některého uživatelské účtu, bude na ni odeslán email s odkazem pro obnovu hesla. Na tomto odkazu bude uživatel vyzván k zadání nového hesla.

\paragraph{F3 Přidání zařízení}
- přihlášenému uživateli se ve správě zařízení objeví možnost přidat ke svému účtu nové zařízení, pokud bude systémem detekované. Pro přidání bude uživatel vyzván k zadání umístění daného zařízení.

\paragraph{F4 Oprávnění na úrovni zařízení}
- na úrovni zařízení půjde nastavovat oprávnění pro jednotlivé uživatele. Toto oprávnění se bude skládat z těchto 3 úrovní:
\begin{itemize}
    \item \textbf{Read} - uživatel bude moci si pouze zobrazit veškeré údaje o zařízení
    \item \textbf{Control} - stejné oprávnění jako \uv{read} + navíc zařízení může ovládat
    \item \textbf{Write} - uživatel může editovat veškeré informace o zařízení (včetně oprávnění)
\end{itemize}

\paragraph{F6 Typy prvků}
- systém bude podporovat následující typy prvků a umožňí uživateli v rozhraní příslušné interakce.
\begin{itemize}
    \item \textbf{Senzor} - základní prvek, odesílá hodnoty do systému
    \item \textbf{Přepínač} - může se nacházet ve stavu zapnut/vypnut. Uživatel bude moci poslat příkaz ke změně stavu.
    \item \textbf{RGB světlo} - tento prvek má 3 vlastnosti: zapnuto/vypnuto, jas, barva. Uživatel může měnit všechny.
\end{itemize}

\paragraph{F5 Vizualizace dat}
- rozhraní umožní sledování dat ze všech typů prvků v reálném čase. Dále umožní vizualizaci historických naměřených údajů ze senzory, jejíchž výstup je číselná hodnota. Vizualizace bude realizována v podobě grafu vývoje hodnoty za určitý čas.

\paragraph{F7 Editace zařízení}
- jednotlivá zařízení bude možno přejmenovat a změnit jejich umístění.

\paragraph{F8 Notifikace}
- uživatel si bude moci nastavit Push notifikace pro jednotlivé prvky, které ho upozorní na změnu stavu. U senzorů si bude moci nastavit hranici, notifikace se pak odešle pokud ji hodnota překročí resp. sníží pod ni.

\paragraph{F8 Validace}
- veškerá formulářová pole budou interaktivně validována na straně uživatele v rozhraní a následně znovu na straně serveru. Interaktivní validací jsou myšleny tyto scénáře při průchodu formuláře:
\begin{itemize}
    \item Zadání hodnoty a vykliknutí z pole - bude provedena validace a v případě nevalidního vstupu, bude uživatel upozorněn.
    \item Editace hodnoty v poli - validace bude provedena po každé změně (stisknutí klávesy), uživatel bude opět upozorněn v případě nevalidní hodnoty.
\end{itemize}


\subsection{Nefunkční požadavky}

\paragraph{N1 Systém spustinelný na Linux systému}
- systém bude možno provozovat na Linuxovém serveru (Debian) a také na platformě Raspberry Pi (Raspbian).

\paragraph{N2 Responzivní webové rozhraní}
- aplikace bude nabízet responzivní uživatelské rozhraní přizpůsobené pro zobrazení na mobilních zařízeních i stolních počítačích. Uživatelské rozhraní bude kompatibilní s prohlížeči Mozilla Firefox verze 80, Chrome verze 80 a Safari na iOS. Dále bude implementovat tzv. PWA (Progresivní webová aplikace) - bude využívat cache pro statické soubory, pro rychlé načítání a na zařízení Android půjde v aplikaci chrome přidat na plochu a následně vypadat jako nativní aplikace.

\paragraph{N3 Rozhraní realizováno jako SPA}
- SPA (Single page application) je webová aplikace, která utilizuje JavaScript, aby při interakci v rámci aplikace se nemusela celá stránka načítat, ale pouze chytře překresluje potřebné části. Výsledkem je mnohem přijemnější uživatelský zážitek, než při nutnosti překleslovat celou stránku po kliknutí na odkaz.

\paragraph{N4 Koncová zařízení na platformě ESP8266}
- pro jednotlivá zařízení bude použit čip ESP8266 od firmy Espressif jako mikrokontroler, který bude komunikovat po sítí prostřednictvím Wifi sítě.

\paragraph{N5 Výkonnostní požadavky}
- systém bude stabilní a zvládne obsluhovat stovku zařízení, kde každé bude odesílat změnu stavu s periodicitou 30 vteřin. Při tomto dlouhodobém zatížení nebude docházet k pádům systému ani k výraznému zpoždění komunikace (RESTful požadavky pod 500 ms).

\paragraph{N6 Konfigurace systému} %env promněné
- veškerá konfigurace (týkající se externích služeb/komunikace) jako jméno a heslo do databáze, číslo portu pro komunikaci atd. bude konfigurovatelné pomocí promněných prostředí (env variables). Detailní popis promněných bude obsažen v instalační příručce.



\section{Popis domény}
TODO text
Popis jednotlivých enntit:
\begin{itemize}
    \item \textbf{User (Uživatel)} - osoba interagující se systémem
    \item \textbf{Device (Zařízení)} - fyzické zařízení komunikující se systémem, umožňující ovládání Věcí a odesílání dat
    \item \textbf{Thing (Věc)} - např. světlo, meteostanice nebo zásuvka
    \item \textbf{Property (Vlastnost)} - určitá vlastnost Věci (např. pro zásuvku je \uv{vypínač} - vypnuto/zapnuto, pro meteostanici \uv{teplota} - naměřená hodnota)
    \item \textbf{Location (Umístění)} - umístění dané zařízení, specifikována budova a pokoj
    \item \textbf{State (Stav)} - v jakém aktuálním stavu se určitá Věc nachází
\end{itemize}


\section{Analýza případů užití}
Tato kapitola popisuje identifikované případy užití, ve kterých figurují následující aktéři:
\begin{itemize}
    \item User - aktér představující autentizovaného uživatele webového rozhraní.
    \item User ROLE\_ADMIN - aktér představující autentizovaného uživatele s právý kompletní správy systému.
\end{itemize}

\subsection{UC1 - Registrace uživatele}
Neautentizovaný uživatel vyplní registrační formulář obsahující jméno, přijmení, uživatelské jméno, heslo a email. Validace dat bude prováděna dle [F8]. Při zpracování požadavku na serveru bude zajištěna unikátnost uživatelského jména a emailu napříč databází. Uživatel bude informován o úspěchu/neúspěchu akce. Po úspěšném vytvoření bude automaticky přihlášen, pokud nezrušil ve formuláři zaškrtávátku \uv{Automaticky přihlásit}, a bude mu odeslán uvítací email na zadanoou emailovou adresu.

\subsection{UC2 - Přidání nového zařízení}
\label{UC:UC2}
Uživatel ve správě zařízení bude mít sekci \uv{Přidat zařízení}, kde se zobrazí všechna nově detekovaná zařízení, která ještě nemá přidaná. Následně při kliknutí na tlačítko přidat se zobrazí jednoduchý formulář pro zadání umístění a názvu zařízení - bude předvyplněn název, který ohlásilo zařízení. Uživatel formulář potvrdí, systém následně vytvoří dané zařízení, přidá uživateli k němu oprávnění a ná stránce \uv{Ovládání} už bude uživatel moci sledovat aktuální stav Věcí a případně je i ovládat (pokud to umožňují).

\subsection{UC3 - Zobrazení teploty}
Na stránce \uv{Ovládání} bude zobrazeny jednolivé místnosti ve formě Widgetu, ve kterém budou aktuální údaje ze senzorů v dané místnosti. Při rozkliknutí místnosti budou zobrazeny všechny senzory a Věci. Po kliknutí na senzor bude uživateli zobrazena informace o stáří aktuální hodnoty a zobrazen graf vizualizující průběh hodnoty v čase za posledních 24 h.

\subsection{UC4 - Rosvícení světla}
Na stránce \uv{Ovládání} po rozkliknutí místnosti obsahující Věc pro ovládání světla \uv{přepínač} bude zobrazen prvek umožnující uživateli jeho změnit stav. Po kliknutí bude odeslán požadavek pro změnu stavu na server, který ho odešle danému zařízení. Po potvrzení od zařízení, že ke změně opravdu došlo bude ze serveru odeslána tato informace webové aplikaci.

\subsection{UC5 - Editace zařízení}
Uživatel na stránce \uv{Správa zařízení} bude mít zobrazena všechna zařízení, ke kterým má oprávnění pro editaci. Rozhraní umožní pomocí formuláře editaci názvu zařízení, umístění a změnu oprávnění pro jednotlivé uživatele. Dále může zařízení smazat.

\subsection{UC6 - Správa uživatelů}
Uživatel s rolí \uv{admin} může přistoupit na stránku \uv{Správa uživatelů}, kde se mu zobrazí seznam všech registrovaných uživatelů. Jednotlivé uživatele může smazat a pomocí formuláře editovat jejich osobní údaje včetně hesla.

\subsection{UC7 - Device discovery}
Zařízení pokud není ještě není spárované s Platformou, tak při zapnutí odešle status=\uv{init}, následně začne ohlašovat, které Věci má a jejich Vlastnosti. Po jejich odeslání pošle status=\uv{ready} a příslušnému uživateli, ke kterému se zařízení hlásí se zobrazí v rozhraní viz. \hyperref[UC:UC2]{UC2}. Po jeho přidání uživatelem Platforma odešle zařízení API klíč, které si ho uloží a přihlásí se pomocí toho klíče.

\subsection{UC8 - Změna stavu Vlastnosti}
Při požadavku na změnu stavu \uv{Vlastnosti} odešle Platforma požadavek na dané zařízení, které provede změnu a odešlě potvrzovací zprávu, že ke změně úspěšně došlo. V případě nevalidního požadavku, zařízení žádné potvrzení neposílá.


\section{Vybrané technologie}

\subsection{Komunikační protokol}   %https://ieeexplore.ieee.org/abstract/document/8079928
Komunikační protokolů je velké množství a přímo závisí na výběru přenosného média. Při použití specializovaných sítí jako LoRa nebo Zigbee, nemáme moc velkou flexibilitu ve výběru. Zařízení podporující tyto specializované sítě jsou také poměrně drahá, ale umožňují běh na baterii. Zatímco využití WiFi sítě nám dává obrovskou flexibilitu ve výběru protokolů a zařízení podporující připojení k Wifi nebyli nikdo více cenově dostupné jako dnes. Primárně z finančních nároků a možnosti využití stávající WiFi infrastruktury v domácnosti jsem se rozhodl pro WiFi jako bezdrátové médium. Současně díky přímé podpoře IP protokolu nebudu muset vytvářet bridge mezi serverem a sítí se zařízenímí jako např. při použití Bluethooth či LoRa.

%https://www.educba.com/mqtt-vs-websocket/
Z protokolů pro komunikaci je dnes neojoblíběnější HTTP, který ale nativně nepodporuje obousměrnou komunikaci, kdy zařízení může poslat zprávu serveru a stejně tak server zprávu zařízení, kterou pro ovládání zařízení budu potřebovat. Z obousměrných protokolů se velmi osvědčil WebSocket, který lze velmi snadno kombinovat s HTTP. Jedná se o protokol postavený nad TCP, který naváže spojení a obě strany mohou posílat zprávy. Je to velice hezké řešení pro posílání zpráv, ale neumožňuje nativně systematickou filtraci nebo odběr pouze určitých zpráv a nepočítá s během na nespolhlivých zařízeních. Proto přímo pro IOT vznikl otevřený síťový protokol MQTT jehož specifikaci vydává OASIS[??]. Využívá asynchroní pattern publish-subsribe a byl speciálně navržen pro potřeby běhu na jednoduchých embeded zařízení s minimálním datovým tokem.

MQTT se vyvýjí již od roku 1999 a momentálně nejpoužívanější verzí je 3.1.1 pro kterou vznikla specifikace v roce 2014. Protokol primárně běží nad TCP/IP, ale lze využít v jakékoliv síti kde je zaručeno správné pořadí dat, beztrátovost a obousměrnost komunikace. Protokol definuje 2 typy entit: \uv{message broker} a klient. MQTT broker je server, který přijímá všechny zprávy od připojených klientů a přeposílá je správným příjemncům (klientům). MQTT klient je jakékoliv zařízení (od embeded až po server), které komunikuje s brokerem přes síť.

Posílaná data jsou zde hierarchicky rozdělena do tzv. topiců (česky témat). Topic je textový řetězec o maximální délce 65536 Bytů s oddělovačem "/" - ukázka "/house/bedroom/light". Pokud klient chce odeslat (publish) data, tak pošle zprávu brokeru s daty a topicem do kterého zpráva patří. Broker potom zprávu odešle všem klientům, kteří jsou přihlášení k odběru (subscrib) z daného topicu. O odběr se klient musí přihlásit a buď přímo specifikuje plný název topicu nebo částečný s použitím wildcard. MQTT počítá s případnou nespolehlivostí ať koncových zařízení nebo sítě a proto umožňuje klientovy při přihlášení definovat \uv{Last Will and Testament} (LWT), jedná se udání tématu a zprávy, do kterého se odešla v případě nesprávně odpojeného klienta. Takto lze notifikovat ostatní, že došlo je ztráně spojení s daným klientem.

Broker podporuje 3 třídy QoS (Quality of service), kterou lze specifikovat pro každou zprávu jednotlivě v závisloti na její důležitosti, jsou seřazeny vzestupně dle náročnosti na systém (overhead):
\begin{itemize}
    \item \textbf{0 - Maximálně jednou} - zpráva je odeslána pouze jednou a klient ani broker nijak napotvrzují její přjetí
    \item \textbf{1 - Alespoň jednou} - zpráva je odeslaná několikanásobně, dokud není potvrzené její přijetí
    \item \textbf{2 - Právě jednou} - odesílatel a příjemnce navazují dvoucestný hand-shake, aby bylo zaručeno přijmutí zprávy právě jednou
\end{itemize}
% TODO přeložit wildcard, overhead + pěkný obrázek client - broker - client pub/sub s komunikací


\subsection{Backend}    %https://nodejs.org/en/docs/
%spousta jazyků, pro koncepsi asynchroních messages MQTT a Websocket se hodí NodeJS
Vzhledem k povaze MQTT, který je koncepčně založený na asynchroních zprávách jsem si zvolil technologii NodeJS, která je postavené na asynchroní event-driven architektuře. NodeJS je prostředí pro běh JavaScriptu na straně serveru, kterému se v posledních letech dostává velké pozornosti kvůli jeho oblíbenosti mezi vývojáři, je extrémně přívětivý k začínajícím programátorům a má pozvolnou křivkou učení. Díky své architektuře nabízí velice elegantní přístup pro zpracování akcí, kde se musí dlouho čekat na výsledek jako primárně u síťové komunikace. V tradičním jazyce jako Java se taková akce musí řešit pracným vyvtvořením nového vlákna, které čeká na výsledek a následným zpracováním. V NodeJS je programátor od této problematiky odstíněn a může se tak plně věnovat tvorbě aplikační logiky aniž by měl znalosti a zkušenosti s více vláknovým programováním.

%https://vegibit.com/what-is-nodejs/ 
%modeulscount.com - down currently 
\paragraph{NodeJS} má pravděpodobně aktuálně největší a nejaktivější komunitu ze všech jazyků. Pro správu knihoven používá balíčkovací systém npm (jsou i jiné alternativy) ze kterého se stal největší ekosystém na světě, který je zastřešený neziskovou společností "npm, Inc." provozující centrální repozitář se všemi dostupnými moduly pro NodeJS. Díky sve centralizaci je velmi jednoduchý na používání, ale v posledních letech kdy se NodeJS zpopularizoval a nyní se obecně JavaScript řadí mezí nejoblíbenější jazyky, se ukázala centralizace jako poměrně nešťastné řešení, kvůli vysokým nákladům na provoz infrastruktury. Pro představu velikosti ekosystému: npm v roce 2020 obsahoval 1 200 000 modulů a druhý největší systém RubyGems "pouhých" 350 000. Všechny moduly jsou k dispozici zcela zdarma a díky takto aktivní komunitě lidí, kteří dávají k dispozici své knihovny ostatním, je vysoce pravděpodobné, že pokud chceme řešit nějaký problém, tak na něj již existuje knihovna.

%https://expressjs.com/
\paragraph{ExpressJS} je velmi minimalistický webový framework (první verze v roce 2010), který je do dnes velmi oblíbený a v mnohém ovlivnil vývoj většiny frameworků. Jeho největší výhoda je vysoká flexibilita. Nabízí pouze základní definici způsobu pracování s HTTP požadavky a možnost registrovat tzv. middleware - software, který rozšiřuje funkcionalitu. Veškerá funkcionalita je dodávána pomocí middlewarů, které jsou k dispozici jako moduly. Vývojář si tedy může výsledný server poskládat přesně dle svých představ, kterých existují desítky vytvořených přímo od autorů a další stovky od komunity.

\paragraph{MongoDB} je OpenSource dokumentová NoSQL databáze umožňující horizontální škálování. Oproti klasické SQL databázi používá dynamické schéma, díky kterému lze aplikaci mnohem rychleji a jednodušeji vyvíjet bez nutnosti striktního dodržování tabulkového schématu jako v případě SQL databáze. MongoDB používá pro uchovávání dokumentů podobný JSONu (MongoDB nazývá formát BSON), což se velmi snadno kombinuje s jazykem JavaScript, který JSON používá pro nativní objekty.

\paragraph{Mongoose} je knihovna pro NodeJS, která vytváří nad MongoDB objektovou abstrakci a spoustu dalších užitečných funkcí jako validace a type cast. Základním prvkem je definování schéma pro jednotlivé dokumenty, což může vypadat jako návrat do striktního schématu u SQL databází, ale zde je schéma definované pouze na úrovni Mongoose, tedy mnohem flexibilnější a méně striktní.

\subsection{Frontend}
% React + Redux
Prvním světoznámím průkopníkem ve světě JavaScriptu pro tvorbu uživatelského rozhraní byla jQuery, která existuje do dnes, ale spíše se již považuje za přežitek doby. Dnes máme velké množství Frameworků a knihoven pro tvorbu frontendu, ať pro tvorbu na straně serveru nebo přímo na straně uživatele v JavaScriptu. Trend dnešní doby je přesouvat generování rozhraní na stranu uživatele, jak kvůli snížení výkonostních nároků na server, tak spíše kvůli lepší odezvě a uživatelskému zážitků. Mezi nejznámější JavaScriptové frameworky patří bezpochyby Angular, Vue.js, Svelte a nesmíme zapomenout na React, který je sice knihovna, ale řadí se na stejnou úroveň. Já jsem si zvolil jako hlavní prostředek React, právě proto že se jedná o knihovnu. Framework se vyznačuje tím, že vynucuje určité problémy řešit jistým způsobem bez možnosti volby. Má to své výhody a nevýhody a do větších týmu bych rozhodně volil raději Framework. Tento projekt ale budu vytvářet primárně sám a mám velice rád flexibilitu a možnost volby. V začátcích to bývá časově náročnější, ale vidím v tom obrovskou možnost osobního růstu, protože při každé volbě musím hodnoti výhody/nevýhody a nakonec retrospektivně vidím následky svých rozhodnutí. Mimo to je React vytvářet společností Facebook a používán největšími technologickými světa, takže je jistá jeho dlouhodobá podpora a od roku 2013, kdy byla vydána první verze je hojně odzkoušený a ověřený.

%https://reactjs.org/tutorial/tutorial.html#what-is-react
\paragraph{React} je deklarativní, efektivní a flexibilní JavaScriptová knihovna pro tvorbu uživatelského rozhraní. Kód dělí do malých ižolovaných částí nazvaných "komponenta", které se skládání do sebe a mohou tvořit komplexní uživatelská rozhraní. Pro vysoký výkon využívá techniku virtuálního DOM - nejprve si vytvoří virtuální strom podoby rozhraní v paměti, který následně porovná s aktuální podobou vykreslenou v prohlížeči a zmanipuluje pouze ty části které se od posledního vykreslení změnily. Díky tomu je velice efektivní a dokáže vykreslovat komplexní stránky s obrovským množstvím dat.

%https://redux.js.org/introduction/getting-started
\paragraph{Redux} je knihovna pro state management. Ve větších aplikací je potřeba mít úložiště pro tzv. state (reprezentace momentálního stavu aplikace). Čím větší aplikace je, tím více dat je potřeba uchovávat a proto je nutno využít nějaké systematické řešení. React dlouho žádné takové nenabízel a bylo nutno využívat jiné knihovny jako např. Flux nebo Redux. V nejnovějších verzí již takové řešení nabízí, ale já přeferuji mít větší izolaci mezi vizuálnímí komponenty a statem, navíc Redux je odzkoušený časem a nabízí mnohem komplexnější řešení.


\section{Návrh uživatelského rozhraní}




\chapter{Realizace}

\section{Backend}

\subsection{Validace}

\subsection{MQTT schéma}

\section{Frontend}

\section{Knihovna pro ESP8266}

\section{chytrá udírna}

\subsection{Návrh zapojení}

\subsection{Výroba}



\chapter{Testování}

\section{Zátěžové testování}

\section{Zotavení po nenadálé události}

\section{Uživatelské testování}



\begin{conclusion}
    %sem napište závěr Vaší práce
\end{conclusion}

\bibliographystyle{csn690}
\bibliography{mybibliographyfile}

\appendix

\chapter{Seznam použitých zkratek}
% \printglossaries
\begin{description}
    \item[GUI] Graphical user interface
    \item[XML] Extensible markup language
\end{description}


% % % % % % % % % % % % % % % % % % % % % % % % % % % % 
% % Tuto kapitolu z výsledné práce ODSTRAŇTE.
% % % % % % % % % % % % % % % % % % % % % % % % % % % % 
% 
% \chapter{Návod k~použití této šablony}
% 
% Tento dokument slouží jako základ pro napsání závěrečné práce na Fakultě informačních technologií ČVUT v~Praze.
% 
% \section{Výběr základu}
% 
% Vyberte si šablonu podle druhu práce (bakalářská, diplomová), jazyka (čeština, angličtina) a kódování (ASCII, \mbox{UTF-8}, \mbox{ISO-8859-2} neboli latin2 a nebo \mbox{Windows-1250}). 
% 
% V~české variantě naleznete šablony v~souborech pojmenovaných ve formátu práce\_kódování.tex. Typ může být:
% \begin{description}
% 	\item[BP] bakalářská práce,
% 	\item[DP] diplomová (magisterská) práce.
% \end{description}
% Kódování, ve kterém chcete psát, může být:
% \begin{description}
% 	\item[UTF-8] kódování Unicode,
% 	\item[ISO-8859-2] latin2,
% 	\item[Windows-1250] znaková sada 1250 Windows.
% \end{description}
% V~případě nejistoty ohledně kódování doporučujeme následující postup:
% \begin{enumerate}
% 	\item Otevřete šablony pro kódování UTF-8 v~editoru prostého textu, který chcete pro psaní práce použít -- pokud můžete texty s~diakritikou normálně přečíst, použijte tuto šablonu.
% 	\item V~opačném případě postupujte dále podle toho, jaký operační systém používáte:
% 	\begin{itemize}
% 		\item v~případě Windows použijte šablonu pro kódování \mbox{Windows-1250},
% 		\item jinak zkuste použít šablonu pro kódování \mbox{ISO-8859-2}.
% 	\end{itemize}
% \end{enumerate}
% 
% 
% V~anglické variantě jsou šablony pojmenované podle typu práce, možnosti jsou:
% \begin{description}
% 	\item[bachelors] bakalářská práce,
% 	\item[masters] diplomová (magisterská) práce.
% \end{description}
% 
% \section{Použití šablony}
% 
% Šablona je určena pro zpracování systémem \LaTeXe{}. Text je možné psát v~textovém editoru jako prostý text, lze však také využít specializovaný editor pro \LaTeX{}, např. Kile.
% 
% Pro získání tisknutelného výstupu z~takto vytvořeného souboru použijte příkaz \verb|pdflatex|, kterému předáte cestu k~souboru jako parametr. Vhodný editor pro \LaTeX{} toto udělá za Vás. \verb|pdfcslatex| ani \verb|cslatex| \emph{nebudou} s~těmito šablonami fungovat.
% 
% Více informací o~použití systému \LaTeX{} najdete např. v~\cite{wikilatex}.
% 
% \subsection{Typografie}
% 
% Při psaní dodržujte typografické konvence zvoleného jazyka. České \uv{uvozovky} zapisujte použitím příkazu \verb|\uv|, kterému v~parametru předáte text, jenž má být v~uvozovkách. Anglické otevírací uvozovky se v~\LaTeX{}u zadávají jako dva zpětné apostrofy, uzavírací uvozovky jako dva apostrofy. Často chybně uváděný symbol "{} (palce) nemá s~uvozovkami nic společného.
% 
% Dále je třeba zabránit zalomení řádky mezi některými slovy, v~češtině např. za jednopísmennými předložkami a spojkami (vyjma \uv{a}). To docílíte vložením pružné nezalomitelné mezery -- znakem \texttt{\textasciitilde}. V~tomto případě to není třeba dělat ručně, lze použít program \verb|vlna|.
% 
% Více o~typografii viz \cite{kobltypo}.
% 
% \subsection{Obrázky}
% 
% Pro umožnění vkládání obrázků je vhodné použít balíček \verb|graphicx|, samotné vložení se provede příkazem \verb|\includegraphics|. Takto je možné vkládat obrázky ve formátu PDF, PNG a JPEG jestliže používáte pdf\LaTeX{} nebo ve formátu EPS jestliže používáte \LaTeX{}. Doporučujeme preferovat vektorové obrázky před rastrovými (vyjma fotografií).
% 
% \subsubsection{Získání vhodného formátu}
% 
% Pro získání vektorových formátů PDF nebo EPS z~jiných lze použít některý z~vektorových grafických editorů. Pro převod rastrového obrázku na vektorový lze použít rasterizaci, kterou mnohé editory zvládají (např. Inkscape). Pro konverze lze použít též nástroje pro dávkové zpracování běžně dodávané s~\LaTeX{}em, např. \verb|epstopdf|.
% 
% \subsubsection{Plovoucí prostředí}
% 
% Příkazem \verb|\includegraphics| lze obrázky vkládat přímo, doporučujeme však použít plovoucí prostředí, konkrétně \verb|figure|. Například obrázek \ref{fig:float} byl vložen tímto způsobem. Vůbec přitom nevadí, když je obrázek umístěn jinde, než bylo původně zamýšleno -- je tomu tak hlavně kvůli dodržení typografických konvencí. Namísto vynucování konkrétní pozice obrázku doporučujeme používat odkazování z~textu (dvojice příkazů \verb|\label| a \verb|\ref|).
% 
% \begin{figure}\centering
% 	\includegraphics[width=0.5\textwidth, angle=30]{cvut-logo-bw}
% 	\caption[Příklad obrázku]{Ukázkový obrázek v~plovoucím prostředí}\label{fig:float}
% \end{figure}
% 
% \subsubsection{Verze obrázků}
% 
% % Gnuplot BW i barevně
% Může se hodit mít více verzí stejného obrázku, např. pro barevný či černobílý tisk a nebo pro prezentaci. S~pomocí některých nástrojů na generování grafiky je to snadné.
% 
% Máte-li například graf vytvořený v programu Gnuplot, můžete jeho černobílou variantu (viz obr. \ref{fig:gnuplot-bw}) vytvořit parametrem \verb|monochrome dashed| příkazu \verb|set term|. Barevnou variantu (viz obr. \ref{fig:gnuplot-col}) vhodnou na prezentace lze vytvořit parametrem \verb|colour solid|.
% 
% \begin{figure}\centering
% 	\includegraphics{gnuplot-bw}
% 	\caption{Černobílá varianta obrázku generovaného programem Gnuplot}\label{fig:gnuplot-bw}
% \end{figure}
% 
% \begin{figure}\centering
% 	\includegraphics{gnuplot-col}
% 	\caption{Barevná varianta obrázku generovaného programem Gnuplot}\label{fig:gnuplot-col}
% \end{figure}
% 
% 
% \subsection{Tabulky}
% 
% Tabulky lze zadávat různě, např. v~prostředí \verb|tabular|, avšak pro jejich vkládání platí to samé, co pro obrázky -- použijte plovoucí prostředí, v~tomto případě \verb|table|. Například tabulka \ref{tab:matematika} byla vložena tímto způsobem.
% 
% \begin{table}\centering
% 	\caption[Příklad tabulky]{Zadávání matematiky}\label{tab:matematika}
% 	\begin{tabular}{|l|l|c|c|}\hline
% 		Typ		& Prostředí		& \LaTeX{}ovská zkratka	& \TeX{}ovská zkratka	\tabularnewline \hline \hline
% 		Text		& \verb|math|		& \verb|\(...\)|	& \verb|$...$|		\tabularnewline \hline
% 		Displayed	& \verb|displaymath|	& \verb|\[...\]|	& \verb|$$...$$|	\tabularnewline \hline
% 	\end{tabular}
% \end{table}
% 
% % % % % % % % % % % % % % % % % % % % % % % % % % % % 

\chapter{Obsah přiloženého CD}

%upravte podle skutecnosti

\begin{figure}
    \dirtree{%
        .1 readme.txt\DTcomment{stručný popis obsahu CD}.
        .1 exe\DTcomment{adresář se spustitelnou formou implementace}.
        .1 src.
        .2 impl\DTcomment{zdrojové kódy implementace}.
        .2 thesis\DTcomment{zdrojová forma práce ve formátu \LaTeX{}}.
        .1 text\DTcomment{text práce}.
        .2 thesis.pdf\DTcomment{text práce ve formátu PDF}.
        .2 thesis.ps\DTcomment{text práce ve formátu PS}.
    }
\end{figure}

\end{document}
