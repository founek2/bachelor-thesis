% arara: pdflatex
% arara: pdflatex
% arara: pdflatex

% options:
% thesis=B bachelor's thesis
% thesis=M master's thesis
% czech thesis in Czech language
% slovak thesis in Slovak language
% english thesis in English language
% hidelinks remove colour boxes around hyperlinks
 
\documentclass[thesis=B,czech]{FITthesis}[2019/12/23]
 
\usepackage[utf8]{inputenc} % LaTeX source encoded as UTF-8
\usepackage{adjustbox}   

% \usepackage{amsmath} %advanced maths
% \usepackage{amssymb} %additional math symbols
 
\usepackage{dirtree} %directory tree visualisation
   
% % list of acronyms
% \usepackage[acronym,nonumberlist,toc,numberedsection=autolabel]{glossaries}
% \iflanguage{czech}{\renewcommand*{\acronymname}{Seznam pou{\v z}it{\' y}ch zkratek}}{}
% \makeglossaries

\newcommand{\tg}{\mathop{\mathrm{tg}}} %cesky tangens
\newcommand{\cotg}{\mathop{\mathrm{cotg}}} %cesky cotangens
\newcommand\tstrut{\rule{0pt}{2.4ex}}
\newcommand\bstrut{\rule[-1.0ex]{0pt}{0pt}}

% % % % % % % % % % % % % % % % % % % % % % % % % % % % % % 
% ODTUD DAL VSE ZMENTE
% % % % % % % % % % % % % % % % % % % % % % % % % % % % % % 

\department{Katedra softwarového inženýrství}
\title{IoT platforma s webovým rozhraním}
\authorGN{Martin} %(křestní) jméno (jména) autora
\authorFN{Skalický} %příjmení autora
\authorWithDegrees{Martin Skalický} %jméno autora včetně současných akademických titulů
\author{Martin Skalický} %jméno autora bez akademických titulů
\supervisor{Ing. Jiří Mlejnek}
%\acknowledgements{Doplňte, máte-li komu a za co děkovat. V~opačném případě úplně odstraňte tento příkaz.}
\abstractCS{Ze slova IOT se v posledních letech stal buzzword, pod kterým si lze představit téměř cokoliv od chytré žárovky až po automatizovanou linku. IOT platforma označuje  platformu, ke které lze připojit chytré věci jako např. chytrou žárovku či teplotní senzor a následně data ze zařízení zobrazovat, analyzovat a vzdáleně ovládat mimo jiné ze svého telefonu. Tato práce se zabývá porovnáním aktuálních platforem na trhu pro domácnosti a následným návrhem vlastního řešení včetně implementace. Výsledná platforma bude primárně určena pro domácí kutily, kteří si vyrobý DIY zařízení. Uživatel bude s platformou interagovat pomocí Progresivní webové aplikace. Součástí práce je dokumentace pro připojení zařízení k platformě včetně několika celých řešení založených na chipu ESP8266.}
\abstractEN{Sem doplňte ekvivalent abstraktu Vaší práce v~angličtině.}
\placeForDeclarationOfAuthenticity{V~Praze}
\declarationOfAuthenticityOption{4} %volba Prohlášení (číslo 1-6)
\keywordsCS{ESP8266, PWA, MQTT}
\keywordsEN{Nahraďte seznamem klíčových slov v angličtině oddělených čárkou.}
% \website{http://site.example/thesis} %volitelná URL práce, objeví se v tiráži - úplně odstraňte, nemáte-li URL práce

\begin{document}

% \newacronym{CVUT}{{\v C}VUT}{{\v C}esk{\' e} vysok{\' e} u{\v c}en{\' i} technick{\' e} v Praze}
% \newacronym{FIT}{FIT}{Fakulta informa{\v c}n{\' i}ch technologi{\' i}}

\begin{introduction}
    Tato práce se zabývá tvorbou IOT Platformy. Platforma je jedním z nejdůležitějších prvků ve světě internetu věcí (IOT). Internet věcí je označení pro síť fyzických zařízení, která jsou vybavena konektivitou tj. od chytré žárovky až po chytré vozidlo. Tyto zařízení mohou spolu komunikovat přes různé technologie od WiFi, Bluethooth, LTE až po specializované sítě optimalizované pro nízkou spotřeba, které umožňují provoz chytrých zařízení na baterii s životností až desítky let (LoRa, Sigfox).
    
    V domácnostech se zatím nejvíce prosazuje využití WiFi, prototže tuto síť má dnes doma téměř každý. Takovýchto chytrých zařízení lze mít v domě desítky a přirozeně se nabízí otázka možnosti tyto zařízení ?kombinovat? a automatizovat. A k tomuto účelu je právě potřeba Platforma, která zařízení bude "zastřešovat" a dá nám k dispozici jednotné uživatelské rozhraní ne jenom k zobrazování dat např. ukáže teplotu z meteostanice, ale umožní ovládat jednotlivá zařízení např. z telefonu zatažení žaluzií. Toto jsou pouze základní funkce a chytřejší řešení umožňuní automatizaci např. pokud čidlo zaznamená pohyb v domě, tak se automaticky zapne kotel pro vytápění. 
    %představit koncepci koncovích zařízení
    %využití - sledování vláhy, zalévání, udírna, ovládání hlasem
    
    % Takovýchto řešení již dnes existuje celá řada a podíváme se na výhody a nevýhody těch nejznámějších řešení jak z komerčního, tak i OpenSource světa.
    % Následně se budeme věnovat návrhu vlastního řešení s důrazem na bezpečnost, kterou stále spousta výrobců opomíjí. Moje řešení bude mít webové rozhraní, které umožní ovládání ze všech běžných zařízení bez nutnosti vytvářet nativní aplikace.


\end{introduction}

\chapter{Cíl práce}
    % návrh vlastní platformy, cíleno pro kutily, bezpečnost pochybná u komerčních řešení -> já chci mít velkou, moje řešení bude více decentralizované
    Cílem této práce je porovnat výhody a nevýhody těch nejznámějších řešení jak z komerčního, tak i OpenSource světa a následně navrhnout a realizovat vlastní řešení. Tato Platforma bude koncipována primárně pro cloudové nasazení. Bude umožňovat registraci uživatelům, kteří si potom budou moci spravovat svá vlastní zařízení včetně oprávnění. Na rozdíl od komerčních řešení, která jsou založena čistě na centrální správně, tak tady bude umožněn jak centralizovaný přístup (přes platformu), tak i možnost decentralizovaného - zařízení se budou moci přihlásit k odběru události z ostatních zařízení a na základně toho reagovat. Otevře se tak možnost využít pouze nejzákladnější funkce z platformy, ale veškerou automatizaci a následné reakce implementovat pouze ve firmwaru jednotlivých zařízení. Tento přístup je pro implementaci složitější, ale zvyšuje následnou odolnost v případě výpadku Platformy.

    Řešení bude cíleno primárně pro domáci kutily, kteří si budou moci připojit libovolné DIY zařízení, nebo pro specializované nasazení, které vyžaduje vysokou spolehlivost.

    Vzhledem k rozsahu Bakalářské práce není cílem nahradit existující řešení, která již obsahují velké množství funkcí, ale vytvořit možnou alternativu a demonstrovat složitost vytvoření celého řešení za použití moderních technologií. 

    

\chapter{Existující řešení IOT Platforem}
Tato kapitola se zabývá definici IOT Platformy a analýzi již existujících řešení. V závěru kapitoly je matice obsahující porovnání známích řešení vůči jejich funkcím.

\section{Co je IOT Platforma}
    "IOT Platforma je více vrstvá technologie, která umožňuje přímočaré zajištění, ovládání a automatizaci připojených zařízení ve světě internetu věci. Zjednodušeně propojuje Váš hardware, jakkoli rozdílný, do cloudu s možností různorodé konektivity, obsahuje bezpečnostní mechanizmy a široké možnosti pro zpracování dat. Pro vývojáře, IOT Platforma nabízí soubor předpřipravených funkcí, které vysoce zvyšují rychlost vývoje aplikací pro připojená zařízení a řeší škálování a kompatibilitu napříč zařízeními" [*, překlad autora] %https://www.kaaproject.org/blog/what-is-iot-platform 

\subsection{Definice pojmů}
V této sekci jsou vysvětleny pojmy, které budou použity v následujících kapitolách.
    
\begin{itemize}
    \item \textbf{Platforma} -
    \item \textbf{Koncové zařízení} -
    \item \textbf{GateWay} -
    \item \textbf{DIY} -
\end{itemize}

\section{Bezpečnost a soukromí}
Při výběru Platformy by důležitým kritériem měla být bezpečnost. Na první pohled se to nemusí zdát být důležité, co se může stát když bude s platformou komunikovat čidlo pohybu a někdo se dokáže dostat k těmto údajům? Například pro zloděje mohou být taková data zlatý důl, protože bude přesně vědět kdy je dům prázdný.
 
\section{Cílová skupina}
    dasdasd

\section{Komunikace}    % rozdíly v komunikačním mediu wifi, bluethoot, mesh řešení Z-wave a zigbee, LoRa, spotřeba, bateriový provoz, složitost instalace
    Pro komunikaci mezi zařízeními se nejčastěmi používá bezdrátová komunikaci, kvůli jednoduchosti instalace bez nutnosti většího zásahu do stávající infrastruktury. Tento typ lze rozdělit do dvou kategorií:
    \begin{itemize}
        \item \textbf{Centrální} - Každé zařízení komunikuje pouze s jedním centrálním prvke, přes který jde veškerá komunikace. Mezi nejznámější technologii tohoto typu patří Wifi, kterou dnes zná každý.
        \item \textbf{Decentrlizovanou} - V této síti komunikují zařízení přímo s ostatními bez jakéholiv prostředníka. Síť je díky tomu mnohem odolnější vůči výpadků, protože zde není tzv. "Single point of failure". Zpravidla mývá nižší datovou propustnost a je složitější pro nasazení a následnou zprávu.
    \end{itemize}
    Vzhledem k rozšířenosti Wifi, kterou dnes najdeme v každé domácnosti, se přirozeně nabízí využít tuto možnost i pro internet věcí. A k tomu v posledních letech opravdu došlo. Díky extrémně levnému chipu ESP8266, který se dnes v ČR dá koupit za 40 Kč, došlo k masivní penetraci trhu s chytrými zařízeními využívající právě Wifi. Bohužel tato technologie má i svá negativa a největší jsou spotřeba elektrické energie a maximální počet připojenách zařízení (desítky na jeden centrální prvek). Jednotlivá zařízení musí neustále komunikovat s velkým výkonem a kvůli je možný rozumný provoz na baterii pouze v řádu týdnů.

    Proto pro účel vznikly speciální sítě, které sice na rozdíl od Wifi umožní přenos v desítka kp za sekundu (tisícina rychlosti běžné Wifi) a jsou energeticky mnohem úspornější (umožnují provoz až desítky let na baterii), mají mnohonásobně větší dosah a centrální prvek dokáže současně komunikovat s tisíci zařízeními. 
    
    Nejvíce rozšířenými z centrálně orientovaných jsou SigFox a LoRa. SigFox je komerční řešení, kde se platí za každé připojené zařízení. Oproti tomu síť LoRa používá otevřený standard pro komunikaci LoRaWAN. Protože se jedná o otevřený standard, tak kdokoliv může vytvořit kompatibilní zařízení. Dále krom možnosti využití komerční infrastruktury, kam lze připojit svá zařízení za poplatek, tak díky otevřenosti má každý možnost si za pár tisíc postavit vlastní GateWay a provozovat libovolná zařízení bez jakýhkoliv poplatků.

    Z technologií typu mesh se nejvíce ujali Zigbee vs Z-Wave. Zigbee je otevřený standart, zatímco Z-Wave není. Spotřeba energie je u obou technologií velmi podobná ale liší v dosahu.



\section{Komerční řešení} % hotové řešení, cloud, ale drahé
%easy to use, but paid
Na trhu dnes existuje velké množství komerčních řešení od známých výrobců, někteří jsou známí spíše výrobou harwaru jako Philips a Xiaomi, jiní se zaměřují spíše na nabídku služeb a integraci zařízení ostatních výrobců pod svojí Platformu jako Blynk nebo Google. Pro koncového zákazníka mají Komerční řešení obrovskou výhodu v jednoduchosti prvotního nasazení a následné obsluhy. Stačí zakoupit centrální jednotku a libovolná zařízení od stejného výrobce a vše krásně funguje. Problém ale nastává ve chvíli, kdy potřebují řešení škálovat či customizovat dle svých potřeb, protože si dodavatel za úpravy na ,,míru" začne účtovat obrovské částky a zákazníkovi nezbývá nic jiného než platit. Sám si potřebné změny udělat nemůže, protože nemá zdrojové kódy a migrace k jinému produktu by znamenal obrovské náklady a problémy se stávájícími integracemi, protože různá řešení mívají různá rozhraní.
  
%security
Aspekt bezpečnosti u uzavřených řešení bývá diskutabilní. Pravidelné bezpečnostní audity kvůli nákladům provádím málo kdo. Výrobci samozřejmně vždy tvrdí, že bezpečnost je u nich na prvním místě, ale bohužel tento aspekt je v přímém kontrastu s jenododuchostí použití, což je pro výrobce mnohem důležitější protože pokud se řešení dobře a jednoduše ovládá, tak mnohem spíše si ho zákazníci oblíbí, než pokud bude maximálně zabezpečeno, ale uživatel bude muset provádět úkony návíc čistě kvůli bezpečnosti, která mu na první pohled nepřínáší přidanou hodnotu.

%Cloud dependent
U Platformy očekáváme možnost vzdáleného ovládání, tedy přístup odkudkoli z internetu. Málokdo má však doma veřejnou IP adresu, aby si mohl celé řešení provozovat doma tvz. self hosting. V Praxi si tedy uživatel pořidí domů GateWay, která komunikoje s chytrými zařízeními v domácnosti a současně s cloudem výrobce, přes který lze přistupovat na Platformu a ovládat všechny zařízení. Takové řešení se velmi osvědčilo díky jednoduchosti, protože neklade žádné nároky na uživatele jako např. veřejnou IP adresu. Problém však může nastat ve chvíli, kdy výrobce daného řešení po několika letech ukončí činnost a s tím přestane provozovat svojí cloudovou infrastrukturu, na které je závislá GateWay a vzdálený přístup z internetu. V lepším případě bude zachována funkčnost v lokální sití v horším přestane řešení fungovat úplně. Najednou uživatelovi zbyde doma spousta funkčního (po fyzické stránce) harwaru, který nemůže využívat.

Výše jsem nastínil nejhorší možný scénář, který naštěstí v poslední době již přestává platit, protože výrobci společně vytvářejí otevřené standardy pro komunikaci, které by měli zaručit kompatibilitu zařízení napříč jednotlivými výrobci. Bohužel standardů vzniká současně více a ne všichni je plně dodržují, takže nekompatibilita ještě bude delší dobu přetrvávat i když ne v takovém měřítku jako před pár lety. Kromně rozdílných protokolů je také nekompatibilita v různých technologiích přenosu mezi nejznámější patří WiFi, Bluethooth, LoRa, Zigbee a Sigfox. 
%podpora jiných výrobců? integrace? -> závislé na tom co výrobce se rozhodne implementovat

\section{OpenSource řešení}
% nepopulární/špatná reputace mezi lidmy, často potřeba znalosti problematiky, Free, flexibilní, customizovatelné
OpenSource řešení mají mezi širší veřejností špatnou reputaci, protože na rozdíl od komerčních "PlugAndPlay" produktů většinou vyžadují určité povědomí o dané problematice. Je to způsobeno tím, že se snaží pokrýt celou doménu stejně jako komerční řešení, ale oproti nim se zlomkem vývojářů a financí. Následkem toho není prvotní nastavení pro laika zcela přímočaré a může se střetnout s problémy. Avšak překonání prvotních nesnázích přináší spoustu pozitiv. 

Jedním z nejatraktivnějších je samozřejmně cena. OpenSource řešení jsou zpravidla zcela zdarma, případně nabícejí placenou podporu. Mě osobně na OpenSource nejvíce zaujala komunita. Pokud se projekt dostane do určité známosti, tak kolem něho začně vznikat komunita lidí, primárně technologických nadšenců ale i lidí z IT praxe, kteří mezi sebou komunikují a spolupracují na vylepšení daného řešení, ať už přímo (napsání části funkcionality) nebo nepřímo (komunikace s vývojáři). Potom i obyčejný uživatel, který chce řešení využít, tak při objevení potíží, může požádat komunitu o pomoc a protože to jsou nadšení lidé, tak velmi rádi pomohou. 

Pokud máme dostatečné technické znalosti, tak si můžeme prohlédlou přímo zdrojové kódy a sami si zhodnotit kvalitu i bezpečnost. U větších projektů to však již není tak úplně možné při desítkách tisíc rádků kódu, ale existují lidé, kteří tomu opravdu věnují čas a mohou tak objevit zranitelnosti. Dále OpenSource projekty bývají mnohem více sdílné ohledně architektury kterou využívají a je možno se v dokumentaci dočíst, jak vlastně řešení funguje interně, na rozdíl od komerčních, kde je to tzv. BlackBox.

OpenSource platformy bývají postavené na systému Pluginů, tedy obsahují určitou základní sadu funkcí a dále lze funkčnost rozšiřovat pomocí instalace Pluginů. Ty mohou vytvářet přímo autoři nebo kdokoli jiný dle potřeb. Díky tomu jsou velmi robustní a podporují širokou škálu zařízení od různých výrobců napříč technologiemi a pokud ne, tak s trochou znalostí v programování si může každý dopsat plugin dle potřeb pro podporu daného zařízení.


\section{Známé Platformy}  %rešerže jednotlivých platforem + závěrečná matice
Tato sekce s zabývá analýzou 4 nejznámějších Platforem a v závěru jejich vzájemným porovnáním.

\subsection{Blynk}
Blynk se označuje jako harware-agnostic IOT Platforma s white-label mobilními aplikacemi. Umožnuje navrhnovat vlastní aplikace formou DragAndDrop pro ovládání zařízení, analýzu telemetrických dat a správu nasazených produktů ve velkém měřítku. Své řešení nabízejí jak pro domácí nasazení, tak i jako enterprise řešení pro větší firmy. Mají \textbf{3 cenové tarify}:
\begin{itemize}
\item \textbf{Free} je omezené pouze pro osobní užití, obsahuje cloudový hosting, umožňuje připojit maximálně 5 zařízení zdarma a součástí je mobilní aplikace pro Android a iOS.
\item \textbf{StartUp} je určeny pro komerční využití a cenou začínají na \$415/měsíc. Součástí je deployment vlastních aplikací na AppStore/Google Play, neomezený počet zařízení a uživatelů, garantované podpora
\item \textbf{Business} začíná na \$1000/měsíc a nabízí navíc OTA aktualizace koncových zařízení (vzdáleně), webové rozhraní, datovou analýzou a dalších funkce.
\end{itemize}

Hardware-agnostic znamená, že nejsou omezeni pouze na určitý hardware a umožňují připojit v podstatě libovolné zařízení. Pro připojení maji definované rozhraní nad jednotlivými protokoli. Podporují custom TCP/IP, WebSocket, HTTP a nově i MQTT (zatím k němu nemají ale dokumentaci). Dávají k dispozici knihovný pro různé harwarové platformy, takže připojení k platformě je potom otázka dvou řádků kódu. K dispozici je velmi přehledná a detailní dokumentace.

Nativní aplikace pro iOS a Android umožnuje vytvářet vlastní dashboardy pomocí již předpřipravených Widgetů, kterých je opravdu velké množství, ale jsou placené za tzv. Energii, což je měna která lze dobíjet za peníze, dále definovat vlastní widgety a upravit chování celé aplikace. Následně lze takto upravenou aplikaci vyexportovat a přímo nahrát na Google Play a AppStore pod vlastním názvem. Tento přístup nabízí elegantní možnost pro tvorbu vlastního řešení, které následně je možné nabízet jako vlastní produkt.

Výhodou cloudové řešení je přístup k platformě odkudkoliv z internetu. Zároveň je potom ale funčknost odkázána na dostupnost internetového připojení a představa dat někde v cloudu se také každému nemusí líbit. Pro tento připad je možnost hostovat si vlastní Blynk server, který je dostupný jako OpenSource server napsaný v Javě. %https://github.com/blynkkk/blynk-server

\subsection{Thingspeaks}
ThingSpeak™ je služba analytické IoT platformy od MathWorks®, tvůrců MATLAB® a Simulink®. Jedná se o hardware-agnostic platformu s webovým rozhraním, která se plně zaměřuje na analýzu dat. Je ideální pro lidi se zkušeností s Matlab, protože je postavena právě na této platformě. Umožňuje v cloudu sběr dat, jejich analýzu přímo pomocí Matlab kódu, vizualizaci dat a definování reakcí. Pro různé harwarové platformy mají připravené knihovny a nativě podporují komunikace pomocí HTTP a MQTT protokolů. Řešení nabízejí podle následujících tarifů, kde omezení jsou primárně ohledně maximálního počtu zpráv, počtu kanálů do kterého posílají zařízení zprávy a minimálního časového odstupu mezi zprávami v rámci jednoho kanálu.
\begin{itemize}
    \item \textbf{Free} ~8 200 messages/day, počet kanálů 4, interval mezi zprávami 15s, Matlab maximální doba běhu 20s
    \item \textbf{STANDARD} ~90 000 messages/day, počet kanálů 250, interval mezi zprávami 60s, Matlab maximální doba běhu 20s
    \item TODO matice pro tarify jednotlivé
\end{itemize}

\subsection{Home Assistant}
OpenSource domací automatizace, která dává lokální kontrolu a soukromí na první místo - takto se prezentuje Home Assistant. Tato platforma není tolik zaměřena na koncová zařízení jako předchozí, ale funguje jako integrátor komerčních/OpenSource řešení pod jednotné rozhraní. Obsahuje systém pro tvorbu automatizace, tedy vytváření reakcí na jednotlivé akce. Dokáže se napojit buď přímo na jednotlivá zařízení nebo na jejich GateWay a umožnit ovládání všech zařízení od různorodých výrobců, kteří často vynucují použití vlastní aplikace, pod jednotné rozhraní jak webové, tak ve formě nativní aplikace. Integrace je řešena pomocí pluginární systému, kde zpravidla jeden plugin obsahuje integraci pro jednoho výrobce/jednu GateWay. Většina pluginů vzniká přímo od komunity této platformy. V době psání této práce obsahuje 1743 pluginů.

Celá platforma je zdarma a pro její zprovoznění stačí Raspberry Pi, na SD kartu nahrát předpřipravený image a zapnout. Prvnotním nastavením Vás následně provede webové rozhraní nebo nativní aplikace, záleží na Vaší volbě. 

Platforma podporuje velké množství komerčních produkté mezi nejznámější patří Ikea TRÅDFRI, Philips Hue či Google Assistant. Samozřejmně podporují i OpenSource řešení mezi nejznámější patří ESPHome, což je framework pro konfiguraci ESP chipů (ESP8266/ESP32), který řeší vrstvu komunikace a zapojení do platformy. Stačí tedy pouze dodefinovat chování na určité události a chytré zařízení je připravené.

\subsection{OpenHAb}
%projekt s dlouho historií (v1 in 2010), lepší dokumentace, trochu složitější nastavení, mohutnější ale více možnost oproti HA, systém Addonů 324 aktuálně
OpenHab je OpenSource projekt s dlouhou historií, který vznikl již v roce 2010. Cílí na stejný segment jako Home Assistant, tedy  propojení existujících řešení pod jednotné rozhraní a jejich automatizaci. Jedná se o hardware agnostic platformu, která komunikuje přímo s koncovými zařízeními nebo příslušnou GateWay. V základu obsahuje více funkcionalit, zatímco Homa Assistant je více minimalistický. Pro rozšířování funkcnionality používá systém doplňků (aktuálně 324), které vyvájejí autoři a především komunita. Od prvopočátku projektu je zde kladen velký důraz na nativní aplikace na rozdíl od Home assistantu, který dlouhou dobu neměl oficiální aplikaci pro Android. Webové rozhraní je samozřejmostí. Velkou výhodou je možnost využití cloud instance zcela zdarma, buď jako plnohodnotnou platformu nebo pouze pro přístup z internetu k vlastní instanci.

Prvotní instalace je stejně jednoduchá jako u Home Assistentu. Rozdíl přichází při přidávání jednotlivých zařízení, kde je proces trochu komplikovanější. OpenHab se snaží nabídnout pokročilejší funkcionalitu, která bohužel částečně zesložiťuje jednotlivé procesy. Na druhou stranu umožňuje větší flexibilitu.
 
Dokumentace projektu je na velmi vysoké úrovni s velmi detailním popisem. Pravděpodobně díky tomu, že projekt existuje již 10 let a má silnou základnu v komunitě i přes to, že je přibližně poloviční oproti té, kterou má Home Assistant. 

\subsection{Porovnání}
% TODO závěrečné porovnání - Blynk více orientované na koncová zařízení, ThingSpeaks primárně pro analýzu dat s Matlabem, HA a openHab jsou HUBy pro domácí automatizaci 
Jednotlivé Platforma se některými funkcemi překrývají a v jiných jsou zase jedinečné. Při výběru je důležité si stanovit na co Platformu chceme využívat a jaké funkce vyžadujeme. 

Blynk primárně cílí na podnikatelský segment, a nejvíce se hodí firmám, kteří chtějí na této Platforma vysvtavět své řešení, které následně budou přeprodávat pod svojí vlastní Značkou, jak díky příme možnosti exportu aplikace na AppStore a Google Play, hromadné zprávě zařízení nebo ACL. Kvůli chybějící podpoře komerčních zařízení lze využít pro domácnost pouze s DYI zařízeními.

ThingSpeaks míří primárně na zpracování dat díky svému ekosystému postaveném kolem MATLAB®. Pro veškeré zpracování, analýzi a vizualizace stačí znalost prostředí MATLAB®, který je světově známý.

Home Assistant je progresivní OpenSource Platforma, která umožní integraci komerčních řešení pod jednotné rozhraní a domácí automatizaci. Příjemné uživatelské rozhraní.

OpenHab projekt s dlouho historií. Funkčně se velmi podobá Home Assistantu, ale snaží se uživatelům nabídnout více funkčnosti. Uživatelské rozhraní je občas trochu složitější.


\begin{center} % pro addony přidat poznámku 324 obsahuje 2585 věcí
    \begin{tabular}{ |c| m{5em}| m{5em}|m{5em}|m{4em}| m{5em}| m{4em}| m{4em}| } 
     \hline 
     Platforma & Podpora komerčních produktů & Vlastní zařízení & Hosting & ACL & Nativní aplikace & Správa zařízení & Cena \\
     \hline
     Blynk & Ne & Ano & self-hosted, cloud & Pouze Enterprise & Ano & Ano &  Omezený Free plan\\
     \hline
     ThingSpeaks & 6 vendorů (primárně LoRa) & Ano & cloud & Ano & Pouze pro náhled na data & Ne & Omezený Free plan\\
     \hline
Home Assistant & pomocí Pluginů (1743) & 3rd party knihovny & self-hosted & Ano & Ano & Ne & Ano\\
\hline
openHab & pomocí Doplňků (324) & 3rd party knihovny & self-hosted, cloud & Ne & Ano & Ne & Ano\\
     \hline
    \end{tabular}
    \end{center} 


\chapter{Analýza a návrh}

\section{Analýza procesů}

\section{Stanovení požadavků}

\section{Návrh architektury}

\section{Výběr technologií}

\section{Návrh uživatelského rozhraní}




\chapter{Realizace}

\section{Server}

\section{Klientská část}

\section{Komunikace}

\section{Knihovna pro ESP8266}

\section{chytrá udírna}

\subsection{Návrh zapojení}

\subsection{Výroba}



\chapter{Testování}

\section{Zátěžové testování}

\section{Zotavení po nenadálé události}

\section{Uživatelské testování}



\begin{conclusion}
	%sem napište závěr Vaší práce
\end{conclusion}

\bibliographystyle{csn690}
\bibliography{mybibliographyfile}

\appendix

\chapter{Seznam použitých zkratek}
% \printglossaries
\begin{description}
	\item[GUI] Graphical user interface
	\item[XML] Extensible markup language
\end{description}


% % % % % % % % % % % % % % % % % % % % % % % % % % % % 
% % Tuto kapitolu z výsledné práce ODSTRAŇTE.
% % % % % % % % % % % % % % % % % % % % % % % % % % % % 
% 
% \chapter{Návod k~použití této šablony}
% 
% Tento dokument slouží jako základ pro napsání závěrečné práce na Fakultě informačních technologií ČVUT v~Praze.
% 
% \section{Výběr základu}
% 
% Vyberte si šablonu podle druhu práce (bakalářská, diplomová), jazyka (čeština, angličtina) a kódování (ASCII, \mbox{UTF-8}, \mbox{ISO-8859-2} neboli latin2 a nebo \mbox{Windows-1250}). 
% 
% V~české variantě naleznete šablony v~souborech pojmenovaných ve formátu práce\_kódování.tex. Typ může být:
% \begin{description}
% 	\item[BP] bakalářská práce,
% 	\item[DP] diplomová (magisterská) práce.
% \end{description}
% Kódování, ve kterém chcete psát, může být:
% \begin{description}
% 	\item[UTF-8] kódování Unicode,
% 	\item[ISO-8859-2] latin2,
% 	\item[Windows-1250] znaková sada 1250 Windows.
% \end{description}
% V~případě nejistoty ohledně kódování doporučujeme následující postup:
% \begin{enumerate}
% 	\item Otevřete šablony pro kódování UTF-8 v~editoru prostého textu, který chcete pro psaní práce použít -- pokud můžete texty s~diakritikou normálně přečíst, použijte tuto šablonu.
% 	\item V~opačném případě postupujte dále podle toho, jaký operační systém používáte:
% 	\begin{itemize}
% 		\item v~případě Windows použijte šablonu pro kódování \mbox{Windows-1250},
% 		\item jinak zkuste použít šablonu pro kódování \mbox{ISO-8859-2}.
% 	\end{itemize}
% \end{enumerate}
% 
% 
% V~anglické variantě jsou šablony pojmenované podle typu práce, možnosti jsou:
% \begin{description}
% 	\item[bachelors] bakalářská práce,
% 	\item[masters] diplomová (magisterská) práce.
% \end{description}
% 
% \section{Použití šablony}
% 
% Šablona je určena pro zpracování systémem \LaTeXe{}. Text je možné psát v~textovém editoru jako prostý text, lze však také využít specializovaný editor pro \LaTeX{}, např. Kile.
% 
% Pro získání tisknutelného výstupu z~takto vytvořeného souboru použijte příkaz \verb|pdflatex|, kterému předáte cestu k~souboru jako parametr. Vhodný editor pro \LaTeX{} toto udělá za Vás. \verb|pdfcslatex| ani \verb|cslatex| \emph{nebudou} s~těmito šablonami fungovat.
% 
% Více informací o~použití systému \LaTeX{} najdete např. v~\cite{wikilatex}.
% 
% \subsection{Typografie}
% 
% Při psaní dodržujte typografické konvence zvoleného jazyka. České \uv{uvozovky} zapisujte použitím příkazu \verb|\uv|, kterému v~parametru předáte text, jenž má být v~uvozovkách. Anglické otevírací uvozovky se v~\LaTeX{}u zadávají jako dva zpětné apostrofy, uzavírací uvozovky jako dva apostrofy. Často chybně uváděný symbol "{} (palce) nemá s~uvozovkami nic společného.
% 
% Dále je třeba zabránit zalomení řádky mezi některými slovy, v~češtině např. za jednopísmennými předložkami a spojkami (vyjma \uv{a}). To docílíte vložením pružné nezalomitelné mezery -- znakem \texttt{\textasciitilde}. V~tomto případě to není třeba dělat ručně, lze použít program \verb|vlna|.
% 
% Více o~typografii viz \cite{kobltypo}.
% 
% \subsection{Obrázky}
% 
% Pro umožnění vkládání obrázků je vhodné použít balíček \verb|graphicx|, samotné vložení se provede příkazem \verb|\includegraphics|. Takto je možné vkládat obrázky ve formátu PDF, PNG a JPEG jestliže používáte pdf\LaTeX{} nebo ve formátu EPS jestliže používáte \LaTeX{}. Doporučujeme preferovat vektorové obrázky před rastrovými (vyjma fotografií).
% 
% \subsubsection{Získání vhodného formátu}
% 
% Pro získání vektorových formátů PDF nebo EPS z~jiných lze použít některý z~vektorových grafických editorů. Pro převod rastrového obrázku na vektorový lze použít rasterizaci, kterou mnohé editory zvládají (např. Inkscape). Pro konverze lze použít též nástroje pro dávkové zpracování běžně dodávané s~\LaTeX{}em, např. \verb|epstopdf|.
% 
% \subsubsection{Plovoucí prostředí}
% 
% Příkazem \verb|\includegraphics| lze obrázky vkládat přímo, doporučujeme však použít plovoucí prostředí, konkrétně \verb|figure|. Například obrázek \ref{fig:float} byl vložen tímto způsobem. Vůbec přitom nevadí, když je obrázek umístěn jinde, než bylo původně zamýšleno -- je tomu tak hlavně kvůli dodržení typografických konvencí. Namísto vynucování konkrétní pozice obrázku doporučujeme používat odkazování z~textu (dvojice příkazů \verb|\label| a \verb|\ref|).
% 
% \begin{figure}\centering
% 	\includegraphics[width=0.5\textwidth, angle=30]{cvut-logo-bw}
% 	\caption[Příklad obrázku]{Ukázkový obrázek v~plovoucím prostředí}\label{fig:float}
% \end{figure}
% 
% \subsubsection{Verze obrázků}
% 
% % Gnuplot BW i barevně
% Může se hodit mít více verzí stejného obrázku, např. pro barevný či černobílý tisk a nebo pro prezentaci. S~pomocí některých nástrojů na generování grafiky je to snadné.
% 
% Máte-li například graf vytvořený v programu Gnuplot, můžete jeho černobílou variantu (viz obr. \ref{fig:gnuplot-bw}) vytvořit parametrem \verb|monochrome dashed| příkazu \verb|set term|. Barevnou variantu (viz obr. \ref{fig:gnuplot-col}) vhodnou na prezentace lze vytvořit parametrem \verb|colour solid|.
% 
% \begin{figure}\centering
% 	\includegraphics{gnuplot-bw}
% 	\caption{Černobílá varianta obrázku generovaného programem Gnuplot}\label{fig:gnuplot-bw}
% \end{figure}
% 
% \begin{figure}\centering
% 	\includegraphics{gnuplot-col}
% 	\caption{Barevná varianta obrázku generovaného programem Gnuplot}\label{fig:gnuplot-col}
% \end{figure}
% 
% 
% \subsection{Tabulky}
% 
% Tabulky lze zadávat různě, např. v~prostředí \verb|tabular|, avšak pro jejich vkládání platí to samé, co pro obrázky -- použijte plovoucí prostředí, v~tomto případě \verb|table|. Například tabulka \ref{tab:matematika} byla vložena tímto způsobem.
% 
% \begin{table}\centering
% 	\caption[Příklad tabulky]{Zadávání matematiky}\label{tab:matematika}
% 	\begin{tabular}{|l|l|c|c|}\hline
% 		Typ		& Prostředí		& \LaTeX{}ovská zkratka	& \TeX{}ovská zkratka	\tabularnewline \hline \hline
% 		Text		& \verb|math|		& \verb|\(...\)|	& \verb|$...$|		\tabularnewline \hline
% 		Displayed	& \verb|displaymath|	& \verb|\[...\]|	& \verb|$$...$$|	\tabularnewline \hline
% 	\end{tabular}
% \end{table}
% 
% % % % % % % % % % % % % % % % % % % % % % % % % % % % 

\chapter{Obsah přiloženého CD}

%upravte podle skutecnosti

\begin{figure}
	\dirtree{%
		.1 readme.txt\DTcomment{stručný popis obsahu CD}.
		.1 exe\DTcomment{adresář se spustitelnou formou implementace}.
		.1 src.
		.2 impl\DTcomment{zdrojové kódy implementace}.
		.2 thesis\DTcomment{zdrojová forma práce ve formátu \LaTeX{}}.
		.1 text\DTcomment{text práce}.
		.2 thesis.pdf\DTcomment{text práce ve formátu PDF}.
		.2 thesis.ps\DTcomment{text práce ve formátu PS}.
	}
\end{figure}

\end{document}
