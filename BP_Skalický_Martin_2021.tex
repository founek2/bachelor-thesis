% arara: pdflatex
% arara: pdflatex
% arara: pdflatex

% options:
% thesis=B bachelor's thesis 
% thesis=M master's thesis
% czech thesis in Czech language 
% slovak thesis in Slovak language
% english thesis in English language
% hidelinks remove colour boxes around hyperlinks 
 
\documentclass[thesis=B,czech]{FITthesis}[2019/12/23]
  
\usepackage[utf8]{inputenc} % LaTeX source encoded as UTF-8
\usepackage{adjustbox}   

% \usepackage{amsmath} %advanced maths
% \usepackage{amssymb} %additional math symbols
 
\usepackage{dirtree} %directory tree visualisation
\usepackage{hyperref} 
   
% % list of acronyms
% \usepackage[acronym,nonumberlist,toc,numberedsection=autolabel]{glossaries}
% \iflanguage{czech}{\renewcommand*{\acronymname}{Seznam pou{\v z}it{\' y}ch zkratek}}{}
% \makeglossaries

\newcommand{\tg}{\mathop{\mathrm{tg}}} %cesky tangens
\newcommand{\cotg}{\mathop{\mathrm{cotg}}} %cesky cotangens
\newcommand\tstrut{\rule{0pt}{2.4ex}}
\newcommand\bstrut{\rule[-1.0ex]{0pt}{0pt}}

% % % % % % % % % % % % % % % % % % % % % % % % % % % % % % 
% ODTUD DAL VSE ZMENTE
% % % % % % % % % % % % % % % % % % % % % % % % % % % % % % 

\department{Katedra softwarového inženýrství}
\title{IoT platforma s webovým rozhraním}
\authorGN{Martin} %(křestní) jméno (jména) autora
\authorFN{Skalický} %příjmení autora
\authorWithDegrees{Martin Skalický} %jméno autora včetně současných akademických titulů
\author{Martin Skalický} %jméno autora bez akademických titulů
\supervisor{Ing. Jiří Mlejnek}
%\acknowledgements{Doplňte, máte-li komu a za co děkovat. V~opačném případě úplně odstraňte tento příkaz.}
\abstractCS{Ze slova IOT se v posledních letech stal buzzword, pod kterým si lze představit téměř cokoliv od chytré žárovky až po automatizovanou linku. IOT platforma označuje  platformu, ke které lze připojit chytré věci jako např. chytrou žárovku či teplotní senzor a následně data ze zařízení zobrazovat, analyzovat a vzdáleně ovládat mimo jiné ze svého telefonu. Tato práce se zabývá porovnáním aktuálních platforem na trhu pro domácnosti a následným návrhem vlastního řešení včetně implementace. Výsledná platforma bude primárně určena pro domácí kutily, kteří si vyrobý DIY zařízení. Uživatel bude s platformou interagovat pomocí Progresivní webové aplikace. Součástí práce je dokumentace pro připojení zařízení k platformě včetně několika celých řešení založených na chipu ESP8266.}
\abstractEN{Sem doplňte ekvivalent abstraktu Vaší práce v~angličtině.}
\placeForDeclarationOfAuthenticity{V~Praze}
\declarationOfAuthenticityOption{4} %volba Prohlášení (číslo 1-6)
\keywordsCS{ESP8266, PWA, MQTT}
\keywordsEN{Nahraďte seznamem klíčových slov v angličtině oddělených čárkou.}
% \website{http://site.example/thesis} %volitelná URL práce, objeví se v tiráži - úplně odstraňte, nemáte-li URL práce

\begin{document}

% \newacronym{CVUT}{{\v C}VUT}{{\v C}esk{\' e} vysok{\' e} u{\v c}en{\' i} technick{\' e} v Praze}
% \newacronym{FIT}{FIT}{Fakulta informa{\v c}n{\' i}ch technologi{\' i}}

\begin{introduction}
    %sdělit že téma je nové, aktuální, je ho potřeba řešit, komu to bude prospěšné, proč jsme si ho zvolili - vypadá dobře oborová motivace -> vyřešení daného problému někomu pomůže (ideálně nějaké komunitě), sdělit povrchově čím se zabývá práce ->rozvedení v kapitole Cíl, pak na konci úvodu stručné představení práce, aby si čtenář udělal představu jak budu postupovat

    Internet věcí je horkým tématem posledních několika let, ale jeho vývoji předcházela spousta trnitých cest a slepých uliček. Pod kouzelnou zkratkou IoT se pro mnohé skrývá příslib pokroku od chytré domácnosti až po revoluci v průmyslu. Internet věcí je označení pro síť fyzických zařízení, které dokáží spolu komunikovat ať už napřímo nebo pomocí prostředníka. Pro efektivní správu se zařízení připojují k centrální Platformě, která sbírá data z jednolivých zařízení a dle definovaných pravidel zařízením posílá příkazy. Například v domácnosti čidlo zaregistruje příchod majitele domů a Platforma v reakce zapne vytápějí. Současně poskytuje uživatelské rozhraní, pomocí kterého lze jednotlivá zařízení ovládat přímo a definovat scénáře, na základě kterých se vykonává automatizace jako uvedené automatické zapnutí vytápějí při příchodu. Platforma je tedy nedílnou součástí světa IoT a od jejích funkcí se odvíjí možnost využití plného potenciálu.

    Na trhu již dnes existují hotová řešení, ale je velmi problematické se mezi nimi zorientovat a často bývají velmi drahá. Důvod vzniku této práce pochází z osobní negativní zkušenosti s komerční platformou s cílem vytvoření dostupné otevřené Platformy pro technické entuziasty a bastlíře, kteří si chtějí jako já vytvářet levná zařízení a jednoduše je spravovat/ovládat.

    Teoretická část práce se věnuje analýze aktulních řešení na trhu se závěrečným porovnáním. Praktická část se zaměřuje na návrh, implementaci a nasazení vlastního řešení.

    %představit koncepci koncovích zařízení
    %využití - sledování vláhy, zalévání, udírna, ovládání hlasem

    % Takovýchto řešení již dnes existuje celá řada a podíváme se na výhody a nevýhody těch nejznámějších řešení jak z komerčního, tak i OpenSource světa.
    % Následně se budeme věnovat návrhu vlastního řešení s důrazem na bezpečnost, kterou stále spousta výrobců opomíjí. Moje řešení bude mít webové rozhraní, které umožní ovládání ze všech běžných zařízení bez nutnosti vytvářet nativní aplikace.


\end{introduction}

\chapter{Cíl práce}
% návrh vlastní platformy, cíleno pro kutily, bezpečnost pochybná u komerčních řešení -> já chci mít velkou, moje řešení bude více decentralizované
Cílem této práce je porovnat výhody a nevýhody aktuálních řešení jak z komerčního, tak i OpenSource světa a následně navrhnout a realizovat vlastní řešení. Tato Platforma bude koncipována primárně pro cloudové nasazení. Bude umožňovat registraci uživatelům, kteří si potom budou moci spravovat svá vlastní zařízení včetně oprávnění. Na rozdíl od komerčních řešení, která jsou založena čistě na centrální správně, tak tady bude umožněn jak centralizovaný přístup (přes platformu), tak i možnost decentralizovaného - zařízení se budou moci přihlásit k odběru události z ostatních zařízení a na základně toho reagovat. Otevře se tak možnost využít pouze nejzákladnější funkce z platformy, ale veškerou automatizaci a následné reakce implementovat pouze ve firmwaru jednotlivých zařízení. Tento přístup je pro implementaci složitější, ale zvyšuje následnou odolnost v případě výpadku Platformy.

Řešení bude cíleno primárně pro domáci kutily, kteří si budou moci připojit libovolné DIY (\uv{Udělěj si sám}) zařízení, nebo pro specializované nasazení, které vyžaduje vysokou spolehlivost.

Vzhledem k rozsahu Bakalářské práce není cílem nahradit existující řešení, která již obsahují velké množství funkcí, ale vytvořit možnou alternativu a demonstrovat složitost vytvoření celého řešení za použití moderních technologií.



\input{01analysis.tex}



\chapter{Analýza a návrh}
Tato kapitola se zabývá analýzou problémů v oblasti IOT Platformy a návrhem architektury řešení.

\section{Popis domény} % nativní podpora pro switch, sensor, generic - jednolivé popsat, abych se mohl na ně odkázat v UC
Tato kapitola obsahuje popis jednotlivých entit/pojmů se kterými Platforma pracuje.
\begin{itemize}
    \item \textbf{Uživatel} - fyzická osoba, která interaguje se systémem přímo pomocí pomocí uživatelského rozhraní nebo skrze připojená zařízení.
    \item \textbf{Zařízení (Device)} - fyzické zařízení, které komunikuje s Platformou, dále se dělí na Věci
    \item \textbf{Věc (Thing)} - logické uskupení Vlastností, např. meteostanice
    \item \textbf{Vlastnost (Property)} - určitá veličina, jejíž hodnota se odesílá na Platformu (např. teplota), která případně umožňuje být Platformou změněna (nastavena)
    \item \textbf{Umístění} - kde je dané zařízení umístěno, specifikující budovu a místnost, např. doma/kuchyň
    \item \textbf{Stav (State) Věci} - v jakém aktuálním stavu se určitá Věc nachází, skládá se ze stavů příslušných Vlastností
\end{itemize}


\section{Analýza případů užití}
Tato kapitola popisuje identifikované případy užití, ve kterých figurují následující aktéři:
\begin{itemize}
    \item User - aktér představující autentizovaného uživatele webového rozhraní.
    \item User ROLE\_ADMIN - aktér představující autentizovaného uživatele s právý kompletní správy systému.
    \item Zařízení - aktér představující koncové zařízení komunikující s Platformou
\end{itemize}

\subsection{UC1 - Registrace uživatele}
Neautentizovaný uživatel vyplní registrační formulář obsahující jméno, přijmení, uživatelské jméno, heslo a email. Validace dat bude prováděna dle [F8]. Při zpracování požadavku na serveru bude zajištěna unikátnost uživatelského jména a emailu napříč databází. Uživatel bude informován o úspěchu/neúspěchu akce. Po úspěšném vytvoření bude automaticky přihlášen, pokud nezrušil ve formuláři zaškrtávátku \uv{Automaticky přihlásit}, a bude mu odeslán uvítací email na zadanoou emailovou adresu.

\subsection{UC2 - Obnovení hesla}
- součástí přihlašovacího formuláře bude odkaz na stránku pro obnovení zapomenutého hesla, kde bude uživatel dotázán na emailovou adresu, kterou použil při registraci. Po zadání, pokud daná emailová adresa je součástí některého uživatelské účtu, bude na ni odeslán email s odkazem pro obnovu hesla. Na tomto odkazu bude uživatel vyzván k zadání nového hesla.

\subsection{UC3 - Objevení a spárování nového zařízení}
Zařízení pokud není ještě není spárované s Platformou, tak po zapnutí vyzve uživatele k zadání svéhu uživatelského jména na Platformě. Toto zadání bude umožněno vytvořením Wifi přístupového bodu, na kterém poběží kaptivní portál - po připojení telefonem/počitačem se automaticky zobrazí webová stránka s formulářem pro zadání údajů. Následně se zařízení připojí k Platformě, ohlásí jaké má Věci a popíše jejich Vlastnosti, a bude ji informovat, kterému uživateli (podle zadaného uživ. jména) má zobrazit možnost přidání nového zařízení. Pokud si uživatel dané zařízení přidá (viz. UC3), tak následně Platforma odešle zařízení API klíč, které si ho uloží a přihlásí se pomocí toho klíče k Platformě (nyní je spárované).

\subsection{UC4 Popis věcí}
Zařízení při ohlašování definice věcí a jejich vlastností musí oznámit mimo jiné typ věci. Platforma bude podporovat mimo generického (generic) typu další 3 typy věcí, pro které budou předdefinované vlastnosti a speciální zobrazení v uživatelském rozhraním:
\begin{itemize}
    \item Switch - přepínač s vlastností \uv{power}, která se nachází ve stavu on/off. V rozhraní bude Věc reprezentována dvou stavovým přepínačem, který při kliknutí odešle změnu stavu na druhý než je aktuálně.
    \item Activator - spínač s vlastností \uv{activate}, který má pouze jeden stav. V rozhraní bude Věc reprezentována tlačítkem, které na stisk odešlě aktivaci zařízení
    \item Sensor - senzor s jednou vlastnosti, jejíž definice záleží na zařízení. Jediné omezené je, že stav musí být číselná hodnota. V rozhraní bude zobrazena aktuální hodnota vlastnosti.
    \item Generic - obecný typ u kterého zařízení popíše strukturu a datové typy příslušných vlastností. V uživatelském rozhraní bude Věc reprezentována jako Widget, který po kliknutí zobrazí Dialogové okno umožňující zobrazení a ovládání všech vlastností dle konfigurace.
\end{itemize}


\subsection{UC5 - Přidání nového zařízení}
Uživatel ve správě zařízení bude mít sekci \uv{Přidat zařízení}, kde se zobrazí všechna nově detekovaná zařízení, která ještě nemá přidaná. Následně při kliknutí na tlačítko přidat se zobrazí jednoduchý formulář pro zadání umístění a názvu zařízení - bude předvyplněn název, který ohlásilo zařízení. Uživatel formulář potvrdí, systém následně vytvoří dané zařízení, přidá uživateli k němu oprávnění a ná stránce \uv{Ovládání} už bude uživatel moci sledovat aktuální stav Věcí a případně je i ovládat (pokud to umožňují).


\subsection{UC6 - Správa zařízení}
Uživatel na stránce \uv{Správa zařízení} bude tyto dvě sekce:
\begin{itemize}
    \item \uv{Přidat zařízení} - zde se zobrazí všechna nově detekovaná zařízení, která ještě nemá přidaná. Následně při kliknutí na tlačítko přidat se zobrazí jednoduchý formulář pro zadání umístění a názvu zařízení - bude předvyplněn název, který ohlásilo zařízení. Uživatel formulář potvrdí, systém následně vytvoří dané zařízení, přidá uživateli k němu oprávnění a ná stránce \uv{Ovládání} už bude uživatel moci sledovat aktuální stav Věcí a případně je i ovládat (pokud to umožňují).
    \item  \uv{Správa} - zde budou zobrazena všechna zařízení, ke kterým má oprávnění pro editaci. Rozhraní umožní pro každé zařízení jeho smazání a pomocí formuláře editaci názvu, umístění a změnu oprávnění pro jednotlivé uživatele. Tyto oprávnění budou rozděleny na tyto tři úrovně:
          \begin{itemize}
              \item \textbf{Read} - uživatel bude moci si pouze zobrazit veškeré údaje o zařízení
              \item \textbf{Control} - stejné oprávnění jako \uv{read} + navíc zařízení může ovládat
              \item \textbf{Write} - uživatel může editovat veškeré informace o zařízení (včetně oprávnění)
          \end{itemize}
\end{itemize}

\subsection{UC7 - Správa uživatelů}
Uživatel s rolí \uv{admin} může přistoupit na stránku \uv{Správa uživatelů}, kde se mu zobrazí seznam všech registrovaných uživatelů. Jednotlivé uživatele může smazat a pomocí formuláře editovat všechny jejich osobní údaje včetně hesla.

\subsection{UC8 - Notifikace}
Rozhraní umožní uživateli nastavit pravidla libovolnou Věc, při kterých se odešlě Web Push notifikace. Pro nastavení bude zobrazený formulář umožňující výběr z vlastností dané věci, po vybrání se zobrazí výběr akce, při jejímž splnění chce uživatel obdržet notifikaci (překročení hodnoty / vždy / hodnota bude rovna) a případné pole pro zadání limitní hodnoty. Dále půjde zobrazit rozšířené nastavení pro konkrétní notifikační pravidlo umožňující nastavení času a konkrétních dní v týdnu, kdy bude pravidlo platné (ve výchozím stavu bude platné pořád).

\subsection{UC9 - Validace}
Uživatel při vyplňování veškerých formulářů v uživatelském rozhraní obdrží interaktivní vazbu v případě zadání nevalidních údajů. Interaktivní vazbou jsou myšleny následující scénáře při průchodu formuláře:
\begin{itemize}
    \item Zadání nové hodnoty a vykliknutí z pole - bude provedena validace a v případě nevalidního vstupu, bude uživatel vizuálně  upozorněn.
    \item Editace již zadané hodnoty v poli - validace bude provedena po každé změně (stisknutí klávesy), uživatel bude opět vizuálně upozorněn v případě nevalidního vstupu.
\end{itemize}


\subsection{Nefunkční požadavky}

\paragraph{N1 Systém spustinelný na Linux systému}
- systém bude možno provozovat na Linuxovém serveru (Debian) a také na platformě Raspberry Pi (Raspbian).

\paragraph{N2 Responzivní webové rozhraní}
- aplikace bude nabízet responzivní uživatelské rozhraní přizpůsobené pro zobrazení na mobilních zařízeních i stolních počítačích. Uživatelské rozhraní bude kompatibilní s prohlížeči Mozilla Firefox verze 80, Chrome verze 80 a Safari na iOS. Dále bude implementovat tzv. PWA (Progresivní webová aplikace) - bude využívat cache pro statické soubory, pro rychlé načítání a na zařízení Android půjde v aplikaci chrome přidat na plochu a následně vypadat jako nativní aplikace.

\paragraph{N3 Rozhraní realizováno jako SPA}
- SPA (Single page application) je webová aplikace, která utilizuje JavaScript, aby při interakci v rámci aplikace se nemusela celá stránka načítat, ale pouze chytře překresluje potřebné části. Výsledkem je mnohem přijemnější uživatelský zážitek, než při nutnosti překleslovat celou stránku po kliknutí na odkaz.

\paragraph{N4 Koncová zařízení na platformě ESP8266}
- pro jednotlivá zařízení bude použit čip ESP8266 od firmy Espressif jako mikrokontroler, který bude komunikovat po sítí prostřednictvím Wifi sítě.

\paragraph{N5 Výkonnostní požadavky}
- systém bude stabilní a zvládne obsluhovat stovku zařízení, kde každé bude odesílat změnu stavu s periodicitou 30 vteřin. Při tomto dlouhodobém zatížení nebude docházet k pádům systému ani k výraznému zpoždění komunikace (RESTful požadavky pod 500 ms).

\paragraph{N6 Konfigurace systému} %env promněné
- veškerá konfigurace (týkající se externích služeb/komunikace) jako jméno a heslo do databáze, číslo portu pro komunikaci atd. bude konfigurovatelné pomocí promněných prostředí (env variables). Detailní popis promněných bude obsažen v instalační příručce.

\section{Vybrané technologie}

\subsection{Komunikační protokol}   %https://ieeexplore.ieee.org/abstract/document/8079928
Komunikační protokolů je velké množství a přímo závisí na výběru přenosného média. Při použití specializovaných sítí jako LoRa nebo Zigbee, nemáme moc velkou flexibilitu ve výběru. Zařízení podporující tyto specializované sítě jsou také poměrně drahá, ale umožňují běh na baterii. Zatímco využití WiFi sítě nám dává obrovskou flexibilitu ve výběru protokolů a zařízení podporující připojení k Wifi nebyli nikdo více cenově dostupné jako dnes. Primárně z finančních nároků a možnosti využití stávající WiFi infrastruktury v domácnosti jsem se rozhodl pro WiFi jako bezdrátové médium. Současně díky přímé podpoře IP protokolu nebudu muset vytvářet bridge mezi serverem a sítí se zařízenímí jako např. při použití Bluethooth či LoRa.

%https://www.educba.com/mqtt-vs-websocket/
Z protokolů pro komunikaci je dnes neojoblíběnější HTTP, který ale nativně nepodporuje obousměrnou komunikaci, kdy zařízení může poslat zprávu serveru a stejně tak server zprávu zařízení, kterou pro ovládání zařízení budu potřebovat. Z obousměrných protokolů se velmi osvědčil WebSocket, který lze velmi snadno kombinovat s HTTP. Jedná se o protokol postavený nad TCP, který naváže spojení a obě strany mohou posílat zprávy. Je to velice hezké řešení pro posílání zpráv, ale neumožňuje nativně systematickou filtraci nebo odběr pouze určitých zpráv a nepočítá s během na nespolhlivých zařízeních. Proto přímo pro IOT vznikl otevřený síťový protokol MQTT jehož specifikaci vydává OASIS[??]. Využívá asynchroní pattern publish-subsribe a byl speciálně navržen pro potřeby běhu na jednoduchých embeded zařízení s minimálním datovým tokem.

MQTT se vyvýjí již od roku 1999 a momentálně nejpoužívanější verzí je 3.1.1 pro kterou vznikla specifikace v roce 2014. Protokol primárně běží nad TCP/IP, ale lze využít v jakékoliv síti kde je zaručeno správné pořadí dat, beztrátovost a obousměrnost komunikace. Protokol definuje 2 typy entit: \uv{message broker} a klient. MQTT broker je server, který přijímá všechny zprávy od připojených klientů a přeposílá je správným příjemncům (klientům). MQTT klient je jakékoliv zařízení (od embeded až po server), které komunikuje s brokerem přes síť.

Posílaná data jsou zde hierarchicky rozdělena do tzv. topiců (česky témat). Topic je textový řetězec o maximální délce 65536 Bytů s oddělovačem "/" - ukázka "/house/bedroom/light". Pokud klient chce odeslat (publish) data, tak pošle zprávu brokeru s daty a topicem do kterého zpráva patří. Broker potom zprávu odešle všem klientům, kteří jsou přihlášení k odběru (subscrib) z daného topicu. O odběr se klient musí přihlásit a buď přímo specifikuje plný název topicu nebo částečný s použitím wildcard. MQTT počítá s případnou nespolehlivostí ať koncových zařízení nebo sítě a proto umožňuje klientovy při přihlášení definovat \uv{Last Will and Testament} (LWT), jedná se udání tématu a zprávy, do kterého se odešla v případě nesprávně odpojeného klienta. Takto lze notifikovat ostatní, že došlo je ztráně spojení s daným klientem.

Broker podporuje 3 třídy QoS (Quality of service), kterou lze specifikovat pro každou zprávu jednotlivě v závisloti na její důležitosti, jsou seřazeny vzestupně dle náročnosti na systém (overhead):
\begin{itemize}
    \item \textbf{0 - Maximálně jednou} - zpráva je odeslána pouze jednou a klient ani broker nijak napotvrzují její přjetí
    \item \textbf{1 - Alespoň jednou} - zpráva je odeslaná několikanásobně, dokud není potvrzené její přijetí
    \item \textbf{2 - Právě jednou} - odesílatel a příjemnce navazují dvoucestný hand-shake, aby bylo zaručeno přijmutí zprávy právě jednou
\end{itemize}
% TODO přeložit wildcard, overhead + pěkný obrázek client - broker - client pub/sub s komunikací

\subsection{Výhody jednotného jazyku}
Využití jednotného jazyka pro vývoj Backendu a Frontendu přináší obrovskou výhodu v podobě možnosti sdílet nejenom definice pro objekty, ale i přímo části kódu. Toto je velmi vhodné například pro jednotné validace formulářů, různé datové transformace a sdílení aplikační logiky pro frontend \uv{optimistické aktualizace} (aktuální trend, nečekat na potvrzení požadavku ze serveru, ale rozhraní aktualizivat, jako by požadavek byl úspěšný a pouze v případě neúspěchu zobrazit stav ze serveru). Dále jednotný jazyk umožňuje programátorům při vývoji v případě potřeby pohodlně pracovat na obou částech aplikace aniž by se museli učit nový jazyk.

\paragraph{TypeScript} je superset JavaScriptu, který navíc přidává komplexní typový systém a rozhodl jsem se ho využít jako hlavní programovací jazyk pro frontend i backend. Jedná se o OpenSource jazyk vyvýjený společností Microsoft, který jeho vznikem chtěl usnadnit přechod C\# a .NET vývojářum k webovým aplikacím. Mnoho lidí z JavaScript komunity považuje TypeScript jako kontroverzní počin, protože přidává složitost k velmi elegantnímu jazyku a zvyšuje časovou náročnost vývoje. Já jsem dlouho dobu tento názor také zastával, ale v posledních letech při práci na větších projektech a díky zkušeností z jiných jazyků (včetně striktně typových jako C++ a Java), jsem změnil svůj názor ve prospěch TypeScriptu. Souhlasím, že na první pohled prodlužuje dobu vývoje. Programátor musí psát věci navíc oproti čistému JavaScriptu, ale v dlouhodobém životním cyklu se tato práce \uv{na víc} mnohonásobně vrátí. A to v podobě statické kontroly typů, která minimalizuje riziko pádu aplikace a umožňuje  lepší statickou analýzu kódu, a dále jako největší přínost pro mne jako programátora TypeScript přináší funkční \uv{našeptávání} ve vývojovém prostředí, které pro JavaScriptu i přes veškeré snahy bohužel funguje ve velmi omezené míře.

\subsection{Backend}    %https://nodejs.org/en/docs/
%spousta jazyků, pro koncepsi asynchroních messages MQTT a Websocket se hodí NodeJS
Vzhledem k povaze MQTT, který je koncepčně založený na asynchroních zprávách jsem si zvolil technologii NodeJS, která je postavené na asynchroní event-driven architektuře. NodeJS je prostředí pro běh JavaScriptu na straně serveru, kterému se v posledních letech dostává velké pozornosti kvůli jeho oblíbenosti mezi vývojáři, je extrémně přívětivý k začínajícím programátorům a má pozvolnou křivkou učení. Díky své architektuře nabízí velice elegantní přístup pro zpracování akcí, kde se musí dlouho čekat na výsledek jako primárně u síťové komunikace. V tradičním jazyce jako Java se taková akce musí řešit pracným vyvtvořením nového vlákna, které čeká na výsledek a následným zpracováním. V NodeJS je programátor od této problematiky odstíněn a může se tak plně věnovat tvorbě aplikační logiky aniž by měl znalosti a zkušenosti s více vláknovým programováním.

%https://vegibit.com/what-is-nodejs/ 
%modeulscount.com - down currently 
\paragraph{NodeJS} má pravděpodobně aktuálně největší a nejaktivější komunitu ze všech jazyků. Pro správu knihoven používá balíčkovací systém npm (jsou i jiné alternativy) ze kterého se stal největší ekosystém na světě, který je zastřešený neziskovou společností "npm, Inc." provozující centrální repozitář se všemi dostupnými moduly pro NodeJS. Díky sve centralizaci je velmi jednoduchý na používání, ale v posledních letech kdy se NodeJS zpopularizoval a nyní se obecně JavaScript řadí mezí nejoblíbenější jazyky, se ukázala centralizace jako poměrně nešťastné řešení, kvůli vysokým nákladům na provoz infrastruktury. Pro představu velikosti ekosystému: npm v roce 2020 obsahoval 1 200 000 modulů a druhý největší systém RubyGems "pouhých" 350 000. Všechny moduly jsou k dispozici zcela zdarma a díky takto aktivní komunitě lidí, kteří dávají k dispozici své knihovny ostatním, je vysoce pravděpodobné, že pokud chceme řešit nějaký problém, tak na něj již existuje knihovna.

%https://expressjs.com/
\paragraph{ExpressJS} je velmi minimalistický webový framework (první verze v roce 2010), který je do dnes velmi oblíbený a v mnohém ovlivnil vývoj většiny frameworků. Jeho největší výhoda je vysoká flexibilita. Nabízí pouze základní definici způsobu pracování s HTTP požadavky a možnost registrovat tzv. middleware - software, který rozšiřuje funkcionalitu. Veškerá funkcionalita je dodávána pomocí middlewarů, které jsou k dispozici jako moduly. Vývojář si tedy může výsledný server poskládat přesně dle svých představ, kterých existují desítky vytvořených přímo od autorů a další stovky od komunity.

\paragraph{MongoDB} je OpenSource dokumentová NoSQL databáze umožňující horizontální škálování. Oproti klasické SQL databázi používá dynamické schéma, díky kterému lze aplikaci mnohem dynamičtěji vyvíjet na rozdíl od SQL databází, kde i malá změna tabulkového schéma v mnoha případech znamená velmi komplikovanou migraci dat. MongoDB používá pro uchovávání dokumentů podobný JSONu (MongoDB nazývá formát BSON), což se velmi snadno kombinuje s jazykem JavaScript, který JSON používá pro nativní objekty.

\paragraph{Mongoose} je knihovna pro NodeJS, která vytváří nad MongoDB objektovou abstrakci a spoustu dalších užitečných funkcí jako validace a type cast. Základním prvkem je definování schéma pro jednotlivé dokumenty, což může vypadat jako návrat do striktního schématu u SQL databází, ale zde je schéma definované pouze na úrovni Mongoose, tedy mnohem flexibilnější a méně striktní. MongoDB sice nabízí oficiální knihovnu pro přístup do databáze, ale preferuji Mongoose, protože díky definici schémat mám obecně větší kontrolu nad daty, která se dostanou do databáze.

\paragraph{AgendaJS} je knihovna na perzistentní plánování úkolů (jobs) pro NodeJS. Podporuje zpracování/plánování/perzistenci úkolů a jejich opětovné zpracování v případě chyby. Tato knihovna bude primárně využita pro zajištění odeslání emailů a pro spouštění případných plánovaných akcí. Proč v souvislosti s odesláním emailů? Jejich zpracování je závislé na třetí straně - emailovém serveru, který nemusí být vždy dostupný. Pokud systém bude mět odeslat email, tak tímto způsobem bude zajištěno, že i v případě selhání bude email opětovně odeslán jakmile bude možné.

\paragraph{Socket.IO} umožňuje navázíní obousměrného spojení s kompatibilním klientem. Je založený na vlastním protokolu využívající HTTP spojení nebo WebSocket a vyznačuje se vysokou spolehlivostí a rychlostí umožňující real-time komunikaci. Jedná se o velice populární a časem ověřené řešení, které zajišťuje kompatibilitu i s prohlížeči nepodporující technologii WebSocket.


\subsection{Frontend}
% React + Redux
Prvním světoznámím průkopníkem ve světě JavaScriptu pro tvorbu uživatelského rozhraní byla knihovna jQuery, která existuje do dnes, ale spíše se již považuje za přežitek doby. Dnes máme velké množství Frameworků a knihoven pro tvorbu frontendu, ať pro tvorbu na straně serveru nebo přímo na straně uživatele v JavaScriptu. Trend dnešní doby je přesouvat generování rozhraní na stranu uživatele, jak kvůli snížení výkonostních nároků na server, tak spíše kvůli lepší odezvě a uživatelskému zážitků. Mezi nejznámější JavaScriptové frameworky patří bezpochyby Angular, Vue.js, Svelte a nesmíme zapomenout na React, který je sice knihovna, ale řadí se na stejnou úroveň. Já jsem si zvolil jako hlavní prostředek React, právě proto že se jedná o knihovnu. Framework se vyznačuje tím, že vynucuje určité problémy řešit jistým způsobem bez možnosti volby. Má to své výhody a nevýhody a do větších týmu bych rozhodně volil raději Framework. Tento projekt ale budu vytvářet primárně sám a mám velice rád flexibilitu a možnost volby. V začátcích to bývá časově náročnější, ale vidím v tom obrovskou možnost osobního růstu, protože při každé volbě musím hodnoti výhody/nevýhody a nakonec retrospektivně vidím následky svých rozhodnutí. Mimo to je React vytvářet společností Facebook a používán největšími technologickými světa, takže je jistá jeho dlouhodobá podpora a od roku 2013, kdy byla vydána první verze je hojně odzkoušený a ověřený.

%https://reactjs.org/tutorial/tutorial.html#what-is-react
\paragraph{React} je deklarativní, efektivní a flexibilní JavaScriptová knihovna pro tvorbu uživatelského rozhraní. Kód dělí do malých ižolovaných částí nazvaných "komponenta", které se skládání do sebe a mohou tvořit komplexní uživatelská rozhraní. Pro vysoký výkon využívá techniku virtuálního DOM - nejprve si vytvoří virtuální strom podoby rozhraní v paměti, který následně porovná s aktuální podobou vykreslenou v prohlížeči a zmanipuluje pouze ty části které se od posledního vykreslení změnily. Díky tomu je velice efektivní a dokáže vykreslovat komplexní stránky s obrovským množstvím dat.

%https://redux.js.org/introduction/getting-started
\paragraph{Redux} je knihovna pro state management. Ve větších aplikací je potřeba mít úložiště pro tzv. state (reprezentace momentálního stavu aplikace). Čím větší aplikace je, tím více dat je potřeba uchovávat a proto je nutno využít nějaké systematické řešení. React dlouho žádné takové nenabízel a bylo nutno využívat jiné knihovny jako např. Flux nebo Redux. V nejnovějších verzí již takové řešení nabízí, ale já přeferuji mít větší izolaci mezi vizuálnímí komponenty a statem, navíc Redux je odzkoušený časem a nabízí mnohem komplexnější řešení.

\section{Návrh uživatelského rozhraní}


\chapter{Realizace}

\section{Backend}
% Popsat rozdělení na 2 BE servery, API rozhraní, komunikace webSocket -> ukázat sexy diagram, Agenda (recoverable jobs)

\subsection{Bezpečnost}
% JWT, helmet, NoSql injection, HTTPS (nginx proxy), rabbitMQ jako broker + custom authBackend

\subsection{Validace}
% Vlastní framework, interaktivní validac na FE -> best UX, descriptory

\subsection{MQTT schéma}
% popis Homie komunikace + vlastní modifikace, ukázka na diagramu

\section{Frontend}
% Popis PWA, SPA, struktura složek, webSocket, no cookie (local storage)

\section{Knihovna pro ESP8266}
% Platformio, C++, pubsubclient, OTA

\section{chytrá udírna}

\subsection{Návrh zapojení}

\subsection{Výroba}

\input{04testing.tex}



\begin{conclusion}
    %sem napište závěr Vaší práce
\end{conclusion}

\bibliographystyle{csn690}
\bibliography{mybibliographyfile}

\appendix

\chapter{Seznam použitých zkratek}
% \printglossaries
\begin{description}
    \item[GUI] Graphical user interface
    \item[XML] Extensible markup language
\end{description}


% % % % % % % % % % % % % % % % % % % % % % % % % % % % 
% % Tuto kapitolu z výsledné práce ODSTRAŇTE.
% % % % % % % % % % % % % % % % % % % % % % % % % % % % 
% 
% \chapter{Návod k~použití této šablony}
% 
% Tento dokument slouží jako základ pro napsání závěrečné práce na Fakultě informačních technologií ČVUT v~Praze.
% 
% \section{Výběr základu}
% 
% Vyberte si šablonu podle druhu práce (bakalářská, diplomová), jazyka (čeština, angličtina) a kódování (ASCII, \mbox{UTF-8}, \mbox{ISO-8859-2} neboli latin2 a nebo \mbox{Windows-1250}). 
% 
% V~české variantě naleznete šablony v~souborech pojmenovaných ve formátu práce\_kódování.tex. Typ může být:
% \begin{description}
% 	\item[BP] bakalářská práce,
% 	\item[DP] diplomová (magisterská) práce.
% \end{description}
% Kódování, ve kterém chcete psát, může být:
% \begin{description}
% 	\item[UTF-8] kódování Unicode,
% 	\item[ISO-8859-2] latin2,
% 	\item[Windows-1250] znaková sada 1250 Windows.
% \end{description}
% V~případě nejistoty ohledně kódování doporučujeme následující postup:
% \begin{enumerate}
% 	\item Otevřete šablony pro kódování UTF-8 v~editoru prostého textu, který chcete pro psaní práce použít -- pokud můžete texty s~diakritikou normálně přečíst, použijte tuto šablonu.
% 	\item V~opačném případě postupujte dále podle toho, jaký operační systém používáte:
% 	\begin{itemize}
% 		\item v~případě Windows použijte šablonu pro kódování \mbox{Windows-1250},
% 		\item jinak zkuste použít šablonu pro kódování \mbox{ISO-8859-2}.
% 	\end{itemize}
% \end{enumerate}
% 
% 
% V~anglické variantě jsou šablony pojmenované podle typu práce, možnosti jsou:
% \begin{description}
% 	\item[bachelors] bakalářská práce,
% 	\item[masters] diplomová (magisterská) práce.
% \end{description}
% 
% \section{Použití šablony}
% 
% Šablona je určena pro zpracování systémem \LaTeXe{}. Text je možné psát v~textovém editoru jako prostý text, lze však také využít specializovaný editor pro \LaTeX{}, např. Kile.
% 
% Pro získání tisknutelného výstupu z~takto vytvořeného souboru použijte příkaz \verb|pdflatex|, kterému předáte cestu k~souboru jako parametr. Vhodný editor pro \LaTeX{} toto udělá za Vás. \verb|pdfcslatex| ani \verb|cslatex| \emph{nebudou} s~těmito šablonami fungovat.
% 
% Více informací o~použití systému \LaTeX{} najdete např. v~\cite{wikilatex}.
% 
% \subsection{Typografie}
% 
% Při psaní dodržujte typografické konvence zvoleného jazyka. České \uv{uvozovky} zapisujte použitím příkazu \verb|\uv|, kterému v~parametru předáte text, jenž má být v~uvozovkách. Anglické otevírací uvozovky se v~\LaTeX{}u zadávají jako dva zpětné apostrofy, uzavírací uvozovky jako dva apostrofy. Často chybně uváděný symbol "{} (palce) nemá s~uvozovkami nic společného.
% 
% Dále je třeba zabránit zalomení řádky mezi některými slovy, v~češtině např. za jednopísmennými předložkami a spojkami (vyjma \uv{a}). To docílíte vložením pružné nezalomitelné mezery -- znakem \texttt{\textasciitilde}. V~tomto případě to není třeba dělat ručně, lze použít program \verb|vlna|.
% 
% Více o~typografii viz \cite{kobltypo}.
% 
% \subsection{Obrázky}
% 
% Pro umožnění vkládání obrázků je vhodné použít balíček \verb|graphicx|, samotné vložení se provede příkazem \verb|\includegraphics|. Takto je možné vkládat obrázky ve formátu PDF, PNG a JPEG jestliže používáte pdf\LaTeX{} nebo ve formátu EPS jestliže používáte \LaTeX{}. Doporučujeme preferovat vektorové obrázky před rastrovými (vyjma fotografií).
% 
% \subsubsection{Získání vhodného formátu}
% 
% Pro získání vektorových formátů PDF nebo EPS z~jiných lze použít některý z~vektorových grafických editorů. Pro převod rastrového obrázku na vektorový lze použít rasterizaci, kterou mnohé editory zvládají (např. Inkscape). Pro konverze lze použít též nástroje pro dávkové zpracování běžně dodávané s~\LaTeX{}em, např. \verb|epstopdf|.
% 
% \subsubsection{Plovoucí prostředí}
% 
% Příkazem \verb|\includegraphics| lze obrázky vkládat přímo, doporučujeme však použít plovoucí prostředí, konkrétně \verb|figure|. Například obrázek \ref{fig:float} byl vložen tímto způsobem. Vůbec přitom nevadí, když je obrázek umístěn jinde, než bylo původně zamýšleno -- je tomu tak hlavně kvůli dodržení typografických konvencí. Namísto vynucování konkrétní pozice obrázku doporučujeme používat odkazování z~textu (dvojice příkazů \verb|\label| a \verb|\ref|).
% 
% \begin{figure}\centering
% 	\includegraphics[width=0.5\textwidth, angle=30]{cvut-logo-bw}
% 	\caption[Příklad obrázku]{Ukázkový obrázek v~plovoucím prostředí}\label{fig:float}
% \end{figure}
% 
% \subsubsection{Verze obrázků}
% 
% % Gnuplot BW i barevně
% Může se hodit mít více verzí stejného obrázku, např. pro barevný či černobílý tisk a nebo pro prezentaci. S~pomocí některých nástrojů na generování grafiky je to snadné.
% 
% Máte-li například graf vytvořený v programu Gnuplot, můžete jeho černobílou variantu (viz obr. \ref{fig:gnuplot-bw}) vytvořit parametrem \verb|monochrome dashed| příkazu \verb|set term|. Barevnou variantu (viz obr. \ref{fig:gnuplot-col}) vhodnou na prezentace lze vytvořit parametrem \verb|colour solid|.
% 
% \begin{figure}\centering
% 	\includegraphics{gnuplot-bw}
% 	\caption{Černobílá varianta obrázku generovaného programem Gnuplot}\label{fig:gnuplot-bw}
% \end{figure}
% 
% \begin{figure}\centering
% 	\includegraphics{gnuplot-col}
% 	\caption{Barevná varianta obrázku generovaného programem Gnuplot}\label{fig:gnuplot-col}
% \end{figure}
% 
% 
% \subsection{Tabulky}
% 
% Tabulky lze zadávat různě, např. v~prostředí \verb|tabular|, avšak pro jejich vkládání platí to samé, co pro obrázky -- použijte plovoucí prostředí, v~tomto případě \verb|table|. Například tabulka \ref{tab:matematika} byla vložena tímto způsobem.
% 
% \begin{table}\centering
% 	\caption[Příklad tabulky]{Zadávání matematiky}\label{tab:matematika}
% 	\begin{tabular}{|l|l|c|c|}\hline
% 		Typ		& Prostředí		& \LaTeX{}ovská zkratka	& \TeX{}ovská zkratka	\tabularnewline \hline \hline
% 		Text		& \verb|math|		& \verb|\(...\)|	& \verb|$...$|		\tabularnewline \hline
% 		Displayed	& \verb|displaymath|	& \verb|\[...\]|	& \verb|$$...$$|	\tabularnewline \hline
% 	\end{tabular}
% \end{table}
% 
% % % % % % % % % % % % % % % % % % % % % % % % % % % % 

\chapter{Obsah přiloženého CD}

%upravte podle skutecnosti

\begin{figure}
    \dirtree{%
        .1 readme.txt\DTcomment{stručný popis obsahu CD}.
        .1 exe\DTcomment{adresář se spustitelnou formou implementace}.
        .1 src.
        .2 impl\DTcomment{zdrojové kódy implementace}.
        .2 thesis\DTcomment{zdrojová forma práce ve formátu \LaTeX{}}.
        .1 text\DTcomment{text práce}.
        .2 thesis.pdf\DTcomment{text práce ve formátu PDF}.
        .2 thesis.ps\DTcomment{text práce ve formátu PS}.
    }
\end{figure}

\end{document}
