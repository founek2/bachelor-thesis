\chapter{Testování}
Tato kapitola se zabývá testováním implemetovaného řešení nasazeného v produkčním prostředí. V první fázi je otestována stabilita řešení při velké zátěži a ve druhé uživatelská přívětivost procesu přidání zařízení a samotného webového rozhraní. Produkční prostředí běží na virtuálním serveru se systémem Debian (linux), přidělenými hardwarovými prostředky: CPU (5 vláken Ryzen 3600), 9~GB RAM, HDD (7200 otáček). Použitá verze NodeJS 12.22.1, MongoDB 4.2.13 a RabbitMQ 3.8.14.

\section{Zátěžové testování}
Simulovat zátěž reálnými zařízeními lze pouze v jednotkách kusů, což s dnešním výkoným hardwarem není absolutně žádný problém obsloužit. Protože potřebuji ověřit jak se systém bude chovat pod opravdovou zátěží - desítky paralelních požadavků a stovky připojených zařízení - napsal jsem automatizovaný test, kterému pouze řeknu počet uživatelů a počet zařízení, které mezi uživatele má rozprostřít a o vše ostatní se test postará sám (pomocí HTTP požadavků stejně jako by uživatel interagoval přes webové rozhraní).

Nejprve test vytvoří příslušný počet uživatelů, následně virtuální zařízení (komunikující přes MQTT protokol stejně jako reálná), pro jednotlivé uživatele získá objevená zařízení, všechna je uživatelům přidá a následně zařízení žačnou odesílát data ze senzorů a jiných vlastností simulující reálný provoz - data jsou odesíláná v náhodném rozmezí 2 - 10 vteřin, aby se provoz rozložil v čase. Test probíhal v následující konfiguraci:
\begin{itemize}
    \item Počet uživatelů 15
    \item Počet zařízení 200
    \item Konfigurace zařízení - 2 senzory (teplota, vlhkost) a přepínač, data ze všech tří vlastností jsou odesíláná paralelně
    \item počet paralelních požadavků 10
\end{itemize}
Pro sledování prostředků byl využit \uv{Free monitoring od MongoDB}, který vykresluje do grafu zátěž cpu, dobu zpracování DB dotazů, zátěž disků a dalších metriky s periodou 1s.

% popsat které části jsou náročné na DB maybe? Kolik DB query bylo během testu, čím jsem monitoroval prostředky, query max time, cpu usage

% \section{Zotavení po nenadálé události}

\section{Uživatelské testování}