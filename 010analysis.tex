\chapter{Analýza}
Tato kapitola se zabývá definici IOT Platformy, analýzy již existujících řešení a jejich vzájemným porovnání. V závěru kapitoly následně návrhem vlastního řešení.


\section{Definice IOT Platformy}
"IOT Platforma je více vrstvá technologie, která umožňuje přímočaré zajištění, ovládání a automatizaci připojených zařízení ve světě internetu věci. Zjednodušeně propojuje Váš hardware, jakkoli rozdílný, do cloudu s možností různorodé konektivity, obsahuje bezpečnostní mechanizmy a široké možnosti pro zpracování dat. Pro vývojáře, IOT Platforma nabízí soubor předpřipravených funkcí, které vysoce zvyšují rychlost vývoje aplikací pro připojená zařízení a řeší škálování a kompatibilitu napříč zařízeními" \cite[překlad autora]{kaaproject}

\subsection{Definice pojmů}
V této sekci jsou vysvětleny pojmy, které budou použity v následujících kapitolách.

\begin{itemize}
    \item \textbf{Platforma} - Platformou se rozumý programové řešení umožňující propojení různorodých zařízení a jejich následnou obsluhu
    \item \textbf{Koncové zařízení} - Zařízení, které dokáže komunikovat po sítí a nabízí nějakou funkcionality (např. měření teploty nebo ovládání světla)
    \item \textbf{Bridge} - Bridge se rozumý síťové zařízení, které funguje jako prostředník mezi koncovými zařízeními a jinou sítí. Agreguje jednotlivá zařízení a nabízí rozhraní pro komunikaci s nimi.
\end{itemize}


\section{Vlastnosti}

\subsection{Komunikace}    % rozdíly v komunikačním mediu wifi, bluethoot, mesh řešení Z-wave a zigbee, LoRa, spotřeba, bateriový provoz, složitost instalace
Pro komunikaci mezi zařízeními se nejčastěmi používá bezdrátová komunikace, kvůli jednoduchosti instalace bez nutnosti většího zásahu do stávající infrastruktury. Tento typ lze rozdělit do dvou kategorií:
\begin{itemize}
    \item \textbf{Centralizované} - Každé zařízení komunikuje pouze s jedním centrálním prvkem, přes který jde veškerá komunikace. Mezi nejznámější technologii tohoto typu patří Wifi.
    \item \textbf{Decentrlizovanou} - V této síti komunikují zařízení přímo s ostatními bez jakéholiv prostředníka. Pokud zařízení nemohou komunikovat napřímo, tak využívají ostatní pro předání zprávy. Síť je díky tomu mnohem odolnější vůči výpadků, protože zde není tzv. \uv{Jediný bod selhání} (Single point of failure). Zpravidla mívá nižší datovou propustnost a je složitější pro nasazení a následnou správu. Velkou výhodou je snadnější rozšiřitelnost pokrytí, protože každé přidané zařízení rozšiřuje signál a tímto způsobem lze zařízení řetězit. Pro podrobější popis doporučuji \cite{mesh}.
\end{itemize}
Vzhledem k rozšířenosti Wifi, kterou dnes najdeme v každé domácnosti, se přirozeně nabízí její využití i pro internet věcí. A k tomu v posledních letech opravdu došlo. Díky extrémně levnému chipu ESP8266, který se dnes i u nás v ČR dá koupit za 70 Kč \cite{hadex}, došlo k masivní penetraci trhu s chytrými zařízeními využívající právě Wifi. Bohužel tato technologie má i svá negativa, největšími jsou spotřeba elektrické energie a limit maximálního počtu připojených zařízení na jeden centrální prvek (řádově desítky). Vysoká spotřeba je dána nutností časté komunikace jen kvůli udržení aktivního spojení a proto je možné provozovat zařízení na baterie pouze v jednotkách dní, maximálně týdnů.

Pro bateriový provoz vznikly speciální sítě, které sice na rozdíl od Wifi umožní přenos v desítka kp za sekundu (tisícina rychlosti běžné Wifi), ale jsou energeticky mnohem úspornější \cite{wifi-vs-ble} (umožnují provoz až desítky let na malou baterii), mají mnohonásobně větší dosah a umožňují propojení mnohem většího počtu zařízení (stovky).

Poměrně rozšířenými z centrálně orientovaných sítí jsou u nás SigFox a LoRa. SigFox je komerční řešení, kde se platí za každé připojené zařízení \cite{sigfox-price}. Oproti tomu síť LoRa používá otevřený standard pro komunikaci LoRaWAN \cite{lora}. Protože se jedná o otevřený standard, tak kdokoliv může vytvořit a provozovat kompatibilní zařízení \cite{lora}. Samozřejmně také lze využití komerční infrastrukturu, kam lze připojit svá zařízení za poplatek, tuto službu nabízí např. České radiokomunikace \cite{cra}, ale díky otevřenosti má každý možnost si za pár tisíc postavit vlastní GateWay (centrální prvek) a provozovat libovolná zařízení bez jakýhkoliv poplatků a prostředníků.

%https://thesmartcave.com/z-wave-vs-zigbee-home-automation/
Z decentralizovaných sítí se velmi rozšířily Zigbee a Z-Wave. Zigbee je otevřený standard, který dokáže pracovat, jak v pásmu 2.4GHz, tak i 900 MHz \cite{zigbee}. Nemá omezení na maximální počet zařízení zřetězených za sebou a dokáže vytvořit síť skládající se až ze 65 tisíc zařízení \cite{zigbee}. Z-Wawe je naopak uzavřený standard, který funguje pouze v pásmu 800-900 MHz \cite{zwawe}. Limituje maximální počet přeposlání zprávy na 4 a podporuje síť o velikosti až 256 zařízení \cite{zwawe}. Obě sítě jsou energeticky velmi úsporné a umožňují běh zařízení na obyčejnou knoflíkovou baterii po dobu až několika let\cite{zigbee, zwawe}.

\subsection{Automatizace}
%https://www.iot-now.com/2020/06/10/98753-iot-home-automation-future-holds/
Automatizace je ve světě IoT pravděpodobně nejdůležitějším tématem a každá IOT Platforma by ji měla umožňovat, protože dává možnost využít zařízení úplně novým způsobem \cite{what-future-hold}. Principiálně se jedná o možnost definování reakcí na jednotlivé události. Událost můžemý být změna teploty, otevření okna nebo detekce pohybu a reakce změna stavu zařízení - zhasnutí světla nebo zapnutí televize \cite{what-future-hold}. V podstatě jediným limitem je zde lidská představivost. Modelový scénář:

Představme si moderní dům, ve kterém jsou všechny věci, které nás napadnou, chytré, což s dnešními technologickými možnostmi není sci-fi, ale naopak možná realita. Majitel domu, říkejme mu Joe, přichází večer unavený domů a odemyká dveře. Vejde do vnitř a světlo na chodbě a v kuchyni již svítí. Jde přímo do kuchyně, protože po dlouhém dni v práci má hlad a usedá s jídlem ke stolu. Nemá rád ticho, tak řekne: ,,Alexo, zapni hudbu" a ze Sterea se spustí Beethoven, protože Alexa ví, že je to Joeův oblíbený skladatel klasické hudby. Joe cítí jak se po místnosti rozprostřívá příjemné teplo ze zapnuté klimatizace. Po příjemné večeři odchází do druhého patra do koupelny, samozřejmně se nemusí starat o zapnuté Stereo ani světla, protože se vše samo vypne, jakmile odejde. Ve sprše pustí vodu, která má automaticky teplotu nastavenou specificky dle Joeovi preference 36 °C i přes to, že 20 min předním se sprchovala jeho přítelkyně, která si libuje v teplejší vodě. Po sprše jde do ložnice a ulehá do postele zatímco se kontroluje, jestli jsou všechny dveře zamčené, okna zavřená a zapíná se alarm pro případný pohyb ve spodním patře. A jak mohlo být vše uzpůsobené Joeovím preferencím a vše zapnuté ještě před jeho vstupem do domu? Protože zvonek u dveří má kameru s rozpoznáváním obličeje - Joea tedy poznal a vše nastavil.

Takto tedy může vypadat automatizace v domácnosti, která zpříjemní život a odprostí Vás od spousty všedních věcí. Vše nastavené dle osobních preferencí a to nejen určité rodiny ale na úrovni jednotlivců v domácnosti.


\subsection{Bezpečnost a soukromí}
Při výběru Platformy by důležitým kritériem měla být bezpečnost. Na první pohled se to nemusí zdát být důležité, co se může stát když bude s platformou komunikovat čidlo pohybu a někdo se dokáže dostat k těmto údajům? Například pro zloděje mohou být taková data zlatý důl, protože bude přesně vědět kdy je dům prázdný.

Bezpečnost je potřeba zde sledovat hned na několika faktorech. Prvním je komunikační médium. Pokud zařízení komunikují bezdrátově, tak by komunikace měla být šifrovaná, aby se nedala jednoduše odposlechnout. Druhým faktorem je bezpečnost samotné platformy. Pokud bude platforma dostupná pouze na interní síťi v domácnosti, tak bezpečnost na první pohled ohrožená není, pokud se ale zamyslíme nad tím, kolik dnes doma máme chytrých zařízení, tedy takových, která dokáží komunikovat přes internet, tak zjistíme že jich je velké množství, protože dnes už např. chytrou televizi má doma téměř každý a je otázka na kolik věříme výrobcům těchto zařízení, že kladou důraz na jejich bezpečnost. Stačí aby nějaký vir napadl naší televizi či jiné zařízení a případný útočník má plný přístup k platformě pouze získáním přístupu do interní sítě. Proto by platforma měla využívat alespoň systém pro identifikaci, ideálně i autentifikaci a to nejen v případě, že je přístupná z internetu ale i z vnitřní sítě.

\subsection{Cílová skupina}
Internet věcí lze využít napříč všemi sférami. Od jednoduché meteostanice, která bude měřit venku teplotu, přes tvz. chytrou domácnost, kde Vám lednička pošle nákupní seznam na email podle chybějících potravin, přes využití v průmyslu pro sběr různorodých dat a jejich následnou analýzu ať pro zvýšení kvality nebo detekci poruchy, ještě před tím než k ní dojde. Tato práce cílí na využití IoT v běžné domácnost a implementací Platformy určené pro kutily a technické entusiasty, kteří chtějí mít svá data pod kontrolou, vytvářejí si různorodá zařízení a hledají Platformu s důrazem na bezpečnost a flexibilitu.


\section{Existující řešení}
Tato kapitola se zabývá pohledem na aktuálnní řešení jak Komerční, tak i OpenSource. Poukazuje na výhody a nevýhody z obou světů, následně se zaměřuje na analýzu konkrétních Platforem a jejich porovnáním.
\subsection{Komerční řešení} % hotové řešení, cloud, ale drahé
%easy to use, but paid
Na trhu dnes existuje velké množství komerčních řešení od známých výrobců. Někteří jsou známí spíše výrobou harwaru jako Philips a Xiaomi, jiní se zaměřují spíše na nabídku služeb a integraci zařízení ostatních výrobců pod svojí Platformu jako Amazon nebo Google. Pro koncového zákazníka mají Komerční řešení obrovskou výhodu v jednoduchosti nasazení a následné obsluhy. Stačí zakoupit centrální jednotku, libovolná zařízení od stejného výrobce a vše krásně funguje. Avšak problém nastává ve chvíli, kdy potřebují řešení škálovat či customizovat dle svých potřeb, protože si dodavatel za úpravy na \uv{míru} začne účtovat obrovské částky a zákazníkovi nezbývá nic jiného než platit. Sám si potřebné změny udělat nemůže, protože nemá zdrojové kódy a migrace k jinému produktu by znamenal obrovské náklady a problémy se stávájícími integracemi, protože různá řešení mívají různá rozhraní.

%security
Aspekt bezpečnosti u uzavřených řešení bývá diskutabilní. Pravidelné bezpečnostní audity kvůli vysokým nákladům provádí málo kdo. Výrobci samozřejmně vždy tvrdí, že bezpečnost je u nich na prvním místě, ale bohužel tento aspekt je v přímém kontrastu s jenododuchostí použití, což je pro výrobce mnohem důležitější protože pokud se řešení dobře a jednoduše ovládá, tak mnohem spíše si ho zákazníci oblíbí, než pokud bude maximálně zabezpečeno, ale uživatel bude muset provádět úkony návíc čistě kvůli bezpečnosti, která mu na první pohled nepřínáší přidanou hodnotu.

%Cloud dependent
Od Platformy očekáváme možnost vzdáleného ovládání, tedy přístup odkudkoli z internetu. Málokdo má však doma veřejnou IP adresu, aby si mohl celé řešení provozovat doma \uv{self-hosted}. V Praxi si tedy uživatel pořidí domů Bridge, který komunikuje s chytrými zařízeními v domácnosti a současně s cloudem výrobce, přes který lze přistupovat na Platformu a ovládat všechny zařízení. Takové řešení se velmi osvědčilo díky jednoduchosti, protože neklade žádné nároky na uživatele jako např. veřejnou IP adresu. Problém však může nastat ve chvíli, kdy výrobce daného řešení po několika letech ukončí činnost a s tím přestane provozovat svojí cloudovou infrastrukturu, na které je závislý Bridge a vzdálený přístup z internetu. V lepším případě bude zachována funkčnost v lokální sití, v horším přestane řešení fungovat úplně. Najednou uživatelovi zbyde doma spousta funkčního (po fyzické stránce) harwaru, který nemůže využívat.

Výše jsem nastínil nejhorší možný scénář, který naštěstí v poslední době již přestává platit, protože výrobci společně vytvářejí otevřené standardy pro komunikaci, které by měli zaručit kompatibilitu zařízení napříč jednotlivými výrobci. Bohužel standardů vzniká současně více a ne všichni je plně dodržují, takže nekompatibilita ještě bude delší dobu přetrvávat i když ne v takovém měřítku jako před pár lety. Kromně rozdílných protokolů je také nekompatibilita v různých technologiích přenosu mezi nejznámější patří WiFi, Bluethooth, LoRa, Zigbee a Sigfox.
%podpora jiných výrobců? integrace? -> závislé na tom co výrobce se rozhodne implementovat

\subsection{OpenSource řešení}
% nepopulární/špatná reputace mezi lidmy, často potřeba znalosti problematiky, Free, flexibilní, customizovatelné
OpenSource řešení mají mezi širší veřejností špatnou reputaci, protože na rozdíl od komerčních \uv{Plug\&Play} produktů většinou vyžadují určité povědomí o dané problematice. Je to způsobeno tím, že se snaží pokrýt celou doménu stejně jako komerční řešení, ale oproti nim se zlomkem vývojářů a financí. Následkem toho není prvotní nastavení pro laika zcela přímočaré a může se střetnout s problémy. Avšak překonání prvotních nesnázích přináší následně spoustu pozitiv.

Jedním z nejatraktivnějších lákadel je samozřejmně cena. OpenSource řešení jsou zpravidla zcela zdarma, případně nabízejí placenou podporu. Mě osobně na OpenSource nejvíce zaujala komunita. Pokud se projekt dostane do určité známosti, tak kolem něho začně vznikat komunita lidí, primárně technologických nadšenců ale i lidí z IT praxe, kteří mezi sebou komunikují a spolupracují na vylepšení daného řešení, ať už přímo (napsání části funkcionality) nebo nepřímo (komunikace s vývojáři). Potom i obyčejný uživatel, který chce řešení využít, tak při objevení potíží, může požádat komunitu o pomoc a protože to jsou nadšení lidé, jsou velmi ochotní.

Pokud máme dostatečné technické znalosti, tak si můžeme prohlédlou přímo zdrojové kódy a sami si zhodnotit kvalitu i bezpečnost. U větších projektů to však již není tak úplně možné při desítkách tisíc rádků kódu, ale existují lidé, kteří tomu opravdu věnují čas a mohou tak objevit zranitelnosti. Dále OpenSource projekty bývají mnohem více sdílné ohledně architektury kterou využívají a je možno se v dokumentaci dočíst, jak vlastně řešení funguje interně, na rozdíl od komerčních, kde je to tzv. \uv{BlackBox}.

OpenSource platformy bývají postavené na systému Pluginů, tedy obsahují určitou základní sadu funkcí a dále lze funkčnost rozšiřovat pomocí instalace Pluginů. Ty mohou vytvářet přímo autoři nebo kdokoli jiný dle potřeb. Díky tomu jsou velmi robustní a podporují širokou škálu zařízení od různých výrobců napříč technologiemi a pokud ne, tak s trochou znalostí v programování si může každý dopsat plugin dle potřeb pro podporu daného zařízení.


\subsection{Známé Platformy}
Tato sekce s zabývá analýzou 4 vybraných Platforem.

\paragraph{Blynk}
Blynk se označuje jako harware-agnostic IOT Platforma s white-label mobilními aplikacemi \cite{blynk}. Umožnuje navrhnovat vlastní aplikace formou DragAndDrop pro ovládání zařízení, analýzu telemetrických dat a správu nasazených produktů ve velkém měřítku. Své řešení nabízejí jak pro domácí nasazení, tak i jako enterprise řešení pro větší firmy \cite{blynk}. Mají \textbf{3 cenové tarify} \cite{blynk-pricing}:
\begin{itemize}
    \item \textbf{Free} je omezené pouze pro osobní užití, obsahuje cloudový hosting, umožňuje připojit maximálně 5 zařízení zdarma a součástí je mobilní aplikace pro Android a iOS.
    \item \textbf{StartUp} je určeny pro komerční využití a cenou začínají na \$415/měsíc. Součástí je deployment vlastních aplikací na AppStore/Google Play, neomezený počet zařízení a uživatelů, garantované podpora
    \item \textbf{Business} začíná na \$1000/měsíc a nabízí navíc OTA aktualizace koncových zařízení (vzdáleně), webové rozhraní, datovou analýzou a dalších funkce.
\end{itemize}

Hardware-agnostic znamená, že nejsou omezeni pouze na určitý hardware a umožňují připojit v podstatě libovolné zařízení. Pro připojení maji definované rozhraní nad jednotlivými protokoly. Podporují custom TCP/IP, WebSocket, HTTP a nově i MQTT (zatím k němu nemají ale dokumentaci). Dávají k dispozici knihovný pro různé harwarové platformy, takže připojení k platformě je potom otázka dvou řádků kódu. K dispozici je velmi přehledná a detailní dokumentace.\cite{blynk-doc}

Nativní aplikace pro iOS a Android umožnuje vytvářet vlastní dashboardy pomocí již předpřipravených Widgetů, kterých je opravdu velké množství, ale jsou placené za tzv. Energii, což je měna která lze dobíjet za peníze, dále definovat vlastní widgety a upravit chování celé aplikace. Následně lze takto upravenou aplikaci vyexportovat a přímo nahrát na Google Play a AppStore pod vlastním názvem. Tento přístup nabízí elegantní možnost pro tvorbu vlastního řešení, které následně je možné nabízet jako vlastní produkt.\cite{blynk}

Výhodou cloudové řešení je přístup k platformě odkudkoliv z internetu. Zároveň je potom ale funčknost odkázána na dostupnost internetového připojení a představa dat někde v cloudu se nemusí líbit. Pro tento připad je možnost hostovat si vlastní Blynk server, který je dostupný jako OpenSource server napsaný v Javě. \cite{blynk-server}

\paragraph{Thingspeaks}
ThingSpeak™ je služba analytické IoT platformy od MathWorks®, tvůrců MATLAB® a Simulink®. Jedná se o hardware-agnostic platformu s webovým rozhraním, která se plně zaměřuje na analýzu dat. Je ideální pro lidi se zkušeností s Matlab, protože je postavena právě na této platformě. Umožňuje v cloudu sběr dat, jejich analýzu přímo pomocí Matlab kódu, vizualizaci dat a definování reakcí. Pro různé harwarové platformy mají připravené knihovny a nativě podporují komunikace pomocí protokolů HTTP a MQTT. \cite{thingspeaks}

Řešení nabízejí podle různých tarifů \cite{thingspeaks-pricing}, kde omezení jsou primárně ohledně maximálního počtu zpráv, počtu kanálů do kterého posílají zařízení zprávy a minimálního časového odstupu mezi zprávami v rámci jednoho kanálu. Dva základní tarify:
\begin{itemize}
    \item \textbf{Free} ~8 200 messages/day, počet kanálů 4, interval mezi zprávami 15s, Matlab maximální doba běhu 20s.
    \item \textbf{STANDARD} ~90 000 messages/day, počet kanálů 250, interval mezi zprávami 60s, Matlab maximální doba běhu 20s.
\end{itemize}

\paragraph{Home Assistant}
OpenSource domací automatizace, která dává lokální kontrolu a soukromí na první místo - takto se prezentuje Home Assistant. Tato platforma není tolik zaměřena na koncová zařízení jako předchozí, ale funguje jako integrátor komerčních/OpenSource řešení pod jednotné rozhraní. Obsahuje systém pro tvorbu automatizace, tedy vytváření reakcí na jednotlivé akce. Dokáže se napojit buď přímo na jednotlivá zařízení nebo na jejich Bridge a umožnit ovládání všech zařízení od různorodých výrobců, kteří často vynucují použití vlastní aplikace, pod jednotné rozhraní jak webové, tak ve formě nativní aplikace. Integrace je řešena pomocí pluginární systému, kde zpravidla jeden plugin obsahuje integraci pro jednoho výrobce/jeden Bridge. Většina pluginů vzniká přímo od komunity této platformy. V době psání této práce obsahuje 1743 pluginů.\cite{ha}

Celá platforma je zdarma a pro její zprovoznění stačí Raspberry Pi, na SD kartu nahrát předpřipravený image a zapnout. Prvnotním nastavením Vás následně provede webové rozhraní nebo nativní aplikace, záleží na Vaší volbě.\cite{ha-doc}

Platforma podporuje velké množství komerčních produkté mezi nejznámější patří Ikea TRÅDFRI, Philips Hue či Google Assistant \cite{ha-integrations}. Samozřejmně podporují i OpenSource řešení mezi nejznámější patří ESPHome \cite{esphome}, což je framework pro konfiguraci ESP chipů (ESP8266/ESP32), který řeší vrstvu komunikace a zapojení do platformy - stačí pouze dodefinovat chování na určité události a chytré zařízení je připravené.

\paragraph{OpenHAb}
%projekt s dlouho historií (v1 in 2010), lepší dokumentace, trochu složitější nastavení, mohutnější ale více možnost oproti HA, systém Addonů 324 aktuálně
OpenHab je OpenSource projekt s dlouhou historií, který vznikl již v roce 2010. Cílí na stejný segment jako Home Assistant, tedy  propojení existujících řešení pod jednotné rozhraní a jejich automatizaci. Jedná se o hardware agnostic platformu, která komunikuje přímo s koncovými zařízeními nebo příslušným Bridge. V základu obsahuje více funkcionalit, zatímco Homa Assistant je více minimalistický. Pro rozšířování funkcnionality používá systém doplňků (aktuálně 324 \cite{openhab-addons}), které vyvájejí autoři a především komunita. Od prvopočátku projektu je zde kladen velký důraz na nativní aplikace na rozdíl od Home assistantu, který dlouhou dobu neměl oficiální aplikaci pro Android. Webové rozhraní je samozřejmostí. Velkou výhodou je možnost využití cloud instance zcela zdarma, buď jako plnohodnotnou platformu nebo pouze pro přístup z internetu k vlastní instanci.\cite{openhab}

Prvotní instalace je stejně jednoduchá jako u Home Assistentu. Rozdíl přichází při přidávání jednotlivých zařízení, kde je proces trochu komplikovanější. OpenHab se snaží nabídnout pokročilejší funkcionalitu, která bohužel částečně zesložiťuje jednotlivé procesy. Na druhou stranu umožňuje větší flexibilitu.\cite{openhab-doc}

Dokumentace projektu je na velmi vysoké úrovni s velmi detailním popisem. Pravděpodobně díky tomu, že projekt existuje již 10 let a má silnou základnu v komunitě i přes to, že dle porovnání aktivity na GitHubu v počtu přispěvatelů (86 vs. 2 444) je oproti té, kterou má Home Assistant mnohonásobně menší.

\subsection{Porovnání}
% TODO závěrečné porovnání - Blynk více orientované na koncová zařízení, ThingSpeaks primárně pro analýzu dat s Matlabem, HA a openHab jsou HUBy pro domácí automatizaci 
Jednotlivé Platforma se některými funkcemi překrývají a v jiných jsou zase jedinečné. Při výběru je důležité si stanovit na co Platformu chceme využívat a jaké funkce vyžadujeme.

Blynk primárně cílí na podnikatelský segment, a nejvíce se hodí firmám, kteří chtějí na této Platforma vysvtavět své řešení, které následně budou přeprodávat pod svojí vlastní značkou. To díky příme možnosti exportu aplikace na AppStore a Google Play, hromadné zprávě zařízení a ACL (seznam oprávnění vázaný k zařízení, který specifikuje kdo k němu může přistupovat a jaké operace provádět). Kvůli chybějící podpoře komerčních zařízení lze využít pro domácnost pouze s DYI zařízeními.

ThingSpeaks míří primárně na zpracování dat díky svému ekosystému postaveném kolem MATLAB®. Pro veškeré zpracování, analýzi a vizualizace stačí znalost prostředí MATLAB®, který je světově známý a velmi oblíbený mezi akademiky.

Home Assistant je progresivní OpenSource Platforma, která umožní integraci komerčních řešení pod jednotné rozhraní a domácí automatizaci s příjemným uživatelským rozhraním.

OpenHab projekt s dlouho historií. Funkčně se velmi podobá Home Assistantu, ale snaží se uživatelům nabídnout více funkčnosti. Uživatelské rozhraní je občas trochu složitější.


\begin{center} % pro addony přidat poznámku 324 obsahuje 2585 věcí
    \begin{tabular}{ |c| m{5em}| m{5em}|m{5em}|m{4em}| m{5em}| m{4em}| m{4em}| }
        \hline
        Platforma      & Podpora komerčních produktů & Vlastní zařízení   & Hosting            & ACL              & Nativní aplikace         & Správa zařízení & Cena              \\
        \hline
        Blynk          & Ne                          & Ano                & self-hosted, cloud & Pouze Enterprise & Ano                      & Ano             & Omezený Free plan \\
        \hline
        ThingSpeaks    & 6 vendorů (primárně LoRa)   & Ano                & cloud              & Ano              & Pouze pro náhled na data & Ne              & Omezený Free plan \\
        \hline
        Home Assistant & pomocí Pluginů (1743)       & 3rd party knihovny & self-hosted        & Ano              & Ano                      & Ne              & Ano               \\
        \hline
        openHab        & pomocí Doplňků (324)        & 3rd party knihovny & self-hosted, cloud & Ne               & Ano                      & Ne              & Ano               \\
        \hline
    \end{tabular}
\end{center}

\subsection{Závěrečný verdikt}
Blynk je první platforma, se kterou jsem se střetl ve světě IoT před třemi lety a bohužel prvním dojem pro mě byl poměrně negativní. Mnohé se od té doby změnilo, ale nepřímá podpora MQTT protokolu a především nutnost platit řešení mě od této Platformi odrazuje. Thingspeaks je hezké řešení, které splňuje většinu mých představ, ale úzká integrace s MatLab a nutnost jeho znalosti pro zpracování dat, je pro mne překážkou ať už z hlediska, že MatLab nepoužívám, tak více z pohledu ceny MatLab prostředí a celého ekosystému. Sám se považuji za OpenSource zastánce a proto mě to táhne k těmto řešením. HomeAssistant je velmi progresivní a zajímáva platforma, která je ale primárně určena pro nasazení v lokální síti (nepočítá s nutností autentizace zařízení), zatímco já bych chtěl primárně Platformu provozovat jako řešení, kde se stačí zaregistrovat a každý kutil může přidávat vlastní zařízení a veškerý tok dat bude oddělen mezi uživateli. OpenHAB řešení mě velmi zaujalo, především možnost hostingu cloudového řešení, zcela zdarma. Bohužel chybějící ACL je pro mne nepřekonatelnou překážkou, protože chci platformu využívat pro více uživatelů a tedy definovat jednotlivá oprávnění. Proto jsem se rozhodl vytvořit si vlastní řešení, které mi dá prostor realizovat vše dle svých představ s důrazem na bezpečnost.

\section{Vlastní řešení}

\subsection{Koncepce}
Chtěl bych vytvořit otevřenou IoT Platformu, která bude určeta pro nejrůznější kutily s elektronikou a technické nadšence. K dispozici bude zdarma veřejná instance sloužící primárně pro uživatele, kteří si chtějí Platformu jednoduše a rychle vyzkoušet a nebo chtějí provozovat pouze pár zařízení. Pro ty kteří chtějí mít plnou kontrolu nad svými daty a být nezávislý na připojení k internetu, bude k dispozici možnost hostingu celého řešení na vlastním hardwaru. K Platformě půjde připojit různorodá zařízení a bude vytvořeno schéma pomocí kterého zařízení popíší Platformě vlastní funkčnost/schopnosti. Na základě těchto informací se automaticky uživateli vygeneruje webové rozhraní ke sledování a ovládání jeho zařízení.

Na bezpečnost bude kladený vysoký důraz. Primárně bude založená na uživatelských účtech, které budou mít oprávnění pouze ke svým zařízením, případně těm ke kterým dostali oprávnění od jiných uživatelů. Každý uživatel by měl mít k dispozici vlastní prostředí, v rámci kterého budou jeho zařízení komunikovat. Uživatel si bude moci snadno přihlásit k odběru této komunikace a tak sledovat všechny zprávy posílané mezí zařízeními a Platformou.


\subsection{Popis domény} % nativní podpora pro switch, sensor, generic - jednolivé popsat, abych se mohl na ně odkázat v UC
Tato kapitola obsahuje popis jednotlivých entit/pojmů se kterými Platforma pracuje.
\begin{figure}[htbp]
    \centering
    \includegraphics[width=\textwidth]{img/domain.pdf}
    \caption{Doména}
\end{figure}
\begin{itemize}
    \item \textbf{Uživatel (User)} - fyzická osoba, která interaguje se systémem přímo pomocí pomocí uživatelského rozhraní.
    \item \textbf{Zařízení (Device)} - fyzické zařízení, které komunikuje s Platformou, dále se dělí na Věci
    \item \textbf{Věc (Thing)} - logické uskupení Vlastností, např. meteostanice
    \item \textbf{Vlastnost (Property)} - určitá veličina, jejíž hodnota se odesílá na Platformu (např. teplota), která případně umožňuje být Platformou změněna/nastavena
    \item \textbf{Umístění (Location)} - kde je dané zařízení umístěno, specifikující budovu a místnost, např. doma-kuchyňe
    \item \textbf{Stav Věci (State)} - v jakém aktuálním stavu se určitá Věc nachází, skládá se ze stavů příslušných Vlastností
    \item \textbf{Notifikační pravidlo (Rule)} - za jaké podmínky se má uživateli odeslat notifikace
\end{itemize}


\subsection{Případy užití}
Tato kapitola popisuje identifikované případy užití, které současně slouží jako podklad funkčních požadavků kladených na řešení. Figurují v nich následující aktéři:
\begin{itemize}
    \item Uživatel - představuje autentizovaného uživatele webového rozhraní.
    \item Administrátor - představuje autentizovaného uživatele s nejvyšším stupněm oprávnění.
    \item Zařízení - představuje koncové zařízení komunikující s Platformou
\end{itemize}

\begin{figure}[htbp]
    \centering
    \includegraphics[width=0.7\textwidth]{img/use_case.pdf}
    \caption{Případy užití}
\end{figure}

\paragraph{UC1 Registrace uživatele}
- neautentizovaný uživatel vyplní registrační formulář obsahující jméno, přijmení, uživatelské jméno, heslo a email. Při zpracování požadavku na serveru bude zajištěna unikátnost uživatelského jména a emailu napříč databází. Uživatel bude informován o úspěchu/neúspěchu akce. Po úspěšné registraci bude automaticky přihlášen, pokud nezrušil ve formuláři zaškrtávátku \uv{Automaticky přihlásit}, a bude mu odeslán uvítací email na zadanoou emailovou adresu.

\paragraph{UC2 Obnovení hesla}
- součástí přihlašovacího formuláře bude odkaz na stránku pro obnovení zapomenutého hesla, kde bude uživatel dotázán na emailovou adresu, kterou použil při registraci. Po zadání, pokud daná emailová adresa je součástí některého uživatelské účtu, bude na ni odeslán email s odkazem pro obnovu hesla. Na tomto odkazu bude uživatel vyzván k zadání nového hesla.

\paragraph{UC3 Objevení a spárování nového zařízení}
- zařízení, pokud není ještě není spárované s Platformou, tak po zapnutí vyzve uživatele k zadání svéhu uživatelského jména na Platformě. Toto zadání bude umožněno vytvořením Wifi přístupového bodu, na kterém poběží kaptivní portál - po připojení telefonem/počitačem se automaticky zobrazí webová stránka s formulářem pro zadání údajů. Následně se zařízení připojí k Platformě, ohlásí jaké má Věci, popíše jejich Vlastnosti, a bude ji informovat, kterému uživateli (podle zadaného uživ. jména) má zobrazit možnost přidání nového zařízení. Pokud si uživatel dané zařízení přidá (viz. \hyperref[UC5]{UC5}), tak následně Platforma odešle zařízení API klíč, které si ho uloží a přihlásí se pomocí toho klíče k Platformě (nyní je spárované).

\paragraph{UC4 Popis věcí}
- zařízení při ohlašování definice věcí a jejich vlastností musí oznámit mimo jiné typ věci. Platforma bude podporovat kromně generického (generic) typu další 3 typy věcí, pro které bude speciální zobrazení v uživatelském rozhraním:
\begin{itemize}
    \item Switch - přepínač, který se nachází ve stavu on/off. V rozhraní bude Věc reprezentována dvou stavovým přepínačem, který při kliknutí odešle změnu o stavu na druhý než ve kterém se aktuálně nachází. (využití např. vypínač světla)
    \item Activator - spínač, který má pouze jeden stav. V rozhraní bude věc reprezentována tlačítkem, které na stisk odešlě aktivaci zařízení. (využití např. ovladač pojízdné brány)
    \item Sensor - v rozhraní bude reprezentován jako widget zobrazující aktuální hodnota první vlastnosti. Po rozkliknutí se zobrazí graf vizualizující průběh hodnoty v čase za posledních 24h.
    \item Generic - obecný typ u kterého zařízení popíše strukturu, datové typy a názvy příslušných vlastností. V uživatelském rozhraní bude věc reprezentována jako Widget, který po kliknutí zobrazí Dialogové okno umožňující zobrazení a ovládání všech vlastností dle konfigurace.
\end{itemize}


\paragraph{UC5 Správa zařízení}
\label{UC5}
- uživatel na stránce \uv{Správa zařízení} bude mít tyto dvě sekce:
\begin{itemize}
    \item \uv{Přidat zařízení} - zde se zobrazí všechna nově detekovaná zařízení, která ještě nemá přidaná. Následně při kliknutí na tlačítko přidat se zobrazí jednoduchý formulář pro zadání umístění a názvu zařízení - bude předvyplněn název, který ohlásilo zařízení. Uživatel formulář potvrdí, systém následně vytvoří dané zařízení, přidá uživateli k němu oprávnění a ná stránce \uv{Ovládání} už bude uživatel moci sledovat aktuální stav Věcí a případně je i ovládat (pokud to umožňují).
    \item  \uv{Správa} - zde budou zobrazena všechna zařízení, ke kterým má uživatel oprávnění. Rozhraní umožní pro každé zařízení, ke kterému má právo pro editaci, jeho smazání a pomocí formuláře editaci - názvu, umístění a změnu oprávnění pro jednotlivé uživatele. Tyto oprávnění budou rozděleny na tyto tři úrovně:
          \begin{itemize}
              \item \textbf{Read} - uživatel může si zobrazit veškeré údaje o zařízení.
              \item \textbf{Control} - uživatel může zařízení ovládat.
              \item \textbf{Write} - uživatel může editovat veškeré informace o zařízení (včetně oprávnění).
          \end{itemize}
\end{itemize}

\paragraph{UC6 Správa uživatelů}
- administrátor může přistoupit na stránku \uv{Správa uživatelů}, kde se mu zobrazí seznam všech registrovaných uživatelů. Jednotlivé uživatele může smazat a pomocí formuláře editovat všechny jejich osobní údaje včetně hesla.

\paragraph{UC7 Notifikace}
- rozhraní umožní uživateli nastavit pravidlo pro libovolnou věc, při kterém se odešlě Web Push notifikace na jeho zařízení. Pro nastavení bude zobrazený formulář umožňující výběr z vlastností dané věci, po vybrání se zobrazí výběr akce, při jejímž splnění chce uživatel obdržet notifikaci (překročení hodnoty / vždy / hodnota bude rovna) a případné pole pro zadání limitní hodnoty. Dále půjde zobrazit rozšířené nastavení pro konkrétní notifikační pravidlo umožňující nastavení času a konkrétních dní v týdnu, kdy bude pravidlo platné (ve výchozím stavu bude vždy). Těchto pravidel si bude moci nastavit libovolný počet pro každé zařízení, ke kterému má oprávnění pro čtení.



\subsection{Nefunkční požadavky}

\paragraph{N1 Řešení spustinelné na Linux systému}
- systém bude možno provozovat na Linuxovém serveru (Debian) a také na platformě Raspberry Pi (verze 3B+/4, OS Raspbian).

\paragraph{N2 Responzivní webové rozhraní}
- aplikace bude nabízet responzivní webové uživatelské rozhraní přizpůsobené pro zobrazení na mobilních zařízeních i stolních počítačích. Uživatelské rozhraní bude kompatibilní s prohlížeči Mozilla Firefox verze 80, Chrome verze 80 a Safari na iOS. Dále bude implementovat tzv. PWA (Progresivní webová aplikace) - bude využívat cache pro statické soubory pro rychlé načítání, spustitelné offline a na zařízení Android půjde v aplikaci chrome přidat na plochu a následně vypadat jako nativní aplikace.

\paragraph{N3 Rozhraní realizováno jako SPA}
- SPA (Single page application) je webová aplikace, která utilizuje JavaScript, aby při interakci v rámci aplikace se nemusela celá stránka načítat, ale pouze chytře překresluje potřebné části. Výsledkem je mnohem přijemnější uživatelský zážitek, než při čekání na stažení a překreslení celé stránku po kliknutí na odkaz.

\paragraph{N4 Validace}
Uživatel při vyplňování veškerých formulářů v uživatelském rozhraní obdrží interaktivní zpětnou vazbu v případě zadání nevalidních údajů. Interaktivní vazbou jsou myšleny následující scénáře při průchodu formuláře:
\begin{itemize}
    \item Zadání nové hodnoty a vykliknutí z pole - bude provedena validace a v případě nevalidního vstupu, bude uživatel vizuálně  upozorněn.
    \item Editace již zadané hodnoty v poli - validace bude provedena po každé změně (stisknutí klávesy), uživatel bude opět vizuálně upozorněn v případě nevalidního vstupu.
\end{itemize}

\paragraph{N5 Koncová zařízení na platformě ESP8266}
- pro jednotlivá zařízení bude použit čip ESP8266 od firmy Espressif jako mikrokontroler, který bude komunikovat po sítí prostřednictvím Wifi sítě.

\paragraph{N6 Výkonnostní požadavky}
- systém bude stabilní a zvládne obsluhovat stovku zařízení, kde každé bude odesílat změnu stavu s periodicitou 30 vteřin. Při tomto dlouhodobém zatížení nebude docházet k pádům systému ani k výraznému zpoždění komunikace (RESTful požadavky pod 200 ms).

\paragraph{N7 Konfigurace systému}
- veškerá konfigurace (týkající se externích služeb/komunikace) jako jméno a heslo do databáze, číslo portu pro komunikaci atd. bude konfigurovatelné pomocí promněných prostředí (env variables). Detailní popis promněných bude obsažen v instalační příručce.


\subsection{Vybrané technologie}

\subsubsection{Komunikační protokol}   %https://ieeexplore.ieee.org/abstract/document/8079928
Komunikační protokolů je velké množství a přímo závisí na výběru přenosného média. Při použití specializovaných sítí jako LoRa nebo Zigbee, nemáme moc velkou flexibilitu ve výběru. Zařízení podporující tyto specializované sítě jsou poměrně drahá, ale umožňují běh na baterii. Zatímco využití WiFi sítě nám dává obrovskou flexibilitu ve výběru protokolů a zařízení podporující připojení k Wifi nebyli nikdo více cenově dostupné jako dnes. Primárně z finančních nároků a možnosti využití stávající WiFi infrastruktury v domácnosti jsem se rozhodl pro WiFi jako bezdrátové médium. Současně díky přímé podpoře IP protokolu nebudu muset vytvářet bridge mezi serverem a sítí se zařízenímí jako např. při použití Bluethooth či LoRa.

%https://www.educba.com/mqtt-vs-websocket/
Z protokolů pro komunikaci je dnes neojoblíběnější HTTP, který ale nativně nepodporuje obousměrnou komunikaci, kdy zařízení může poslat zprávu serveru a stejně tak server zprávu zařízení, kterou pro ovládání zařízení budu potřebovat. Z obousměrných protokolů se velmi osvědčil WebSocket, který lze velmi snadno kombinovat s HTTP. Jedná se o protokol postavený nad TCP, který naváže spojení a obě strany mohou posílat zprávy \cite{websocket}. Je to velice hezké řešení pro posílání zpráv, ale neumožňuje nativně systematickou filtraci nebo odběr pouze určitých zpráv a nepočítá s během na nespolhlivých zařízeních. Proto přímo pro IOT vznikl otevřený síťový protokol MQTT jehož specifikaci vydává nezisková organizace \textit{OASIS}. Využívá asynchroní pattern publish-subsribe a byl speciálně navržen pro potřeby běhu na jednoduchých embeded zařízení s minimálním datovým tokem. Podrobný přehled všech protokolů využívaný pro IOT viz. \cite{protocols}.

\begin{figure}[htbp]
    \centering
    \includegraphics[width=0.7\textwidth]{img/mqtt-communication.jpeg}
    \caption{Ukázka komunikace MQTT \cite{img-mqtt-communication}}
\end{figure}

MQTT se vyvýjí již od roku 1999 a momentálně nejpoužívanější verzí je 3.1.1 pro kterou vznikla specifikace v roce 2014. Protokol primárně běží nad TCP/IP, ale lze využít v jakékoliv síti kde je zaručeno správné pořadí dat, beztrátovost a obousměrnost komunikace. Protokol definuje 2 typy entit: \uv{message broker} a klient. MQTT broker je server, který přijímá všechny zprávy od připojených klientů a přeposílá je příjemncům (klientům). MQTT klient je jakékoliv zařízení (od embeded až po server), které komunikuje s brokerem přes síť. \cite{mqtt}

Posílaná data jsou hierarchicky rozdělena do tzv. topiců (česky témat). Topic je textový řetězec o maximální délce 65536 Bytů s oddělovačem "/" - ukázka "/house/bedroom/light". Pokud klient chce odeslat (publish) data, tak pošle zprávu brokeru s daty a topicem do kterého zpráva patří. Broker potom zprávu odešle všem klientům, kteří jsou přihlášení k odběru (subscribe) z daného tématu. O odběr se klient musí přihlásit a to buď přímo specifikuje plný název topicu nebo částečný s použitím zástupných znaků. MQTT počítá s případnou nespolehlivostí ať koncových zařízení nebo sítě a proto umožňuje klientovy při přihlášení definovat \uv{Last Will and Testament} (\hypertarget{LWT}{LWT}). Při přihlášení klient oznámí téma a zprávu, která se odešla v případě nesprávně odpojeného klienta (výpadek sítě / chyba zařízení). Takto lze notifikovat, že došlo je ztráně spojení s daným klientem. \cite{mqtt}

Broker podporuje 3 třídy QoS (Quality of service), kterou lze specifikovat pro každou zprávu jednotlivě v závisloti na její důležitosti. Seřazeny jsou vzestupně dle náročnosti na systém (overhead) \cite{mqtt}:
\begin{itemize}
    \item \textbf{0 - Maximálně jednou} - zpráva je odeslána pouze jednou a klient ani broker nijak napotvrzují její přjetí
    \item \textbf{1 - Alespoň jednou} - zpráva je odeslaná několikanásobně, dokud není potvrzené její přijetí
    \item \textbf{2 - Právě jednou} - odesílatel a příjemnce navazují dvoucestný hand-shake, aby bylo zaručeno přijmutí zprávy právě jednou
\end{itemize}


\subsubsection{Výhody jednotného jazyku}
Využití jednotného jazyka pro vývoj Backendu a Frontendu přináší obrovskou výhodu v podobě možnosti sdílet nejenom definice pro objekty, ale i přímo části kódu. Toto je velmi vhodné například pro jednotné validace formulářů, různé datové transformace a sdílení aplikační logiky pro frontend \uv{optimistické aktualizace} (aktuální trend, nečekat na potvrzení požadavku ze serveru, ale rozhraní aktualizivat, jako by požadavek byl úspěšný a pouze v případě neúspěchu zobrazit stav ze serveru). Dále jednotný jazyk umožňuje programátorům při vývoji v případě potřeby pohodlně pracovat na obou částech aplikace aniž by se museli učit nový jazyk.


\subsubsection{Backend}    %https://nodejs.org/en/docs/
%spousta jazyků, pro koncepsi asynchroních messages MQTT a Websocket se hodí NodeJS
Vzhledem k povaze MQTT, který je koncepčně založený na asynchroních zprávách jsem si zvolil technologii NodeJS, která je postavené na asynchroní event-driven architektuře \cite{nodejs}. NodeJS je prostředí pro běh JavaScriptu na straně serveru, kterému se v posledních letech dostává velké pozornosti kvůli jeho oblíbenosti mezi vývojáři, je extrémně přívětivý k začínajícím programátorům a má pozvolnou křivkou učení. Díky své architektuře nabízí velice elegantní přístup pro zpracování akcí, kde se musí čekat na výsledek jako např. u síťové komunikace. V tradičním jazyce jako Java nebo C++ se toto čekaní musí řešit pracným vyvtvořením nového vlákna, které čeká na výsledek a následným zpracováním. V NodeJS je programátor od této problematiky odstíněn a může se tak plně věnovat tvorbě aplikační logiky aniž by měl znalosti a zkušenosti s více vláknovým programováním - kvůli této výhodě preferuji NodeJS kdykoliv kdy je potřeba řešit síťovou či asynchronní komunikaci.

%https://vegibit.com/what-is-nodejs/ 
\paragraph{NodeJS} má pravděpodobně aktuálně největší a nejaktivější komunitu ze všech jazyků. Pro správu knihoven používá balíčkovací systém npm (jsou i jiné alternativy) ze kterého se stal největší ekosystém na světě, který je zastřešený neziskovou společností \uv{npm, Inc.} provozující centrální repozitář se všemi dostupnými moduly pro NodeJS. Díky sve centralizaci je velmi jednoduchý na používání, ale v posledních letech kdy se NodeJS zpopularizoval a nyní se JavaScript stal jedním z nejoblíbenější jazyky \cite{survey-languages}, se ukázala centralizace jako poměrně nešťastné řešení, kvůli vysokým nákladům na provoz infrastruktury. Pro představu velikosti ekosystému: npm v roce 2020 obsahoval 1 200 000 modulů a druhý největší systém RubyGems \uv{pouhých} 350 000 \cite{modulecounts}. Všechny moduly jsou k dispozici zcela zdarma a díky takto aktivní komunitě lidí, kteří dávají k dispozici své knihovny ostatním, je vysoce pravděpodobné, že pokud chceme řešit nějaký problém, tak na něj již existuje knihovna.

\paragraph{TypeScript} je superset JavaScriptu, který navíc přidává komplexní typový systém \cite{ts} a rozhodl jsem se ho využít jako hlavní programovací jazyk pro frontend i backend. Jedná se o OpenSource jazyk vyvýjený společností Microsoft, který jeho vznikem chtěl usnadnit přechod C\# a .NET vývojářum k webovým aplikacím \cite{ts}. Mnoho lidí z JavaScript komunity považuje TypeScript jako kontroverzní počin, protože přidává složitost k velmi elegantnímu jazyku a zvyšuje časovou náročnost vývoje. Já jsem dlouho dobu tento názor také zastával, ale v posledních letech při práci na větších projektech a díky zkušeností z jiných jazyků (včetně striktně typových jako C++ a Java), jsem změnil svůj názor ve prospěch TypeScriptu. Souhlasím, že na první pohled prodlužuje dobu vývoje. Programátor musí psát věci navíc oproti čistému JavaScriptu, ale v dlouhodobém životním cyklu se tato práce \uv{na víc} mnohonásobně vrátí. A to v podobě statické kontroly typů, která minimalizuje riziko pádu aplikace a umožňuje  lepší statickou analýzu kódu, a dále jako největší přínost pro mne jako programátora TypeScript přináší funkční \uv{našeptávání} ve vývojovém prostředí, které pro JavaScriptu i přes veškeré snahy bohužel funguje ve velmi omezené míře.

%https://expressjs.com/
\paragraph{ExpressJS} je velmi minimalistický webový framework (první verze v roce 2010), který je do dnes velmi oblíbený a v mnohém ovlivnil vývoj většiny frameworků. Jeho největší výhoda je vysoká flexibilita. Nabízí pouze základní definici způsobu pracování s HTTP požadavky a možnost registrovat tzv. middleware - software, který rozšiřuje funkcionalitu. Veškerá funkcionalita je dodávána pomocí middlewarů, které jsou k dispozici jako moduly. Vývojář si tedy může výsledný server poskládat přesně dle svých představ, kterých existují desítky vytvořených přímo od autorů a další stovky od komunity. \cite{expressjs}


\paragraph{AgendaJS} je knihovna na perzistentní plánování úkolů (jobs) pro NodeJS \cite{agendajs}. Podporuje zpracování/plánování/perzistenci úkolů a jejich opětovné zpracování v případě chyby  \cite{agendajs}. Tato knihovna bude primárně využita pro zajištění odeslání emailů a pro spouštění případných plánovaných akcí. Proč v souvislosti s odesláním emailů? Jejich zpracování je závislé na třetí straně - emailovém serveru, který nemusí být vždy dostupný. Pokud systém bude mět odeslat email, tak tímto způsobem bude zajištěno, že i v případě selhání bude email opětovně odeslán jakmile bude možné.

\paragraph{Socket.IO} knihovna umožnující navázíní obousměrného spojení s kompatibilním klientem. Je založený na vlastním protokolu využívající HTTP spojení nebo WebSocket a vyznačuje se vysokou spolehlivostí a rychlostí umožňující real-time komunikaci. Jedná se o velice populární a časem ověřené řešení, které zajišťuje kompatibilitu i s prohlížeči nepodporující moderní technologii WebSocket.


\subsubsection{Frontend}
% React + Redux
Prvním světoznámím průkopníkem ve světě JavaScriptu pro tvorbu uživatelského rozhraní byla knihovna jQuery, která existuje do dnes, ale spíše se již považuje za přežitek doby. Dnes existuje obrovské množství Frameworků a knihoven pro tvorbu frontendu, ať pro tvorbu na straně serveru nebo přímo na straně uživatele v JavaScriptu. Trend dnešní doby je přesouvat generování rozhraní na stranu uživatele, jak kvůli snížení výkonostních nároků na server, tak spíše kvůli lepší odezvě a uživatelskému zážitků. Mezi nejznámější JavaScriptové frameworky patří bezpochyby Angular, Vue.js, Svelte a nesmím zapomenout na React, který je sice knihovna, ale řadí se na stejnou úroveň. Já jsem si zvolil jako hlavní prostředek pro tvorbu rozhraní React, právě proto že se jedná o knihovnu. Framework se vyznačuje tím, že vynucuje určité problémy řešit jistým způsobem bez možnosti volby. Má to své výhody a nevýhody a do větších týmu bych rozhodně volil raději Framework. Tento projekt ale budu vytvářet primárně sám a mám velice rád flexibilitu a možnost volby. V začátcích to bývá časově náročnější, ale vidím v tom obrovskou možnost osobního růstu, protože při každé volbě musím hodnoti výhody/nevýhody a nakonec retrospektivně vidím následky svých rozhodnutí. Mimo to je React vytvářet společností Facebook a používán největšími technologickými spočnostmi světa (Facebook, Nextflix a Airbnb \cite{react-companies}), takže je jistá jeho dlouhodobá podpora a od roku 2013, kdy byla vydána první verze je dobře odladěný a ověřený.

%https://reactjs.org/tutorial/tutorial.html#what-is-react
\paragraph{React} je deklarativní, efektivní a flexibilní JavaScriptová knihovna pro tvorbu uživatelského rozhraní. Kód dělí do malých izolovaných částí nazvaných \uv{komponenty}, které se skládání do sebe a mohou tvořit komplexní uživatelská rozhraní. Pro vysoký výkon využívá techniku virtuálního DOM - nejprve si vytvoří virtuální strom podoby rozhraní v paměti, který následně porovná s aktuální podobou vykreslenou v prohlížeči a zmanipuluje pouze ty části které se od posledního vykreslení změnily. Díky tomu je velice efektivní a dokáže vykreslovat komplexní stránky s obrovským množstvím dat. \cite{react}


%\section{Návrh uživatelského rozhraní}