\chapter{Analýza}
Tato kapitola se zabývá definici IOT Platformy a analýzy již existujících řešení. V závěru kapitoly je matice obsahující porovnání známých řešení vůči jejich funkcím.

\section{Definice IOT Platformy}
"IOT Platforma je více vrstvá technologie, která umožňuje přímočaré zajištění, ovládání a automatizaci připojených zařízení ve světě internetu věci. Zjednodušeně propojuje Váš hardware, jakkoli rozdílný, do cloudu s možností různorodé konektivity, obsahuje bezpečnostní mechanizmy a široké možnosti pro zpracování dat. Pro vývojáře, IOT Platforma nabízí soubor předpřipravených funkcí, které vysoce zvyšují rychlost vývoje aplikací pro připojená zařízení a řeší škálování a kompatibilitu napříč zařízeními" [*, překlad autora] %https://www.kaaproject.org/blog/what-is-iot-platform 

\subsection{Definice pojmů}
V této sekci jsou vysvětleny pojmy, které budou použity v následujících kapitolách.

\begin{itemize}
    \item \textbf{Platforma} - Platformou se rozumý programové řešení umožňující propojení různorodých zařízení a jejich následnou obsluhu
    \item \textbf{Koncové zařízení} - Zařízení, které dokáže komunikovat po sítí a nabízí nějakou funkcionality (např. měření teploty nebo ovládání světla)
    \item \textbf{Bridge} - Bridge se rozumý síťové zařízení, které funguje jako prostředník mezi koncovými zařízeními a jinou sítí. Agreguje zařízení a nabízí rozhraní pro komunikaci s nimi.
\end{itemize}

\section{Vlastnosti??}

\subsection{Komunikace}    % rozdíly v komunikačním mediu wifi, bluethoot, mesh řešení Z-wave a zigbee, LoRa, spotřeba, bateriový provoz, složitost instalace
Pro komunikaci mezi zařízeními se nejčastěmi používá bezdrátová komunikace, kvůli jednoduchosti instalace bez nutnosti většího zásahu do stávající infrastruktury. Tento typ lze rozdělit do dvou kategorií:
\begin{itemize}
    \item \textbf{Centralizované} - Každé zařízení komunikuje pouze s jedním centrálním prvkem, přes který jde veškerá komunikace. Mezi nejznámější technologii tohoto typu patří Wifi.
    \item \textbf{Decentrlizovanou} - V této síti komunikují zařízení přímo s ostatními bez jakéholiv prostředníka. Pokud zařízení nemohou komunikovat napřímo, tak využívají ostatní pro předání zprávy. Síť je díky tomu mnohem odolnější vůči výpadků, protože zde není tzv. "Single point of failure". Zpravidla mívá nižší datovou propustnost a je složitější pro nasazení a následnou správu. Velkou výhodou je snadnější rozšiřitelnost pokrytí, protože každé přidané zařízení rozšiřuje signál a tímto způsobem lze zařízení řetězit.
\end{itemize}
Vzhledem k rozšířenosti Wifi, kterou dnes najdeme v každé domácnosti, se přirozeně nabízí využít tuto možnost i pro internet věcí. A k tomu v posledních letech opravdu došlo. Díky extrémně levnému chipu ESP8266, který se dnes v ČR dá koupit za 40 Kč, došlo k masivní penetraci trhu s chytrými zařízeními využívající právě Wifi. Bohužel tato technologie má i svá negativa a největší je spotřeba elektrické energie a limit maximálního početu připojených zařízení na jeden centrální prvek (řádově desítky). Vysoká spotřeba je dána nutností časté komunikace jen kvůli udržení aktivního spojení a proto je možné provozovat zařízení na baterie pouze v jednotkách dní až týdnů.

Pro bateriový provoz vznikly speciální sítě, které sice na rozdíl od Wifi umožní přenos v desítka kp za sekundu (tisícina rychlosti běžné Wifi), ale jsou energeticky mnohem úspornější (umožnují provoz až desítky let na malou baterii), mají mnohonásobně větší dosah a umožňují propojení mnohem většího počtu zařízení (stovky).

% TODO - odkud je toto tvrzení převzato
Nejvíce rozšířenými z centrálně orientovaných sítí jsou SigFox a LoRa. SigFox je komerční řešení, kde se platí za každé připojené zařízení. Oproti tomu síť LoRa používá otevřený standard pro komunikaci LoRaWAN. Protože se jedná o otevřený standard, tak kdokoliv může vytvořit a provozovat kompatibilní zařízení. Samozřejmně také lze využití komerční infrastrukturu, kam lze připojit svá zařízení za poplatek, ale díky otevřenosti má každý možnost si za pár tisíc postavit vlastní GateWay (centrální prvek) a provozovat libovolná zařízení bez jakýhkoliv poplatků a prostředníků.

%https://thesmartcave.com/z-wave-vs-zigbee-home-automation/
Z decentralizovaných sítí se nejvíce ujaly Zigbee a Z-Wave. Zigbee je otevřený standard, který dokáže pracovat, jak v pásmu 2.4GHz, tak i 900 MHz. Nemá omezení na maximální počet zařízení zřetězených za sebou a dokáže vytvořit síť skládající se až ze 65 tisíc zařízení. Z-Wawe je naopak uzavřený standard, který funguje pouze v pásmu 800-900 MHz. Limituje maximální počet přeposlání zprávy na 4 a podporuje síť o velikosti až 256 zařízení. Obě sítě jsou energeticky velmi úsporné a umožňují běh zařízení na obyčejnou knoflíkovou baterii po dobu až několika let.

\subsection{Automatizace}
%https://www.iot-now.com/2020/06/10/98753-iot-home-automation-future-holds/
Automatizace je ve světě IoT pravděpodobně nejdůležitějším tématem a každá IOT Platforma by ji měla umožňovat, protože dává možnost využít zařízení úplně novým způsobem. Principiálně se jedná o možnost definování reakcí na jednotlivé události. Událost můžemý být změna teploty, otevření okna nebo detekce pohybu a reakce změna stavu zařízení - zhasnutí světla nebo zapnutí televize. V podstatě jediným limitem je zde lidská představivost. Modelový scénář:

Představme si moderní dům, ve kterém jsou všechny věci, které nás napadnou, chytré, což s dnešními technologickými možnostmi není sci-fi, ale naopak možná realita. Majitel domu, říkejme mu Joe, přichází večer unavený domů a odemyká dveře. Vejde do vnitř a světlo na chodbě a v kuchyni již svítí. Jde přímo do kuchyně, protože po dlouhém dni v práci má hlad a usedá s jídlem ke stolu. Nemá rád ticho, tak řekne: ,,Alexo, zapni hudbu" a ze Sterea se spustí Beethoven, protože Alexa ví, že je to Joeův oblíbený skladatel klasické hudby. Joe cítí jak se po místnosti rozprostřívá příjemné teplo ze zapnuté klimatizace. Po příjemné večeři odchází do druhého patra do koupelny, samozřejmně se nemusí starat o zapnuté Stereo ani světla, protože se vše samo vypne, jakmile odejde. Ve sprše pustí vodu, která má automaticky teplotu nastavenou specificky dle Joeovi preference 36 °C i přes to, že 20 min předním se sprchovala jeho přítelkyně, která si libuje v teplejší vodě. Po sprše jde do ložnice a ulehá do postele zatímco se kontroluje, jestli jsou všechny dveře zamčené, okna zavřená a zapíná se alarm pro případný pohyb ve spodním patře. A jak mohlo být vše uzpůsobené Joeovím preferencím a vše zapnuté ještě před jeho vstupem do domu? Protože zvonek u dveří má kameru s rozpoznáváním obličeje - Joea tedy poznal a vše nastavil.

Takto tedy může vypadat automatizace v domácnosti, která zpříjemní život a odprostí Vás od spousty všedních věcí. Vše nastavené dle osobních preferencí a to nejen určité rodiny ale na úrovni jednotlivců v domácnosti.


\subsection{Bezpečnost a soukromí}
Při výběru Platformy by důležitým kritériem měla být bezpečnost. Na první pohled se to nemusí zdát být důležité, co se může stát když bude s platformou komunikovat čidlo pohybu a někdo se dokáže dostat k těmto údajům? Například pro zloděje mohou být taková data zlatý důl, protože bude přesně vědět kdy je dům prázdný.

Bezpečnost je potřeba zde sledovat hned na několika faktorech. Prvním je komunikační médium. Pokud zařízení komunikují bezdrátově, tak by komunikace měla být šifrovaná, aby se nedala jednoduše odposlechnout. Druhým faktorem je bezpečnost samotné platformy. Pokud bude platforma dostupná pouze na interní síťi v domácnosti, tak bezpečnost na první pohled ohrožená není, pokud se ale zamyslíme nad tím, kolik dnes doma máme chytrých zařízení, tedy takových, která dokáží komunikovat přes internet, tak zjistíme že jich je velké množství, protože dnes už takovou chytrou televizi má doma téměř každý a je otázka na kolik věříme výrobcům těchto zařízení, že kladou důraz na jejich bezpečnost. Stačí aby nějaký vir napadl naší televizi či jiné zařízení a případný útočník má plný přístup k platformě pouze získáním přístupu do interní sítě. Proto by platforma měla využívat alespoň systém pro identifikaci, ideálně i autentifikaci a to nejen v případě, že je přístupná z internetu ale i z vnitřní sítě.

\subsection{Cílová skupina}
Internet věcí lze využít napříč všemi sférami. Od jednoduché meteostanice, která bude měřit venku teplotu, přes tvz. chytrou domácnost, kde Vám lednička pošle nákupní seznam na email podle chybějících potravin, přes využití v průmyslu pro sběr různorodých dat a jejich následnou analýzu ať pro zvýšení kvality nebo detekci poruchy, ještě před tím než k ní dojde. Tato práce se zaměřuje na využití IoT v běžné domácnost a implementací Platformy určené pro kutily a technické entusiasty, kteří chtějí mít svá data pod kontrolou a vytvářet různorodá zařízení, která si k Platformě připojí.


% \section{Procesy} % základní features z pohledu uživatele
% Tato sekce obsahuje popis 3 základních procesů, se kterými uživatel přijde do styku.

% \subsection{Přidání zařízení}%jak probíhá přidání zařízení
% Proces přidání nového zařízení byl měl být pro uživatele co možná nejjednoduší, protože první interakce uživatele se zařízením je právě počáteční zprovoznění, při kterém si uživatele vytváří názor, který se zpětně velmi těžko mění. První dojem by měl tedy být co možná nejvíce přívětivý.

% \subsection{Interakce}% jak zobrazit data a měnit stav
% Následná interakce s jednotlivými zařízení přes Platformu musí být intuitivní a nabídnout uživateli prostředí, které nebude zbytečně komplikované a umožní mu snadné zobrazení dat formou vizualizací.

% \subsection{Automatizace}% jak definovat reakce/schémata
% Platforma by měla nabídnout uživateli možnost definování reakcí na různé akce. Od jednoduchý až po složitější skládání toků pro vizuální programování.

\section{Existující řešení}
Tato kapitola se zabývá pohledem na aktuálnní řešení jak Komerční, tak i OpenSource. Poukazuje na výhody a nevýhody z obou světů, následně se zaměřuje na analýzu 4 konkrétních Platforem a jejich násleným porovnáním.
\subsection{Komerční řešení} % hotové řešení, cloud, ale drahé
%easy to use, but paid
Na trhu dnes existuje velké množství komerčních řešení od známých výrobců. Někteří jsou známí spíše výrobou harwaru jako Philips a Xiaomi, jiní se zaměřují spíše na nabídku služeb a integraci zařízení ostatních výrobců pod svojí Platformu jako Amazon nebo Google. Pro koncového zákazníka mají Komerční řešení obrovskou výhodu v jednoduchosti nasazení a následné obsluhy. Stačí zakoupit centrální jednotku, libovolná zařízení od stejného výrobce a vše krásně funguje. Avšak problém nastává ve chvíli, kdy potřebují řešení škálovat či customizovat dle svých potřeb, protože si dodavatel za úpravy na \uv{míru} začne účtovat obrovské částky a zákazníkovi nezbývá nic jiného než platit. Sám si potřebné změny udělat nemůže, protože nemá zdrojové kódy a migrace k jinému produktu by znamenal obrovské náklady a problémy se stávájícími integracemi, protože různá řešení mívají různá rozhraní.

%security
Aspekt bezpečnosti u uzavřených řešení bývá diskutabilní. Pravidelné bezpečnostní audity kvůli vysokým nákladům provádí málo kdo. Výrobci samozřejmně vždy tvrdí, že bezpečnost je u nich na prvním místě, ale bohužel tento aspekt je v přímém kontrastu s jenododuchostí použití, což je pro výrobce mnohem důležitější protože pokud se řešení dobře a jednoduše ovládá, tak mnohem spíše si ho zákazníci oblíbí, než pokud bude maximálně zabezpečeno, ale uživatel bude muset provádět úkony návíc čistě kvůli bezpečnosti, která mu na první pohled nepřínáší přidanou hodnotu.

%Cloud dependent
Od Platformy očekáváme možnost vzdáleného ovládání, tedy přístup odkudkoli z internetu. Málokdo má však doma veřejnou IP adresu, aby si mohl celé řešení provozovat doma \uv{self-hosted}. V Praxi si tedy uživatel pořidí domů Bridge, který komunikuje s chytrými zařízeními v domácnosti a současně s cloudem výrobce, přes který lze přistupovat na Platformu a ovládat všechny zařízení. Takové řešení se velmi osvědčilo díky jednoduchosti, protože neklade žádné nároky na uživatele jako např. veřejnou IP adresu. Problém však může nastat ve chvíli, kdy výrobce daného řešení po několika letech ukončí činnost a s tím přestane provozovat svojí cloudovou infrastrukturu, na které je závislý Bridge a vzdálený přístup z internetu. V lepším případě bude zachována funkčnost v lokální sití, v horším přestane řešení fungovat úplně. Najednou uživatelovi zbyde doma spousta funkčního (po fyzické stránce) harwaru, který nemůže využívat.

Výše jsem nastínil nejhorší možný scénář, který naštěstí v poslední době již přestává platit, protože výrobci společně vytvářejí otevřené standardy pro komunikaci, které by měli zaručit kompatibilitu zařízení napříč jednotlivými výrobci. Bohužel standardů vzniká současně více a ne všichni je plně dodržují, takže nekompatibilita ještě bude delší dobu přetrvávat i když ne v takovém měřítku jako před pár lety. Kromně rozdílných protokolů je také nekompatibilita v různých technologiích přenosu mezi nejznámější patří WiFi, Bluethooth, LoRa, Zigbee a Sigfox.
%podpora jiných výrobců? integrace? -> závislé na tom co výrobce se rozhodne implementovat

\subsection{OpenSource řešení}
% nepopulární/špatná reputace mezi lidmy, často potřeba znalosti problematiky, Free, flexibilní, customizovatelné
OpenSource řešení mají mezi širší veřejností špatnou reputaci, protože na rozdíl od komerčních \uv{Plug\&Play} produktů většinou vyžadují určité povědomí o dané problematice. Je to způsobeno tím, že se snaží pokrýt celou doménu stejně jako komerční řešení, ale oproti nim se zlomkem vývojářů a financí. Následkem toho není prvotní nastavení pro laika zcela přímočaré a může se střetnout s problémy. Avšak překonání prvotních nesnázích přináší následně spoustu pozitiv.

Jedním z nejatraktivnějších lákadel je samozřejmně cena. OpenSource řešení jsou zpravidla zcela zdarma, případně nabízejí placenou podporu. Mě osobně na OpenSource nejvíce zaujala komunita. Pokud se projekt dostane do určité známosti, tak kolem něho začně vznikat komunita lidí, primárně technologických nadšenců ale i lidí z IT praxe, kteří mezi sebou komunikují a spolupracují na vylepšení daného řešení, ať už přímo (napsání části funkcionality) nebo nepřímo (komunikace s vývojáři). Potom i obyčejný uživatel, který chce řešení využít, tak při objevení potíží, může požádat komunitu o pomoc a protože to jsou nadšení lidé, jsou velmi ochotní.

Pokud máme dostatečné technické znalosti, tak si můžeme prohlédlou přímo zdrojové kódy a sami si zhodnotit kvalitu i bezpečnost. U větších projektů to však již není tak úplně možné při desítkách tisíc rádků kódu, ale existují lidé, kteří tomu opravdu věnují čas a mohou tak objevit zranitelnosti. Dále OpenSource projekty bývají mnohem více sdílné ohledně architektury kterou využívají a je možno se v dokumentaci dočíst, jak vlastně řešení funguje interně, na rozdíl od komerčních, kde je to tzv. \uv{BlackBox}.

OpenSource platformy bývají postavené na systému Pluginů, tedy obsahují určitou základní sadu funkcí a dále lze funkčnost rozšiřovat pomocí instalace Pluginů. Ty mohou vytvářet přímo autoři nebo kdokoli jiný dle potřeb. Díky tomu jsou velmi robustní a podporují širokou škálu zařízení od různých výrobců napříč technologiemi a pokud ne, tak s trochou znalostí v programování si může každý dopsat plugin dle potřeb pro podporu daného zařízení.


\subsection{Známé Platformy}
Tato sekce s zabývá analýzou 4 známích Platforem a v závěru jejich vzájemným porovnáním.

\paragraph{Blynk}
Blynk se označuje jako harware-agnostic IOT Platforma s white-label mobilními aplikacemi. Umožnuje navrhnovat vlastní aplikace formou DragAndDrop pro ovládání zařízení, analýzu telemetrických dat a správu nasazených produktů ve velkém měřítku. Své řešení nabízejí jak pro domácí nasazení, tak i jako enterprise řešení pro větší firmy. Mají \textbf{3 cenové tarify}:
\begin{itemize}
    \item \textbf{Free} je omezené pouze pro osobní užití, obsahuje cloudový hosting, umožňuje připojit maximálně 5 zařízení zdarma a součástí je mobilní aplikace pro Android a iOS.
    \item \textbf{StartUp} je určeny pro komerční využití a cenou začínají na \$415/měsíc. Součástí je deployment vlastních aplikací na AppStore/Google Play, neomezený počet zařízení a uživatelů, garantované podpora
    \item \textbf{Business} začíná na \$1000/měsíc a nabízí navíc OTA aktualizace koncových zařízení (vzdáleně), webové rozhraní, datovou analýzou a dalších funkce.
\end{itemize}

Hardware-agnostic znamená, že nejsou omezeni pouze na určitý hardware a umožňují připojit v podstatě libovolné zařízení. Pro připojení maji definované rozhraní nad jednotlivými protokoly. Podporují custom TCP/IP, WebSocket, HTTP a nově i MQTT (zatím k němu nemají ale dokumentaci). Dávají k dispozici knihovný pro různé harwarové platformy, takže připojení k platformě je potom otázka dvou řádků kódu. K dispozici je velmi přehledná a detailní dokumentace.

Nativní aplikace pro iOS a Android umožnuje vytvářet vlastní dashboardy pomocí již předpřipravených Widgetů, kterých je opravdu velké množství, ale jsou placené za tzv. Energii, což je měna která lze dobíjet za peníze, dále definovat vlastní widgety a upravit chování celé aplikace. Následně lze takto upravenou aplikaci vyexportovat a přímo nahrát na Google Play a AppStore pod vlastním názvem. Tento přístup nabízí elegantní možnost pro tvorbu vlastního řešení, které následně je možné nabízet jako vlastní produkt.

Výhodou cloudové řešení je přístup k platformě odkudkoliv z internetu. Zároveň je potom ale funčknost odkázána na dostupnost internetového připojení a představa dat někde v cloudu se nemusí líbit. Pro tento připad je možnost hostovat si vlastní Blynk server, který je dostupný jako OpenSource server napsaný v Javě. %https://github.com/blynkkk/blynk-server

\paragraph{Thingspeaks}
ThingSpeak™ je služba analytické IoT platformy od MathWorks®, tvůrců MATLAB® a Simulink®. Jedná se o hardware-agnostic platformu s webovým rozhraním, která se plně zaměřuje na analýzu dat. Je ideální pro lidi se zkušeností s Matlab, protože je postavena právě na této platformě. Umožňuje v cloudu sběr dat, jejich analýzu přímo pomocí Matlab kódu, vizualizaci dat a definování reakcí. Pro různé harwarové platformy mají připravené knihovny a nativě podporují komunikace pomocí protokolů HTTP a MQTT. Řešení nabízejí podle následujících tarifů, kde omezení jsou primárně ohledně maximálního počtu zpráv, počtu kanálů do kterého posílají zařízení zprávy a minimálního časového odstupu mezi zprávami v rámci jednoho kanálu.
\begin{itemize}
    \item \textbf{Free} ~8 200 messages/day, počet kanálů 4, interval mezi zprávami 15s, Matlab maximální doba běhu 20s
    \item \textbf{STANDARD} ~90 000 messages/day, počet kanálů 250, interval mezi zprávami 60s, Matlab maximální doba běhu 20s
    \item TODO matice pro tarify jednotlivé?
\end{itemize}

\paragraph{Home Assistant}
OpenSource domací automatizace, která dává lokální kontrolu a soukromí na první místo - takto se prezentuje Home Assistant. Tato platforma není tolik zaměřena na koncová zařízení jako předchozí, ale funguje jako integrátor komerčních/OpenSource řešení pod jednotné rozhraní. Obsahuje systém pro tvorbu automatizace, tedy vytváření reakcí na jednotlivé akce. Dokáže se napojit buď přímo na jednotlivá zařízení nebo na jejich Bridge a umožnit ovládání všech zařízení od různorodých výrobců, kteří často vynucují použití vlastní aplikace, pod jednotné rozhraní jak webové, tak ve formě nativní aplikace. Integrace je řešena pomocí pluginární systému, kde zpravidla jeden plugin obsahuje integraci pro jednoho výrobce/jeden Bridge. Většina pluginů vzniká přímo od komunity této platformy. V době psání této práce obsahuje 1743 pluginů.

Celá platforma je zdarma a pro její zprovoznění stačí Raspberry Pi, na SD kartu nahrát předpřipravený image a zapnout. Prvnotním nastavením Vás následně provede webové rozhraní nebo nativní aplikace, záleží na Vaší volbě.

Platforma podporuje velké množství komerčních produkté mezi nejznámější patří Ikea TRÅDFRI, Philips Hue či Google Assistant. Samozřejmně podporují i OpenSource řešení mezi nejznámější patří ESPHome, což je framework pro konfiguraci ESP chipů (ESP8266/ESP32), který řeší vrstvu komunikace a zapojení do platformy - stačí pouze dodefinovat chování na určité události a chytré zařízení je připravené.

\paragraph{OpenHAb}
%projekt s dlouho historií (v1 in 2010), lepší dokumentace, trochu složitější nastavení, mohutnější ale více možnost oproti HA, systém Addonů 324 aktuálně
OpenHab je OpenSource projekt s dlouhou historií, který vznikl již v roce 2010. Cílí na stejný segment jako Home Assistant, tedy  propojení existujících řešení pod jednotné rozhraní a jejich automatizaci. Jedná se o hardware agnostic platformu, která komunikuje přímo s koncovými zařízeními nebo příslušným Bridge. V základu obsahuje více funkcionalit, zatímco Homa Assistant je více minimalistický. Pro rozšířování funkcnionality používá systém doplňků (aktuálně 324), které vyvájejí autoři a především komunita. Od prvopočátku projektu je zde kladen velký důraz na nativní aplikace na rozdíl od Home assistantu, který dlouhou dobu neměl oficiální aplikaci pro Android. Webové rozhraní je samozřejmostí. Velkou výhodou je možnost využití cloud instance zcela zdarma, buď jako plnohodnotnou platformu nebo pouze pro přístup z internetu k vlastní instanci.

Prvotní instalace je stejně jednoduchá jako u Home Assistentu. Rozdíl přichází při přidávání jednotlivých zařízení, kde je proces trochu komplikovanější. OpenHab se snaží nabídnout pokročilejší funkcionalitu, která bohužel částečně zesložiťuje jednotlivé procesy. Na druhou stranu umožňuje větší flexibilitu.

Dokumentace projektu je na velmi vysoké úrovni s velmi detailním popisem. Pravděpodobně díky tomu, že projekt existuje již 10 let a má silnou základnu v komunitě i přes to, že je přibližně poloviční oproti té, kterou má Home Assistant.

\subsection{Porovnání}
% TODO závěrečné porovnání - Blynk více orientované na koncová zařízení, ThingSpeaks primárně pro analýzu dat s Matlabem, HA a openHab jsou HUBy pro domácí automatizaci 
Jednotlivé Platforma se některými funkcemi překrývají a v jiných jsou zase jedinečné. Při výběru je důležité si stanovit na co Platformu chceme využívat a jaké funkce vyžadujeme.

Blynk primárně cílí na podnikatelský segment, a nejvíce se hodí firmám, kteří chtějí na této Platforma vysvtavět své řešení, které následně budou přeprodávat pod svojí vlastní značkou. To díky příme možnosti exportu aplikace na AppStore a Google Play, hromadné zprávě zařízení a ACL (seznam oprávnění vázaný k zařízení, který specifikuje kdo k němu může přistupovat a jaké operace provádět). Kvůli chybějící podpoře komerčních zařízení lze využít pro domácnost pouze s DYI zařízeními.

ThingSpeaks míří primárně na zpracování dat díky svému ekosystému postaveném kolem MATLAB®. Pro veškeré zpracování, analýzi a vizualizace stačí znalost prostředí MATLAB®, který je světově známý a velmi oblíbený mezi akademiky.

Home Assistant je progresivní OpenSource Platforma, která umožní integraci komerčních řešení pod jednotné rozhraní a domácí automatizaci s příjemným uživatelským rozhraním.

OpenHab projekt s dlouho historií. Funkčně se velmi podobá Home Assistantu, ale snaží se uživatelům nabídnout více funkčnosti. Uživatelské rozhraní je občas trochu složitější.


\begin{center} % pro addony přidat poznámku 324 obsahuje 2585 věcí
    \begin{tabular}{ |c| m{5em}| m{5em}|m{5em}|m{4em}| m{5em}| m{4em}| m{4em}| }
        \hline
        Platforma      & Podpora komerčních produktů & Vlastní zařízení   & Hosting            & ACL              & Nativní aplikace         & Správa zařízení & Cena              \\
        \hline
        Blynk          & Ne                          & Ano                & self-hosted, cloud & Pouze Enterprise & Ano                      & Ano             & Omezený Free plan \\
        \hline
        ThingSpeaks    & 6 vendorů (primárně LoRa)   & Ano                & cloud              & Ano              & Pouze pro náhled na data & Ne              & Omezený Free plan \\
        \hline
        Home Assistant & pomocí Pluginů (1743)       & 3rd party knihovny & self-hosted        & Ano              & Ano                      & Ne              & Ano               \\
        \hline
        openHab        & pomocí Doplňků (324)        & 3rd party knihovny & self-hosted, cloud & Ne               & Ano                      & Ne              & Ano               \\
        \hline
    \end{tabular}
\end{center}

\subsection{Závěrečný verdikt??}
Blynk je první platforma, se kterou jsem se střetl ve světě IoT asi před dvěma roky a bohužel prvním dojem pro mě byl poměrně negativní. Mnohé se od té doby změnilo, ale nepřímá podpora MQTT protokolu a především nutnost platit řešení mě od této Platformi odrazuje. Thingspeaks je hezké řešení, které splňuje většinu mých představ, ale úzká integrace s MatLab a nutnost jeho znalosti pro zpracování dat, je pro mne překážkou ať už z hlediska, že MatLab nepoužívám, tak více z pohledu ceny MatLab prostředí a celého ekosystému. Sám se považuji za OpenSource zastánce a proto mě to táhne k těmto řešením. HomeAssistant je velmi progresivní a zajímáva platforma, která je ale primárně určena pro nasazení v lokální síti (nepočítá s nutností autentizace zařízení), zatímco já bych chtěl primárně Platformu provozovat jako řešení, kde se stačí zaregistrovat a každý kutil může přidávat vlastní zařízení a veškerý tok dat bude oddělen mezi uživateli. OpenHAB řešení mě velmi zaujalo, především možnost hostingu cloudového řešení, zcela zdarma. Bohužel chybějící ACL je pro mne nepřekonatelnou překážkou, protože chci platformu využívat pro více uživatelů a tedy definovat jednotlivá oprávnění. Proto jsem se rozhodl vytvořit si vlastní řešení, které mi dá prostor realizovat vše dle svých představ s důrazem na bezpečnost.

